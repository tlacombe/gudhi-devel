\documentclass[a4paper, 11pt]{article}

%\usepackage{amsmath}
\usepackage{graphics}
\usepackage{epsfig}
%\usepackage{ipe} 
\usepackage{latexsym}
\usepackage{amssymb}
\usepackage{color}
%\usepackage{here}
\usepackage{url}
     
\parindent 0pt
\parskip 3mm

\pagestyle{headings}

\setlength{\textwidth}{18cm}
\setlength{\evensidemargin}{-1cm}
\setlength{\oddsidemargin}{-1cm}

\newcommand{\bblue}[1]{{\color{blue} #1}}
\newcommand{\rred}[1]{{\color{red} #1}}
\newcommand{\ggreen}[1]{{\color{green} #1}}
      
\newcommand{\svsp}{\vspace{3mm}}
\newcommand{\vsp}{\vspace{5mm}}
\newcommand{\bvsp}{\vspace{10mm}}
\newcommand{\ind}{\mbox{}\hspace{2cm}}
\newcommand{\sind}{\mbox{}\hspace{1cm}}
      
\newcommand{\B}{\condmath{I\!\!B}}    
\newcommand{\D}{\mathbb D}              
\newcommand{\E}{\mathbb E}            
\newcommand{\F}{\mathbb F}            
%\newcommand{\Id}{\condmath{I\!\!I}}                  
\newcommand{\I}{\mathbb I}
\newcommand{\M}{\mathbb M}
\newcommand{\N}{\mathbb N}
\newcommand{\Q}{\mathbb Q}
\newcommand{\R}{\mathbb R}
\renewcommand{\S}{\mathbb S}
\newcommand{\Z}{\mathbb Z}


\newcommand{\aaa}{\condmath{\cal A}}            
\newcommand{\cc}{\condmath{\cal C}} 
\newcommand{\e}{\condmath{\varepsilon}}   
\newcommand{\lcal}{{\condmath{\cal L}}      }
\newcommand{\mm}{{\condmath{\cal M}}      }
\newcommand{\pp}{\condmath{{\cal P}}}
\newcommand{\sss}{{\condmath{\cal S}}}
\newcommand{\vv}{\condmath{\cal V}}            
\newcommand{\tc}{\condmath{\cal T}}   

%\newcommand{\+}{{\footnotesize +}}   
\newcommand{\+}{\mbox{{\tiny +}}}   
      
\newcommand{\aff}{{\rm aff}}
\newcommand{\aire}{{\rm aire}}
\newcommand{\area}{\operatorname{area}}
\newcommand{\bis}{{\rm bis}}
\newcommand{\conv}{{\rm conv}}
%\newcommand{\degre}{{\rm deg}}
\newcommand{\del}{{\rm Del}} 
\newcommand{\dels}{\del _{\sss}(E)}
\newcommand{\delc}{{\rm DelC}}     
\newcommand{\dist}{\mbox {dist}}  
\newcommand{\dom}{{\rm dom}}
\newcommand{\insphere}{{\tt in\_sphere}}
\newcommand{\inter}{{\rm int}}
\newcommand{\intd}{\int\!\!\int}  %integrale multiple
\newcommand{\lag}{{\rm Lag}}
\newcommand{\length}{\operatorname{length}}
\newcommand{\lfs}{{\rm lfs}}
\newcommand{\lift}{{\rm lift}}
\newcommand{\ma}{\condmath{\cal M}}
\newcommand{\norm}[1]{\| #1 \|}
\newcommand{\obj}{{\condmath{\cal O}}}
\newcommand{\orient}{{\tt orient}}
\newcommand{\para}{{\condmath{\cal Q}}}
\newcommand{\plan}{P}
\newcommand{\ppv}{{\rm ppv}}
\newcommand{\point}{{\rm point}}
\newcommand{\power}{{\tt power\_test}}
\newcommand{\proba}{{\rm proba}}
% \newcommand{\proj}{\mbox {proj}} 
\newcommand{\proj}{\pi_S} 
\newcommand{\pts}{E}
\newcommand{\rch}{{\rm rch}}
\newcommand{\rips}{{\rm Rips}}
\newcommand{\cech}{{\rm {C}ech}}
\newcommand{\sign}{{\rm sign}}
\newcommand{\str}{{\rm star}}
\newcommand{\surf}{{\condmath{\cal S}}}
\newcommand{\sym}{{\rm sym}}
\newcommand{\tub}{\mbox {tube}}
\newcommand{\thk}{\mbox {thick}}
\newcommand{\vect}[1]{\,\overrightarrow{#1}}
\newcommand{\vol}{{\rm vol}}
\newcommand{\vor}{{\rm Vor}}
\newcommand{\voro}{Vorono�}
\newcommand{\vors}{\vor _{\sss}(E)}
\newcommand{\wfs}{{\rm wfs}}
\newcommand{\wit}{{\rm Wit}}
\newcommand{\suwit}{{\rm SuWit}}
            
\newcommand{\ceil}[1]{\left\lceil #1 \right\rceil}
\newcommand{\floor}[1]{\left\lfloor #1 \right\rfloor}
\newcommand{\bin}[2]{\, \left( \begin{array}{c} #1 \\ #2 \end{array} \! \right)} 
% \renewcommand{\v}[1]{{\bf #1}} % vectors BAD! 
     
\newcommand{\determinant}[4]{\, \left| \begin{array}{cc} #1 & #2 \\ #3 & #4 \end{array} \! \right|}
\newcommand{\Det}[3]{\, \left| \begin{array}{c} #1 \\ #2 \\ #3 \end{array} \! \right|}
\newcommand{\Determinant}[9]{\, \left| \begin{array}{ccc} #1 & #2 & #3 \\ #4 & #5 & #6 \\ #7 & #8 & #9 \end{array} \! \right|}
\def\condmath#1{\leavevmode\ifmmode #1 \else $#1$ \fi}

\newenvironment{proof}
        {\noindent \textbf{Proof.} \hspace{0.3mm}}
        {\hspace{0.3mm}$\square$  \smallskip}      
     
\newcommand{\JPdraw}[1]{
                \begin{figure}[h]
                \begin{center} \input{#1.ltex} \end{center}
                \end{figure}
                }
      
%\newenvironment{proof_e}
%        {\noindent \textbf{Proof.} \hspace{0.3mm}}
%        {\hspace{0.3mm}$\square$  \smallskip}      
     

\newtheorem{theorem}{Theorem}
\newtheorem{Theorem}{Theorem}      
\newtheorem{claim}{Claim}
\newtheorem{corol}{Corollary}
\newtheorem{Corol}{Corollary}
\newtheorem{corollary}{Corollary}
\newtheorem{defin}{Definition}
\newtheorem{definition}{Definition}
\newtheorem{fact}{Fact}
\newtheorem{Fact}{Fact}
\newtheorem{lem}{Lemma} 
\newtheorem{lemma}{Lemma}
\newtheorem{observation}{Observation}
\newtheorem{property}{Property}
\newtheorem{Property}{Property}
\newtheorem{proposition}{Proposition}
\newtheorem{prop}{Proposition}    
\newtheorem{question}{Exercise}
\newtheorem{questions}{Exercises}
\newtheorem{quest}{Question}
\newtheorem{remark}{Remark}  



\newcommand{\cqfd}{\mbox{}\hfill $\Box$ \medskip}

\parindent 0pt
\parskip 2mm



\usepackage{amsmath}
\usepackage{latexsym}
\usepackage{amssymb}
\usepackage{url}

\setlength{\textwidth}{18cm}
\setlength{\evensidemargin}{-1cm}
\setlength{\oddsidemargin}{-1cm}

\newcommand{\ind}{\mbox{}\hspace{2cm}}
\newcommand{\sind}{\mbox{}\hspace{1cm}}
   
\parindent 0pt
\parskip 1mm

\newcommand{\R}{\mathbb R}

\begin{document}

\title{Algorithmic Foundations of \\%Geometric and Topological Data Analysis\\
Geometric Modeling in Higher Dimensions\\ (GeHDi)}
\author{Jean-Daniel Boissonnat}

\maketitle

\begin{abstract}
% Computational geometry has been very successful in providing efficient algorithms and codes for applications in 2 and 3 dimensions.  However, geometric data are ubiquitous and arise from many sources other than measurements of coordinates of points in our familiar 3D world. 

During the past decade, exceptional progress was made with Geometric Modeling and 3D data processing. The new field of Geometry Processing emerged, theoretical foundations together with very efficient practical solutions were obtained for ubiquitous/emblematic problems like surface meshing and surface reconstruction. Extending these techniques to higher dimensions would have tremendous applications in science and engineering but is currently extremely limited and asks for new algorithmic  breakthrough. This project aims at settling the algorithmic foundations of Geometric Modeling in higher dimensions and to propose a well-principled ground-breaking software platform allowing technological advances for varied applications in science and engineering.

\end{abstract}

\paragraph{TODO.} Expand the abstract. Add references. Bag of features. Challenges : préciser les questions ouvertes.

\paragraph{Questions.} 
Geometric Modeling sounds outdated. List of WP. 
\newpage

\section{The principal investigator}

\subsection{Curriculum Vitae}

I was born in Nice, May 18, 1953. I am married and have 3 children. I am a french citizen.

\paragraph{Education}\mbox{}

$\bullet$ I graduated from the Ecole Sup\'erieure d'Electricit\'e (Sup\'elec) in 1976.  Sup\'elec is among the top ``Grandes Ecoles'' in France and the reference in the field of electric energy and information sciences. Sup\'elec is on a par with the best departments of electrical and computer engineering of the top American or European universities.
$\bullet$
I obtained my PhD Thesis in Information Theory from the University of Rennes (1979). 
$\bullet$
I obtained the Habilitation diploma (the highest grade at french universities) in Computer Science from the  University of Nice in 1992.

\paragraph{Professional academic experience}\mbox{}

 $\bullet$ I have been a researcher at INRIA since 1980, first in Rocquencourt and since 1986 in Sophia Antipolis.  $\bullet$  I am currently a {\em Research Director} (on par with full professor) at INRIA Sophia-Antipolis (France) where I lead the Geometrica research group. I was promoted in 2009 at the highest rank (Exceptional Class) for my contributions to research, formation and dissemination. $\bullet$ I held important positions within INRIA. I have been 
the {\em VP for Science}  at INRIA Sophia-Antipolis (500 employees, 38 research groups) (2001-2005). 
I have also been  the {\em chairman of the Evaluation Committee} of the institute, one of the most eminent positions at INRIA with a key role in the definition and the implementation
of the scientific policy of the institute (2005-2009).  This position gave me  a comprehensive view of the research performed in the eight research centers of INRIA.


% \paragraph{Scientific expertise} \mbox{}
% Computational Geometry, Algorithmic Robotics, Geometric Computing, Mesh Generation, Shape Reconstruction, Geometric Learning

\paragraph{Scientific leadership} \mbox{}

% I conducted My {\em scientific research} is mainly in computational geometry and topology, geometric modeling, algorithmic robotics and medical imaging.
% My main {\em contributions} are on  mesh generation, surface reconstruction, motion planning, robust geometric computing, randomized algorithms, Voronoi diagrams, Delaunay triangulations, manifold learning,  and to a lower extent on structural biology and control theory.

My first research group at INRIA, {\em Prisme}, has been the birth place of Computational Geometry in France and played a prominent role in shaping the field~\cite{by-ag-98}. 
I contributed and directed research on Voronoi diagrams, Delaunay triangulations, randomized algorithms, motion planning, exact computing, and, most importantly, initiated 15 years ago the development of the {\em CGAL library} in collaboration with partners in Europe (\url{http://www.cgal.org}{}).

In 2003, I founded the {\em Geometrica} research group in replacement of Prisme. The main objective was to develop Nonlinear Computational Geometry. 
 Geometrica's research in this area has been flourishing and well regarded internationally. Together with my students and collaborators, I made seminal contributions on mesh generation and surface reconstruction (see ref. 1-2, 5, 7-8 in Section~\ref{trackrecord}). A very visible output of this research has been the development of CGAL components. These components are now used worldwide in academia and in industry for various applications in geometric modeling, medical imaging and geology.\footnote{see \url{http://www-sop.inria.fr/geometrica/software/cgalmesh/}}

Together with F. Chazal, we established in 2006 a subgroup of Geometrica in {\em Saclay} (Paris's area)  to strengthen our work on  the emerging field of {\em geometric inference}.  This resulted in influential contributions to the analysis of distance functions, persistent homology, manifold learning and the development of geometric inference~\cite{geometrica-7142i,ccsm-gipm-2011}
\newpage


\paragraph{Publications and patents} \mbox{}

I am the author of over 150 technical publications including 1 text book, 57 journal articles, 91 refereed international conference articles, 12 book chapters.  I edited 4 books. My {\em h-number is 47 with 7786 citations} according to Google Scholar. 

I am the author of {\em 4 patents}~: 2 on mesh generation (Assignee: Institut Francais du P\'etrole (IFP)), robotic surgery (Assignee: Intuitive Surgical Inc.), virtual endoscopy (Assignee: Siemens Corporate Research)).

\paragraph{Software} \mbox{}

I am the author of two software that have been commercialized at large scale by major companies, one by Siemens (Flying Through, installed on Siemens scanners) and one by Dassault Systems (integrated in Catia V5 (Shape Editor)). 

My research group Geometrica has been one of the leader teams of  the CGAL Open Source Project since its start (\url{http://www.cgal.org}{}).  The CGAL library  is now regarded as the {\em gold standard in Computational Geometry} with a huge impact, both in academia and in industry.\footnote{For a few identified academic projects using CGAL (including two ERC Starting Grant projects), see
\url{http://www.cgal.org/projects.html}} Notably, the triangulation package of CGAL, developed within Geometrica, is now integrated in the heart of Matlab.

\paragraph{Supervision of Ph.D. students, postdocs and young researchers} \mbox{}

I have supervised 24 Ph.D. students. All of them are enjoying successful careers in academia or industry. I am currently advising two Ph.D. students. One of my former students, Andreas Fabri, founded in 2003 GeometryFactory, a startup company that commercializes CGAL.

Three members of my research group successfully  created their own research groups at INRIA with varied topics~: J-P. Merlet (Robotics), F. Cazals (Structural Biology), S. Lazard (Computational Geometry). Another member of the group, P. Alliez (who received an ERC starting grant), is in the process of creating his own group on geometry processing and modeling of urban scenes.

\paragraph{Academic awards and honors}\mbox{}

$\bullet$ I received two highly prestigious prizes, the {\em IBM award in Computer Science}  in 1987
and the {\em Grand prize EADS in Information Sciences} in 2006 (awarded by the {\em French Academy of Science}).  $\bullet$  I was nominated to the {\em international Roberval prize} for the french version of my book ”Algorithmic Geometry” coauthored by M. Yvinec (The Roverval prize awards the best scientific textbook in french). $\bullet$  I have been received in the {\em National Order of Merit} (Order of State awarded by the President of the French Republic)  in 2006.
My student C. Maria and I received a best paper awards at the European Symp. on Algorithms (ESA 2012).

\paragraph{Funding ID} \mbox{}

 % I have been the site leader of 8 European projects and the project leader of  the IST Project ECG ("Effective Computational Geometry") (2001-2004). % These projects gave a proeminent position to the European community of computational geometry, and led to the successful development of CGAL.

% I have been the principal investigator of 8 collaborations with french industry. The collaboration with Dassault Syst\`emes, a world leader in Geometric Modeling,  is of particular significance for this project %. We have been collaborating  with Dassault Syst\`emes for more than 10 years 
% (commercialization of software, joint publications).

We obtained with my team a budget of external funds of approximately 1.3 million Euros to support our research activities during the period 2007-2011 which divides into 0.5 from European projects,
0.5 from ANR projects (ANR is the french National Research Agency) and 0.3 from industry.

I am a site leader of the {\em ICT Fet-Open project} Computational Geometric Learning (CGL) which is closely related to the Gudhi project (\url{http://cglearning.eu/}). CGL will end in 2013. 
%http://cordis.europa.eu/fp7/ict/fet-open/) portfolio-cglearning_en.html. 

\newpage

\subsection{10-year track record}
\label{trackrecord}

\paragraph{Top 10 publications as senior researcher}  \mbox{} 

{\em Citations are according to Google Scholar. For journal articles, I added the citations of the conference version of the article. Discrete and Computational Geometry and the Symposium on Computational Geometry are regarded as the most prestigious journal and conference in Computational Geometry.}

%1. Triangulations in CGAL. Comput. Geom. Theory Appl.  Vol. 22 (2002) 5-19. Coauthors: O. Devillers, S. Pion, M. Teillaud, M. Yvinec. (26 citations including those of the conference version)

1. J-D. Boissonnat, F. Cazals. Smooth surface reconstruction via natural neighbour interpolation of
distance functions.  Comput. Geom. Theory Appl. Vol. 22 (2002) 185-203.  (421 citations)

2.  J-D. Boissonnat, F. Cazals. Natural neighbour coordinates of points on a surface.  
Comput. Geom. Theory Appl., Vol. 19, No 2-3, July 2001. (74 citations)

3. D. Attali, J-D. Boissonnat, A. Lieutier. Complexity of the Delaunay Triangulation of Points on Surfaces~: the 
Smooth Case. 20th  ACM      Symposium on Computational Geometry, 2003. 
(74 citations)

4. D. Attali, J-D. Boissonnat. A Linear Bound on the Complexity of the Delaunay Triangulation of Points on Polyhedral Surfaces.  Discrete and Comp. Geometry 31: 369--384
(2004). (61 citations)

5. J-D. Boissonnat, S. Oudot. Provably good sampling and meshing of surfaces. Graphical Models, 67 (2005) 405-451. (198 citations. Graphical Models Top-Cited Article 2005-2010)
% including those of the conference version

6. D. Attali, J-D. Boissonnat, H. Edelsbrunner. Stability and computation of medial axes~: a state-of-the-art report.
In {\em Mathematical Foundations of Scientific Visualization,
Computer Graphics, and Massive Data Exploration},
T. Moeller,   B. Hamann and B. Russell Ed.,
Springer, series Mathematics and Visualization, 2007. (93 citations)

7. J-D. Boissonnat, D. Cohen-Steiner, G. Vegter. Isotopic implicit surface meshing.  Discrete and Computational Geometry,  39: 138-157,  2008. (55 citations)% including those of the conference version)

8. J-D. Boissonnat, C. Wormser and M. Yvinec. Locally uniform anisotropic meshing. 
24th ACM Symposium on Computational Geometry, SoCG'08.
(16 citations but,  in my opinion, one of the most important papers in this list. It  led to several significant results related to the proposal, e.g.~\cite{geometrica-7142i}).

9. J-D. Boissonnat, L. Guibas, S. Oudot. Manifold reconstruction in arbitrary dimensions using witness complexes.
Discrete and Comp. Geom. Vol 42, No 1, 2009. (46 citations)

% 8. An efficient implementation of the Delaunay triangulation and
%   its graph in medium dimension.  25th ACM Symposium on Computational
%   Geometry, SoCG'09.  Coauthors: O. Devillers and S. Hornus.

10. J-D. Boissonnat, F. Nielsen, R. Nock. On Bregman Voronoi diagrams.
Discrete and Comp. Geom. (2), 2010. (76 citations)% including those of the conference version)

%10. Triangulating Smooth Submanifolds  with Light Scaffolding.
%Mathematics in Computer Science, 4(4):431-462, 2011. Coauthor: A. Ghosh.

\vspace{-1mm}

\paragraph{Edited Books and Proceedings}  \mbox{}

$\bullet$
Algorithmic Foundations of Robotics V, Springer 2004. Coeditors:  J. Burdick, 
K. Goldberg, S. Hutchinson.
$\bullet$ Effective Computational Geometry for Curves and Surfaces,
  Springer, 2006. Coeditor:  M. Teillaud. I coauthored two chapters of this book.
$\bullet$  Curves and Surfaces.
Coeditors:  P. Chenin, A. Cohen,  C. Gout, T. Lyche, M-L.  Mazure and L. Schumaker,
Springer Verlag LNCS Vol. 6920, 2012.
$\bullet$
Geometric Computing, special issue of the 
International Journal of Computational Geometry and Applications, Vol. 11, 
No. 1, 2001.
$\bullet$
Computational Geometry, Theory and Applications, Vol. 35 No. 1-2, August 2006.
Special issue on the 20th Symposium on Computational
Geometry.  %Coeditor:   J. Snoeyink.
$\bullet$ 
Discrete and Computational Geometry, Vol. 36, No 4, December 2006.
Special issue on the 20th Symposium on Computational
Geometry.  %Coeditor:   J. Snoeyink.

\vspace{-1mm}

\paragraph{Granted patents} \mbox{}

$\bullet$  Methods and apparatus for planning robotic surgery. 
United States Patent Application 20030109780. Assignee INRIA and
Intuitive Surgical Inc. (2002).  Coauthors: E. Coste-Mani\`ere, L. Adhami,
A. Carpentier, G. Guthart.
$\bullet$  Method and apparatus for fast automatic centerline extraction for virtual 
endoscopy. United States Patent Application  20050033114. Siemens Corporate 
Research (2004). Coauthor B. Geiger.

\vspace{-1mm}

\paragraph{Keynote presentations (since 2004)}\mbox{}

$\bullet$ International Symposium on Voronoi Diagrams, Tokyo (2004).
$\bullet$  Workshop "The World a Jigsaw: Tessellations in the Sciences", Leiden (2006).
$\bullet$  French Academy of Science (2 talks, 2006). 
$\bullet$  Franco Preparata's schriftfest, Brown university (2007). 
$\bullet$ Seventh conference on "Mathematical Methods for Curves and Surfaces", Toensberg, Norway, 2008. 
$\bullet$  Colloquium on Emerging Trends in Visual Computing (ETVC, Ecole Polytechnique, 2008). 
$\bullet$  22th Sibgrapi, Rio de Janeiro (2009).  
$\bullet$  ATMCS 2012 (Algebra and Topology; Methods, Computation, and Science), Edinburgh (2012).

\vspace{-1mm}

\paragraph{Membership to editorial board of international journals}   \mbox{}


I am on the editorial board of 5 international scientific journals, including two among the most prestigious  journals in Computer Science, the {\em Journal of the ACM} and  {\em Algorithmica}, and the first venue in my field {\em Discrete and Computational Geometry}. 


% I have been on the program committee of many international conferences.
% $\bullet$  {\em Algorithmica} (1990-) $\bullet$  {\em The Int. J. on Computational Geometry and Applications} (1991-)
% $\bullet$  {\em Discrete and Computational Geometry } (2006-)
% $\bullet$  {\em The Journal of Computational Geometry} (2009-)
% $\bullet$  {\em The Journal of the ACM }(2010-)

\vspace{-1mm}

\paragraph{Organization of international conferences} \mbox{}

%I have been a member of the Computational Geometry Steering Committee (1999-2001). 
%I chaired the Symposium on Computational Geometry (the top conference of the field) in 1997 and

$\bullet$ I co-chaired the program committee of the  Symposium on Computational Geometry (SCG), the top conference of the field,  in 2004. $\bullet$ I chair the Geometry Week in conjonction with SCG 2013.
$\bullet$ 
I chaired the Workshop on Algorithmic Foundations of Robotics (WAFR) in 2002.
$\bullet$ 
I have been a member of the steering committee of SCG  (1999-2001) and of the scientific committees of the International Conference on Curves and Surfaces (2010) and of  the eighth conference on "Mathematical Methods for Curves and Surfaces" (2012).
$\bullet$ 
I have been on the program committee of  the following international conferences~: Symposium on Geometry Processing (each year since its creation in 2003), 
STACS 2001 (Symposium on Theoretical Aspects of Computer Science),
ESA 2003 (European Symposium on Algorithms),
SCG'04 (Symposium on Computational Geometry)
SMI'05 (Shape Modelling International),
SMP'05 (ACM Symposium on Solid and Physical Modeling),
Curves and Surfaces 2006,
GMP 2012 (Geometric Modeling 
and Processing).

\vspace{-1mm}

\paragraph{International prizes/awards/academy memberships} \mbox{}

$\bullet$ I received the Grand prize EADS in Information Sciences in 2006 (awarded by the French Academy of Science).  $\bullet$ I received the Graphical Models Top-Cited Article for the period 2005-2010 for  my paper with S. Oudot (ref. 5 above). I received, together with my student C. Maria, one of the two best paper awards at the European Symposium on Algorithms ESA 2012~\cite{bm-dssc-2012}.

\vspace{-1mm}

\paragraph{Scientific councils and international visiting committees (2001-)} \mbox{}


$\bullet$  Scientific Council of the Ecole Normale Sup\'erieure de Lyon (2000-2003)
$\bullet$  Member of the AERES Board (French Evaluation Agency for
  Research and Higher Education)
% $\bullet$  Member of working groups 1 (Mod\`eles et calcul) and 2
%   (Logiciels et systèmes informatiques) of Allist\`ene (Alliance des sciences et technologies du num\'erique)
$\bullet$  Chair of the Visiting Committee of LIAMA, a  national center for international research of the chinese ministry of science, hosted by the Institute of Automation of the Chinese Academy of Sciences  (Beijing, 2010)
$\bullet$  Member of the Visiting Committee of the Computer Science department of ULB (Free University of Brussels, 2011)
$\bullet$    Chair of the Visiting Committee of the Geometric Modeling and Scientific Visualization  Center of the King Abdullah University of Science and Technology (KAUST, Saudi Arabia, 2012)




\newpage

\section{Extended synopsis of the project}

\paragraph{Geometry understanding in higher-dimensions.} 
% The central focus of this proposal is the %extraction, representation and
% computer analysis of geometric structures, which we refer to as {\em geometry understanding}.  
% The need for {\em approximating} complex shapes is ubiquitous in science and has become an essential part of {\em scientific computing}. It is by no means limited to 3-dimensions and
% many applications in physics, biology and engineering require a keen understanding of the geometry of a variety of higher dimensional spaces. Let us mention phase space in particle physics, invariant manifolds in dynamical systems, configuration spaces of mechanical systems, conformational spaces of molecules, image manifolds, shape spaces, to name a few.  % A problem of central importance in data analysis is to recover meaningful low-dimensional structures hidden in high-dimensional data.
% In {\em data analysis}, data are often thought as points in some  high-dimensional metric space~\cite{dld-hdda-2000}. Those points are usually not uniformly distributed in the embedding space but 
% %due to the very nature of the system that produced those data, 
% lie close to some {\em low-dimensional manifold}, which reflects the fact that the physical system that produced the data has a moderate number of  degrees of freedom. {\em Extracting}  an approximation of  the manifold  from the data,   {\em inferring some of its geometric and topological properties} are key to understanding the underlying system. % Extracting, representing, approximating, analyzing shapes and infering some of their geometric and topological features 
% These are fundamental questions in % under the umbrella of 
% {\em geometry understanding.}


% <<<<<<< .mine
% The need for computer-aided {\em understanding} of geometric structures, geometry understanding for short, is ubiquitous in science and has become an essential part of {\em scientific computing} and {\em data analysis}. Geometry understanding is by no means limited to 3-dimensions and many applications in physics, biology and engineering require a keen understanding of the geometry of a variety of higher dimensional spaces. Let us mention phase space in particle physics, invariant manifolds in dynamical systems, configuration spaces of mechanical systems, conformational spaces of molecules, image manifolds, shape spaces, to name a few. In {\em data analysis and manifold learning}, data are often thought as points in some high-dimensional metric space~\cite{dld-hdda-2000}. Those points are usually not uniformly distributed in the embedding space but lie close to some {\em low-dimensional manifold}, which reflects the fact that the physical system that produced the data has a moderate number of degrees of freedom.  Understanding the geometry of this manifold is key to understanding the underlying system. We do not refer to a single action and geometry understanding encompases a collection of problems like representing and approximating the manifold, extracting the manifold from the point set, inferring some geometric or topological properties.
% =======
The need for computer-aided {\em understanding} of geometric structures, geometry understanding for short, is ubiquitous in science and has become an essential part of {\em scientific computing} and {\em data analysis}. Geometry understanding is by no means limited to three dimensions and many applications in physics, biology, and engineering require a keen understanding of the geometry of a variety of higher dimensional spaces. Let us mention phase space in particle physics, invariant manifolds in dynamical systems, configuration spaces of mechanical systems, conformational spaces of molecules, image manifolds, and shape spaces, to name a few. In {\em data analysis and manifold learning}, data are often thought as points in some high-dimensional metric space~\cite{dld-hdda-2000}. Those points are usually not uniformly distributed in the embedding space but lie close to some {\em low-dimensional manifold}, which reflects the fact that the physical system that produced the data has a moderate number of degrees of freedom.  Understanding the geometry of this manifold is key to understanding the underlying system. We do not refer to a single action: the term geometry understanding encompases a collection of tasks including representing and approximating the manifold, extracting the manifold from the point set, and inferring some geometric or topological properties.


Let us illustrate our motivation and objectives through the paradigmatic example of {\em energy landscapes of molecules}. Understanding  energy landscapes is a major challenge in chemistry and biology but,  despite a lot of efforts and a wide variety of approaches, little is understood about the actual structures underlying such landscapes. The case of cyclo-octane $C_8H_{16}$, a cyclic alkane used in manufacture of plastics, is instructive. This relatively simple molecule has been studied in chemistry for over 40 years but it is only very recently that its conformational space has been fully understood. By analyzing a dataset of 1M points in $\R^{72}$, each describing a cyclo-octane conformation, it has been shown that the conformation space of cyclo-octane has the unexpected geometry of  a multi-sheeted 2-dimensional surface composed of a sphere and a Klein bottle, intersecting in two rings~\cite{mtcw-tco-2010}.  Besides its fundamental interest, such a discovery opens new avenues for understanding the energy landscape of cyclo-octane. Extending this type of analysis to large molecules, in particular to proteins, would have tremendous implications. Many other examples with high potential remain to be solved in domains as varied as neurosciences, medical imaging, speech recognition and astrophysics. {\em The crux is to have access to robust and efficient geometric data structures and algorithms to understand geometry in higher dimensions}. This is the grand challenge the Gudhi project wants to take up.



\paragraph{The curses of higher-dimensional geometry.} 
Many difficulties have to be faced when processing and analyzing
high-dimensional geometries. First, the dimensionality severely restricts our intuition and ability to visualize the data.  Understanding higher dimensional shapes must hence rely on {\em automated} methods and tools that produce {\em provably correct} results under {\em realistic circumstances}.

Second, major difficulties come from the fact that the complexity of data structures and algorithms used to approximate shapes rapidly grows as the dimensionality increases, which makes them intractable in high dimensions.  This phenomenon, referred to as the {\em curse of dimensionality}, prevents  in particular from subdividing the ambient space, as is usually done in 3-space, since the size of any such subdivision depends exponentially on the ambient dimension. Instead, any practical method must be {\em sensitive to the intrinsic dimension} (usually unknown) of the shape under analysis. As already mentionned, and observed in the cyclo-octane example, the intrinsic dimension can often be assumed to be much lower than the ambient dimension. This is a powerful assumption we refer to as the {\em manifold model}.\footnote{We use the term manifold in a rather liberal way, including stratified manifolds.} %  Note that, in the  cyclo-octane example above, the object of interest is 2-dimensional and lives in a much higher-dimensional ambient space.
% but can be modeled as a {\em low-dimensional manifold}.% This is a common assumption  in many applications  that captures the fact that, most of the time, high-dimensional data are associated to physical systems that have a moderate number of  degrees of freedom.%parameters.


In addition, high-dimensional data often suffer from significant {\em defects}, including sparsity, noise, and outliers that may hide the intrinsic dimension of the underlying structure. This is particularly so in the case of biological data, such as high throughput data from microarray or other sources. % Moreover, the structure and occurrence of geometric features in the data may depend on the {\em scale} at which it is considered, thus requiring the analysis to automatically select the appropriate scale.  These issues have been considered in the statistical community but not to the same extent in a geometric framework.

\paragraph{The emergence of geometry understanding in higher dimensions.}
The last decade has seen tremendous progress in  the understanding of geometry in high-dimensional spaces. %  In robotics~\cite{sml-pa-2006}, randomized techniques have been proposed to capture the topology of configuration spaces and to search paths.
In signal and image processing, and in machine learning, a variety of techniques, known as {\em nonlinear dimensionality reduction} have been proposed to reduce the dimension of data, to learn nonlinear manifolds and to cluster data~\cite{lv-nldr-2007}. % In computational geometry, new approaches have been proposed to solve basic problems like searching nearest neighbours, computing smallest enclosing ellipsoids or approximating convex sets~\cite{hp-gaa-2011}.
Such techniques are widely used but have limited guarantees and 
%address elementary questions (which may be very hard to solve though) and 
impose strong constraints on the dimension or topology of the shapes they can successfully handle. 

The techniques developed in {\em computational geometry and topology}  are complementary. They aim at processing and analyzing shapes with non trivial geometry and topology~\cite{hh-ct-2010}. 
Emblematic problems such as mesh generation and surface reconstruction in 3-dimensions are now well-understood and several provably correct and highly efficient solutions are now available~\cite{geometrica-ecg-book}. 
%The concepts of $\varepsilon$-samples, restricted Delaunay triangulation, anisotropic meshes %emerged together with the first efficient and provably correct algorithms for 

Attempts to analyze higher dimensional shapes led to the development of beautiful pieces of theory with deep roots in various areas of mathematics like Riemannian geometry, geometric measure theory, differential and algebraic topology. Let us mention  the emergence of a sampling theory of geometric objects, and of geometric inference~\cite{geometrica-ccl09}, and the groundbreaking invention and rapid growth of persistent homology~\cite{eh-ph-2008}.
These advances 
attracted  interest in several fields like biological data analysis~\cite{fpgo-airc-2009}, computer vision~\cite{cids-lbsni-2008} and sensor networks~\cite{rg-bptd-2008}. However, 
until now, the applications have been limited  to rather simple cases and to low dimensions.


\paragraph{The grand challenge~: settling the algorithmic foundations.}
Since many of the most promising applications are in higher dimensions, there is a pressing need to elaborate more effective tools that scale to real problems.  We identify the lack of  algorithmic foundations for geometry understanding in higher dimensions as 
the main cause of the current limited impact of geometric and topological methods.  Settling such foundations
%al geometric modeling 
is a challenge of great theoretical and practical significance at the heart of the Gudhi~project.


A {\em tenet} of this proposal is that, to take up the challenge, we need a {\em global approach} involving tight and long-standing interactions between {\em mathematical research, algorithmic design} and {\em advanced software development}. We believe that this is key to obtaining methods with built-in robustness, scalability and guarantees, which are the best promises for {\em impact in the long run.}
% Such a global approach has been successfully carried out in low dimensions with the 
% development of the Computational Geometric Algorithms Library CGAL~\cite{cgal}. 

%We want to build upon this success 
We strongly believe that, by following this paradigm,  our ambitious objective is realistic and can be reached. To pave the way towards this goal, we have identified  four main scientific challenges.

% \paragraph{The emerging field of structural data analysis.} {\em Early geometric approaches, 
% dimension reduction, NL manifold learning, computational topology, geometric inference
% 3d data processing, surface reconstruction, simplification. }


\paragraph{Scientific challenge 1 :  Choosing the right representation.}
%Going beyond affine models and Euclidean spaces.}

% Standard methods like PCA in Machine Learning to deal with high dimensional data assume that the data can be well approximated by some affine subspace of small dimension.  To overcome this 

As discussed above, dimensionality reduction techniques cannot provide precise approximation of complicated shapes (as required in scientific computing) nor compute essential features of a shape like its topological invariants.
% In the last decades, a set of new geometric methods, known as manifold learning, have been developed with the intent of parametrizing nonlinear shapes embedded in high-dimensional spaces. Although widely used, those methods  assume very restrictive hypotheses on the geometry of the manifolds sampled by the datapoints to ensure correctness. 
More expressive representations of shapes are provided % , inspired by what has been done in 3-dimensions, consists in approximating
by {\em simplicial complexes}, the analogue of triangulations in higher dimensions.
%Progress in higher-dimensions so far has been mostly theoretical and a  huge gap  remains to be %filled before  having at one's disposal fully satisfactory solutions of practical significance.% One research direction is to invent simplicial complexes of small complexity and easy to compute that still capture the main features of the underlying shape. see Attali and Carlsson. 
 Simplicial complexes can be used to produce fine meshes well suited to scientific computing purposes~\cite{mh-mpc-2002,boissonnat2010meshing}, or much coarser approximations still useful to infer some important features of shapes such as their homology or some local geometric properties~\cite{geometrica-ccl09,nsw-fhm-2008}. 
 A {\em central tenet} in this project is to regard {\em simplicial complexes as a unifying representation of shapes for geometry understanding in higher dimensions}.%\framebox{still open questions}


 Many types of simplicial complexes can be used and a first issue is to choose the appropriate type. This choice depends on the {\em combinatorial and algorithmic complexities} of the complex, as well as on its {\em power to approximate} a shape.
 The choice of a {\em metric} is another  fundamental issue 
that determines the type and quality of an approximation.
 The simple Euclidean distance in the ambient space, while easy to deal with, is often not the right choice.  As already mentioned, when working in high dimensional spaces, the objects of interest have often an intrinsic dimension much smaller than the ambient dimension. It is thus important to exploit the {\em intrinsic geometry} of the objects. Computational intrinsic geometry has not been seriously tackled yet and even a basic question like the existence of  Delaunay triangulations on Riemannian manifolds has been elusive so far.  This question is of utmost importance for anisotropic mesh generation and optimal approximations. Another important situation is when data are not provided as  point clouds in some Euclidean space, but rather as a matrix of pairwise distances (i.e., a {\em discrete metric space}). Although such data may not be sampled from geometric subsets of Riemannian manifolds, they may still carry some interesting topological structures that need to be understood. 
Lastly, let us mention other {\em pseudo-distances} such as Kullback-Leibler, Itakura-Saito or Bregman divergences that may be preferred in information theory, signal and image processing,
% %  In computer  vision, divergences are used throughout all the process of object recognition: to detect features using statistical, algebraic or geometric techniques, and to use these features throughout classifiers.
% %The recognition of objects in potentially complex scenes is a major issue in Computer Vision. 
% % Prominent advances in object recognition integrate two essential stages: the induction of features of limited size, over a potentially huge feature space, and the use of these features for learning and classification. 
 These divergences are usually not genuine distances (they may not be symmetric nor satisfy the triangular inequality) and  geometric data structures and algorithms need to be revisited in this context~\cite{geometrica-6154a}.




\paragraph{Scientific challenge 2 :  Bypassing the curse of dimensionality.} 
Simplicial complexes have been known and studied for a long time in mathematics, but not so much from a computational point of view. This is however a main issue since the complexity of many geometric algorithms and data structures grows exponentially with increasing dimension. It is thus not possible to partition a high dimensional space, which rules out most, if not all, geometric algorithms developed in low dimensions.
Hence, extending computational geometry in high dimensions cannot be done in a straightforward manner and one has to take advantage of additional structural properties of the problem. % This motivated a number of new concepts (e.g., core sets), new algorithmic paradigms (e.g., locality-sensitive hashing) and new analyses (e.g., smoothed analysis).
In this project, we will address the curse of dimensionality by focusing on the inherent structure in the data which we assume to be of relative {\em low intrinsic dimension}.  We will put the emphasis on {\em output-sensitive} algorithms and on {\em average-case} analysis.  First investigations led to very promising results, such as the design of new simplicial complexes with low complexity and approximation algorithms that scale well with the dimension \cite{geometrica-7142i}.


\paragraph{Scientific challenge 3 : Searching for stable models.} 
When dealing with approximations and samples, one needs stability results to ensure that the quantities that are computed are good approximations of the real ones. This is especially true in higher-dimensions where data are usually corrupted by various types of noise.  When the noise magnitude is small, methods have been proposed to robustly estimate topological and geometric properties of shapes~\cite{nsw-tvu-2011}.  The recent and fast developing theory of {\em persistent homology} provides a powerful tool to study  the homology of sampled spaces and to remove topological noise~\cite{eh-ph-2008}.
However, in  many applications, the noise is non local and the previous methods fail.
Recently,  larger families of noise models  have been considered and statistical approaches  have been proposed to provide shape approximations that are stable with respect to   those types of noise~\cite{gpvw-mme-2011}. These methods however do not provide topological guarantees on the approximation and the question of designing computationally tractable estimators converging at an optimal rate remains open. A major challenge is to design  unifying frameworks that {\em embrace statistical approaches and deterministic methods}, and offer topological guarantees.   

The automatic selection of the relevant {\em scales} at which the geometry of data should be considered is another issue.  This can be understood by considering a point set sampling a curve lying on a manifold, say an helix with a small pitch drawn on a cylinder. Depending on the scale, the expected output of a reconstruction algorithm will be an approximation of the curve or of the cylinder. Hence, the analysis process has to be multiscale or at least be able to automatically select the appropriate scale.  Here also, combining persistent homology and other geometric approaches~\cite{geometrica-bgo-09} with statistical techniques would lead to significant progress.

\paragraph{Scientific challenge 4 : Turning theory into practice.}
A major challenge, if not the most important, is to develop {\em theory} that is {\em of practical significance} for applications.   To take up the challenge, we will %  foster a symbiotic relationship between theory and practice, and
undertake the development of a {\em software platform} devoted to geometry understanding in higher dimensions. 
We consider such a platform as central to our research  for three main reasons.  {\em First}, the software platform will allow {\em large scale experimentations}, which is mandatory to design the right models and data structures. We believe that this will revitalize the current theory and open {\em new vistas for research}, both of a practical and a theoretical nature, leading towards a virtuous circle between theory and experimental research. This has proven to be of utmost importance
when developing the CGAL library and will be even more true in high dimensional geometry.

{\em Second}, maintaining such a platform will help further effort and {\em consolidation in the long run}.  Having a library with interoperable modules will allow us to incrementally add more and more sophisticated tools based on solid foundations.  This is consistent with our long-term vision and our conviction that it is only through such a long standing effort that true impact, both theoretical and applied, can be gained.

{\em Third}, the platform will serve as a unique tool to {\em communicate} with the computational geometry community and with researchers from other fields. 
 In return, we will get feedback from practitioners which will help shape the theoretical models and the software platform.

\paragraph{Objectives and research roadmap.}
The ambition of this proposal is to settle the {\em algorithmic
foundations} of geometry understanding in {\em dimensions higher than
3}.  We intend to develop {\em scalable representations in the form of simplicial complexes} and {\em
practical algorithms} to approximate {\em highly nonlinear shapes}, and to
infer geometric and topological properties from data subject to
significant {\em defects} and under {\em realistic conditions}.
As is common in many applications across science and engineering, we
will assume that the objects of interest can be modeled as {\em
  low-dimensional manifolds} embedded in possibly high-dimensional
spaces. By exploiting the {\em intrinsic properties} of the objects,
we will produce intrinsic dimension-sensitive data structures and algorithms
that will {\em break the current computational
bottleneck.}  %models in the form of {\em simplicial complexes}.

% % interest which can be modeled as {\em low-dimensional manifolds} in
% % many applications.  % We % intend to develop {\em practical
% algorithms to approximate % highly nonlinear manifolds, and to infer
% geometric and topological % properties from data subject to
% significant {\em defects} and under % {\em realistic conditions}.  
To reach these objectives, the guiding principle will be to foster a
symbiotic relationship between theory and practice, and to address
{\em fundamental research} issues along three parallel advancing
fronts. We will simultaneously develop {\em mathematical approaches}
providing theoretical guarantees, {\em effective algorithms} that are
amenable to theoretical analysis and rigorous experimental validation,
and {\em perennial software} development. %  We will undertake the
% development of a high-quality open source {\em software platform} to
% implement the most important geometric data structures and algorithms
% at the heart of geometry understanding in higher dimensions. 
 % The central goal of this proposal is to settle the {\em algorithmic foundations of geometry understanding in higher dimensions}.  We aim at processing general highly nonlinear geometries with nontrivial topologies that can be modeled as  {\em low-dimensional manifolds} embedded in possibly high-dimensional spaces. %  Approximation of such manifolds will be in the form of
% % {\em simplicial complexes}.  We intend to bypass the computational bottleneck by exploiting the {\em intrinsic properties} of the objects and to design data structures and algorithms that scale with the intrinsic dimension of the objects of interest rather than with the ambient dimension. We intend to develop {\em practical algorithms} to mesh or reconstruct highly nonlinear manifolds, and to infer geometric and topological properties from data subject to significant {\em defects} and under {\em realistic conditions}. 
% This proposal addresses {\em fundamental
%   research} issues, and its results are expected to serve as a basis
% for groundbreaking advances for {\em applications in scientific computing
% and data analysis}.  % A major outcome of the project will be a
% % high-quality open source software {\em platform} of components
% % implementing the main results. 
% %
% To reach these objectives, the guiding principle  will be to simultaneously
% develop {\em mathematical approaches} providing theoretical
% guarantees, {\em effective algorithms} that are amenable to both
% theoretical analysis and rigorous experimental validation, and {\em
%   perennial software} development.


The proposal is structured into the following four {\em focus areas}  that address the four scientific challenges listed above.
{\bf A1} -- {\em Intrinsic dimension-sensitive data  structures} --  will address Challenges 1 and 2 by extending current knowledge on the combinatorial and algorithmic properties of simplicial complexes. 
  {\bf A2} --  {\em Triangulation of non Euclidean metric spaces} -- will address Challenges 1 and 2 by developing effective algorithms to mesh or reconstruct manifolds equipped with various metrics.   {\bf A3} -- {\em Robust models for geometric and topological inference} -- will address Challenge 3 by providing the crucial  algorithms for topological data analysis.
 {\bf A4} --  {\em  Software platform for geometric understanding in high dimensions} -- will address Challenge~4 by providing the software environment for experimenting with our new data structures and algorithms, for integrating them in a library of interoperable modules, and for diffusing our results to applied fields. 

\paragraph{Risks and feasibility of the project.} 

% The Gudhi proposal may look too ambitious and risky since we want to simultaneously conduct basic research at the best international level and to develop software of the highest quality.
%. Although risky, we argue that this approach is key to success. It can be argued that  my research group Geometrica is the best team worldwide to take up the challenge and to make this project a success. 
 My personal record as well as the record of my research group Geometrica are strong indications of my ability to take up the challenge of simultaneously conducting basic research at the best international level and developing software of the highest quality with good chances of success.  
Geometrica has been at the cutting edge of research in geometric data structures and algorithms~\cite{by-ag-98}, mesh generation~\cite{geometrica-ecg-book}, manifold reconstruction~\cite{geometrica-7142i,geometrica-bgo-09}, geometric inference and computational topology~\cite{geometrica-ccl09,geometrica-cseh-07}. Geometrica is also one of the leader teams of the CGAL project (www.cgal.org).  We were at the source of successful developments in CGAL like interval arithmetics, 2 and 3-dimensional triangulations (now integrated in the heart of Matlab) and meshing packages. We also took a prominent part in the organization of the CGAL project and community, and in the creation of the spinoff GeometryFactory.
%
%The research won't be conducted in isolation.  
We will benefit from our strong collaborations with the best groups in US and  Europe,
most notably with our partners of the ICT Fet-Open project Computational Geometric Learning (CG-Learning) which is closely related to this project (focus, timetable and management are different though).

% Research in computational geometry and topology is very active in  Europe.  The ICT Fet-Open project Computational Geometric Learning (CG-Learning) is closely related to this project. 
% CG-Learning is an exploratory project touching upon several topics and devoting limited ressource to software development. The Gudhi proposal wants to take over the results of CG-Learning
% and to focus on one of the most promissing emerging research direction,  and to go far beyond prototype developments.  



%We will also benefit from our long-standing collaborations in the USA with Stanford university %(Pr. Guibas) and Ohio State university (Pr. Dey) on topics that are related to this project.



\paragraph{New horizons and opportunities.} 



% Computational geometry and topology have produced beautiful pieces of theory and. Although many of these tools are of great potential impact in applications, implementing those ideas in robust and efficient codes will only be possible through a long-standing alliance between mathematical developments, algorithmic design and advanced programming.  This last point is underestimated and implementation is often considered as an engineering activity that can be let to students. I believe that this is a wrong vheartiew that largely explains the current absence of  reliable software toolbox for geometric modeling in higher-dimensions. % My conviction, built after more than ten years of development of CGAL, is  that the suggested workplan is the only way to make  impact in the long run.

Upon successful completion, the Gudhi project will put geometry understanding in higher-dimensions on new theoretical and algorithmic ground. It will help set up a first class research group with a unique spectrum of expertise covering mathematics, algorithm design and software development.  This new project will further strengthen the leadership of Europe in Geometric Computing.  It will also provide an open platform with no equivalent in USA or Asia.  We foresee the project becoming a catalyst for research in high-dimensional geometry inside and outside of the project.  By implementing the most effective techniques in a reliable and scalable way, the platform will open the way to groundbreaking technological advances in scientific computing and data analysis for applications as varied as numerical simulation, neurosciences, astrophysics and molecular biology. We will keep close contacts with a small set of prominent researchers working in those domains and leap on opportunities arising from our new results and tools.  Specific attention will be paid to the analysis of {\em energy landscapes of molecular systems} and to understand the phase space dynamics of {\em cosmic structure formation}.

% {\footnotesize
% \bibliographystyle{abbrv}
% \bibliography{erc}
% }

%\paragraph{References} \mbox{}\\

\begin{thebibliography}{}
\vspace{-2mm}

{\footnotesize

\bibitem[Boissonnat and Ghosh, 2010b]{boissonnat2010meshing}
Boissonnat, J.-D. and Ghosh, A. (2010b).
\newblock Triangulating smooth submanifolds with light scaffolding.
\newblock {\em Mathematics in Computer Science}, 4(4):431--461.
\vspace{-2mm}

\bibitem[Boissonnat et~al., 2009]{geometrica-bgo-09}
Boissonnat, J.-D., Guibas, L.~J., and Oudot, S. (2009).
\newblock Manifold reconstruction in arbitrary dimensions using witness
  complexes.
\newblock {\em Discrete and Computational Geometry}, 42(1):37--70.
\vspace{-2mm}

\bibitem[Chazal et~al., 2011]{ccsm-gipm-2011}
Chazal, F., Cohen-Steiner, D., and M\'erigot, Q. (2011).
\newblock Geometric inference for probability measures.
\newblock {\em Journal on Foundations of Computational Mathematics},
  11(6):733--751.
\vspace{-2mm}

\bibitem[Donoho, 2000]{dld-hdda-2000}
Donoho, D. (2000).
\newblock High-dimensional data analysis: The curses and blessings of
  dimensionality.
\newblock In {\em Mathematical Challenges of the 21st Century}. AMS.
\vspace{-2mm}

\bibitem[Edelsbrunner and Harer, 2010]{hh-ct-2010}
Edelsbrunner, H. and Harer, J. (2010).
\newblock {\em Computational topology}.
\newblock American Mathematical Society.
\vspace{-2mm}

\bibitem[Ghrist, 2008]{rg-bptd-2008}
Ghrist, R. (2008).
\newblock Barcodes: the persistent topology of data.
\newblock {\em Bull. Amer. Math. Soc., 45(1)}, pages 61--75.
\vspace{-2mm}

\bibitem[Niyogi et~al., 2008]{nsw-fhm-2008}
Niyogi, P., Smale, S., and Weinberger, S. (2008).
\newblock Finding the {H}omology of {S}ubmanifolds with {H}igh {C}onfidence
  from {R}andom {S}amples.
\newblock {\em Discrete and Computational Geometry}, 39(1):419--441.

}

\end{thebibliography}


\newpage

\section{Research proposal (15p.)}

\subsection{State-of-the-art and objectives}

Geometry understanding has undergone huge progress during the past decades. The pressing needs
of multimedia, video games, numerical simulations, manufacturing, computer-aided medicine, culturage heritage and other applications asked for techniques to represent, process and analyze
{\em 3D} shapes.  Progress has been contributed by  varied disciplines, most notably computer graphics, geometry processing, computer-aided design, computational geometry and topology, scientific computing. This considerable research effort resulted in solid theoretical and algorithmic foundations for 3D geometry understanding, and efficient algorithms and codes for applications such as surface reconstruction, mesh generation and point cloud processing~\cite{he-gtmg-2001,geometrica-bcmrv-ms-06,dey-csr-2007}.  
The situation is very different and much less has been done for higher dimensional shapes.%   are much more difficult to handle than surfaces in 3-space for two main reasons~: the curse of dimensionality and the unafordable presence of noise.


\paragraph{Dimension reduction.} A first route towards geometry processing in higher dimensional spaces is to try to reduce the dimension. This is 
one of the most popular approaches to high-dimensional data analysis. {\em Dimensionality reduction} consists in mapping the data points down to a linear subspace, whose dimension supposedly coincides with the intrinsic dimension of the data. This approach is elegant in that it helps detect the intrinsic parameters of the data, and by doing so it also reduces the complexity of the problem. Dimensionality reduction techniques fall into two classes: linear methods, e.g. principal component analysis (PCA) or multi-dimensional scaling (MDS), and non-linear methods, e.g. isomap or locally-linear embedding (LLE). The second class of algorithms is more powerful in that it computes more general (in fact, non-linear) mappings. On the whole, dimensionality reduction works well on data sets sampled from manifolds with low curvature and trivial topology. Although the condition on the curvature is mainly a sampling issue, the condition on the topology is mandatory for the mapping onto a linear subspace to make sense. \framebox{Johnson-Lindenstrauss, Wakin}

\paragraph{Simplicial complex approximations.}
In many problems \framebox{examples?}, a linear subspace or a union of linear subspaces is a too crude model and we need to represent nonlinear geometries like a $k$-dimensional submanifold of $\R^d$.  To deal with such complicated geometries, dimensionality reduction techniques are not sufficient. Another route is to approximate shapes by {\em simplicial complexes}~\cite{hh-ct-2010}.  Simplicial complexes have been introduced by Poincar\'e in the early days of algebraic topology. Their importance in low dimensions cannot be overestimated~: 1-dimensional simplicial complexes, better known as graphs or networks, can be found everywhere, 2-dimensional simplicial complexes are the standard representations for surfaces in computer graphics and 3-dimensional simplicial complexes are the representation of choice for scientific computing and numerical simulation when complicated domains are involved. Simplicial complexes can be defined in any dimension and can be used to construct a piecewise linear approximation of a shape with the right topological type~\cite{geometrica-7142i} or to 
compute the homology of the shape~\cite{hh-ct-2010}. The main bottleneck in using simplicial complexes in higher dimensions is of a computational nature~:  in high dimensions, simplicial complexes can be huge and very difficult to compute, which limits their current use to low dimensions.

\paragraph{Computational issues.}
The high computational cost of simplicial complexes can be balanced in two ways. First, simplicial complexes are flexible enough to approximate subspaces with a complexity that scales with the complexity of the subspace, not of the ambient space. Typically, the subspaces of interest 
 in learning theory, data analysis or dynamical systems have a moderate  intrinsic dimension.

The second reason is that coarse topological features such as the homology groups or the Betti numbers (that are invariant under homotopy equivalence) can be estimated from simplicial complexes of a much smaller size than what is needed to reconstruct a homeomorphic approximation~\cite{co-tpr-2008}. Hence, simplicial complexes may be used to recover important information even if an accurate representation of the data is out of reach.

Simplicial complexes thus have the potential of being the data structure of choice for understanding geometry in high dimensional spaces. The issue is however to define simplicial complexes that are dimension-sensitive and easy to compute.  Various types of simplicial complexes have been proposed such as the \v{C}ech and the Rips complexes, and more recent Delaunay-like complexes such as the $\alpha$-complex~\cite{eks-sspp-83,he-ubds-95}, the witness complex~\cite{deSilva2008,cds-tewc-2004} and the Delaunay tangential complex~\cite{geometrica-7142i}. They differ by their combinatorial and algorithmic complexities, and their power to approximate a shape.  The algorithmic theory of simplicial complexes is nevertheless in its infancy and much less developed than its counterpart for triangulations in small dimensions~\cite{by-ag-98,he-gtmg-2001}. In particular, little is known about efficient and compact representations of simplicial complexes and experiments have only been reported for simplicial complexes of low-dimensions up to now ~\cite{geometrica-6743i,Attali2011,rg-bptd-2008}. \framebox{true?} 


\paragraph{Geometry understanding.}

The last decade has seen tremendous progress in processing high-dimensional shapes and spaces. In {\em robotics}, randomized techniques have been proposed to construct graphs that capture the connectivity of configuration spaces and allow to search paths~\cite{sml-pa-2006}. 
In engineering and science, {\em dynamical systems} are often formulated as a function that satisfies some set of nonlinear equations, for example, the Navier-Stokes equations, Maxwell's equations, or Newton's law. Computing a  solution in the form of a mesh can be done using
higher-dimensional continuation method~\cite{mh-mpc-2002}.  However, theoretical guarantees are very few~\cite{boissonnat2010meshing} and  {\em mesh generation} in high dimensional spaces is still much less developed than its counterpart in low dimensions~\cite{he-gtmg-2001,geometrica-ecg-book}.  

In geometric {\em data analysis}, the tenet is that to effectively exploit datasets, one needs to identify, extract and analyze their underlying geometric structure.  In low dimensions, effective {\em reconstruction} techniques exist that can provide faithful approximations of the underlying structures from samples~\cite{dey-csr-2007}. Further processing makes it possible to study their topological and geometric properties. In high dimensions, however, the data often suffers from significant defects, including sparsity, noise, and outliers, violating sampling conditions required by extant methods. The problem is further compounded by the rapid growth in complexity of the data structures used for reconstruction as the dimensionality of the data increases, making them intractable in high dimensions. 


To face these challenges, researchers have proposed algorithmic tools that can infer some of the properties of the structures underlying the data without full reconstruction. Dimensionality reduction techniques mentionned above have been pretty successful in this vein % : they can infer the intrinsic dimensionality of the data, as well as provide structure-preserving mappings of the data into lower-dimensional spaces when such mappings exist. These techniques however are
but have only been proved to converge when limited to simple shapes with trivial topology.  Recently, {\em topological methods} were proposed to overcome these limitations. They led to the beautiful developments of {\em persistent homology}~\cite{eh-ph-2008} and {\em geometric inference}~\cite{geometrica-ccl09}. Although of a fundamental nature, these advances attracted interest in several fields like data analysis, computer vision or sensor networks~\cite{rg-bptd-2008}. The bottleneck that still prevents applications to benefit from the full potential of these new methods is the lack of efficient data structures and algorithms to construct simplicial complexes in high dimensions.


% \paragraph{Noisy data.} \framebox{to be revised}
% When dealing with approximation and samples, one needs stability results to ensure that the quantities that are computed, geometric or topological invariants, are good approximations of the real ones. {\em Topological persistence} was recently introduced as a powerful tool for the study of the topological invariants of sampled spaces~\cite{eh-ph-2008,rg-bptd-2008}. Given a point cloud in Euclidean space, the approach consists in building a simplicial complex whose elements are filtered by some user-defined function. This filter basically gives an order of insertion of the simplices in the complex. The persistence algorithm, first introduced by Edelsbrunner, Letscher and Zomorodian \cite{elz-tps-2002}, is able to track down the topological invariants of the filtered complex as the latter is being built. As proved by Cohen-Steiner et al. \cite{geometrica-cseh-07}, under reasonable conditions on the input point cloud, and modulo a right choice of filter, the most persistent invariants in the filtration correspond to invariants of the space underlying the data. Thus, the information extracted by the persistence algorithm is global, as opposed to the locality relationships used by the dimensionality reduction techniques. In this respect, topological persistence appears as a complementary tool to dimensionality reduction. In particular, it enables to determine whether the input data is sampled from a manifold with trivial topology, a mandatory condition for dimensionality reduction to work properly. Note however that it does not tell how and where to cut the data to remove unwanted topological features.

% Multiscale reconstruction is a novel approach~\cite{geometrica-bgo-09}. Taking advantage of the ideas of persistence, the approach consists in building a one-parameter family of simplicial complexes approximating the input at various scales. Differently from above, the family may not necessarily form a filtration, but it has other nice properties. In particular, for a sufficiently dense input data set, the family contains a long sequence of complexes that approximate the underlying space provably well, both in a topological and in a geometric sense. In fact, there can be several such sequences, each one corresponding to a plausible reconstruction at a certain scale. Thus, determining the topology and shape of the original object reduces to finding the stable sequences in the one-parameter family of complexes.   However, multiscale reconstruction, at least in its current form, still has a complexity that scales up exponentially with the dimension of the ambient space. Hence, it can only be applied to low-dimensional data sets in practice.


\paragraph{The  virtuous circle  of research and development in computational geometry.}
Experimental research and implementation are by now major components of research in computational geometry.  
Combining in a closer way basic theoretical research and the development of robust software
has been very successful in shaping the field and promoting computational geometry tools outside the community.
% is the study of effective methods (algorithms and data structures) of geometric computing, namely methods that are not only theoretically proved but also work well in practice.  
% Several EC projects (CGAL, GALIA) established an outstanding research momentum and gave a leading role to Europe in this context.  They led to successful techniques and tools, most notably 
One of the major achievements of the field has been the CGAL library that provides a well-organised, robust and efficient software environment for developing geometric applications~\cite{cgal}. CGAL is by now the standard in geometric computing, with a large diffusion worldwide and varied applications in both academia and industry.  My research group at INRIA took a leading role in the development of CGAL since its start more than 10 years ago. More recently, we seconded our basic research on 3D shape processing by practical developments in the form of fast, safe and quality-guaranteed software components for mesh generation and shape reconstruction.  Those components  are now part of the open source library CGAL  and used worldwide in academia and in industry for various applications in geometric modeling, medical imaging and geology ~\cite{cgal:rty-m3-11}.. The functionalities of CGAL in higher dimensions are quite limited. Besides, only very few prototype software have been developed for geometry understanding in higher dimensions (see http://comptop.stanford.edu/programs/).

\paragraph{Objectives.} In view of the state-of-the-art, our main objective is 
to develop an effective theory of geometry understanding in higher dimensions. We aim at processing general geometries represented as simplicial complexes. The computational bottleneck will be bypassed by assuming that the objects of interest are of moderate intrinsic dimension possibly embedded in high dimensional spaces. We intend to develop practical algorithms to mesh or reconstruct highly nonlinear manifolds, and to infer geometric and topological properties from data under realistic conditions. A major outcome of the project will be a high-quality open source library of components implementing the main results.

% Processing and analyzing complex 3D shapes is a fundamental problem with a long history of scientific successes which is on the agenda of several communities like scientific computing, computer graphics, geometry processing and computer-aided design.  Emblematic problems are mesh generation that aims at sampling and meshing a given domain, and surface reconstruction that constructs an approximation of a surface which is only known through a set of points. During the last decade, the computational geometry community established solid theoretical foundations to these problems leading to recent breakthroughs in {\em mesh generation} \cite{geometrica-bcmrv-ms-06} and {\em surface reconstruction} \cite{dey-csr-2007}.  The Geometrica group took a leading role in this research and contributed major theoretical advances as well as practical developments in the form of fast, safe and quality-guaranteed software components for mesh generation and shape reconstruction that are now part of the open source library CGAL~\cite{cgal:rty-m3-11}. Those components are now used worldwide in academia and in industry for various applications in Geometric Modeling, Medical Imaging and Geology.




\subsection{Methodology}
To reach our objectives, we will follow the principles that have been guiding my group for more than 10 years and will simultaneously develop
{\em mathematical approaches} providing guarantees even in the presence of noise or outliers,
{\em effective algorithms} that are amenable to theoretical analysis and fully validated experimentally,
and {\em perennial software} development.

The proposal is structured into the following four workpackages:
{\bf WP 1}:  {\em Dimension-sensitive data  structures} will extend current knowledge about simplicial complexes, and  provide efficient data structures and basic algorithms for their representation, construction and manipulation. 
  {\bf WP 2}:  {\em Triangulating non Euclidean geometric spaces} will develop effective algorithms to mesh or reconstruct manifolds and other spaces equipped with various metrics.   {\bf WP 3}: {\em Robust models for geometric inference, comparison and  clustering} will provide the crucial  algorithms for topological data analysis.
 {\bf WP 4}:  {\em  Software platform for geometric understanding in high dimensions} will provide the software environment for experimenting with our new data structures and algorithms, for integrating them in a coherent library of interoperable modules, and for diffusing our results to applied fields. We now describe in more detail each of these workpackages.


\subsection*{WP 1: Computational geometry in non euclidean spaces} 
Delaunay triangulation and Voronoi diagrams in non-euclidean spaces, e.g. riemannian manifolds,
Bregman and statistical spaces, intrinsic algorithms (discrete metric spaces). Triangulating Riemannian manifolds, stratified shapes, mesh generation, reconstruction.  Parameterization of data/nonlinear shapes. 

In the last decades, a set of new geometric methods, known as manifold learning, have been developed with the intent of parametrizing nonlinear shapes embedded in high-dimensional spaces. Let us mention MDS, LLE, Isomap to name a few. While these methods are able to parametrize nonlinear manifolds, they assume however very restrictive hypotheses on the geometry of the manifolds sampled by the datapoints to ensure correctness.  Moreover, from a computational point of view, a common drawback of these methods is that they usually involve computations of eigenvalues and eigenvectors of matrices of the size of the dataset, preventing to deal with huge datasets without a pre-processing.

Our goal is to design new methods and algorithms for sampling and approximating shapes of any codimension in high dimensional spaces.

Approximation of submanifolds using Delaunay refinement. We intend to extend the Delaunay refinement paradigm to higher-dimensional manifolds. For surfaces of R3, this approach has been proven to have several advantages over grid methods, leading to better quality and complexity of the approximation. We expect this advantage to be even stronger as the dimension increases. Two main issues should be considered. First, since the size of the Delaunay triangulation depends exponentially on the ambient dimension, we intend to compute instead lighter data structures such as the ones proposed in Work Package 2 (witness complex, Rips complex or tangential complex). Second, in order to get guaranteed approximation properties, we will have to propose mechanisms to remove badly shaped simplices (slivers) in higher dimensions. Although techniques have been proposed for sliver removal in higher dimensions, they are either inefficient or do not have theoretical guarantees. We intend to develop new techniques based on optimal sampling of convex function to get rid of slivers. Such an approach has proven to be very successful in practice for surfaces in R3. We intend to give better theoretical foundations to this method and to implement it for higher-dimensional manifolds. An important special case we intend to consider is isosurfacing in arbitrary dimension and codimension [90]. By isosurfacing, one means constructing f−1(c) where f is a discrete function f:Zd→Rk and c a given point in Rk.

================DELAUNAY

===========STABILITY


Delaunay triangulations are one of the most useful constructions in
Computational Geometry that have found applications in many domains of
science. Delaunay triangulations have been extensively studied since
the work of B. Delaunay in 1934 and discovery of new important
properties has never declined. Rather surprisingly though, the
stability of those structures has not been studied in a systematic
way. Related work can be found in the context of kinetic data
structures \cite{Agarwal:2010:KSD:1810959.1810984} or in the context
of robust computation \cite{Salesin:1989:EGB:73833.73857}. Our
motivation comes from recent attemps to extend Delaunay triangulations
beyond Euclidean spaces. A first example is the generation of
anisotropic meshes. In such an application, a metric tensor field is given
that varies over a domain of $\rem$ we want to mesh. Anisotropic
Voronoi diagrams and anisotropic Delaunay triangulations then emerge
as natural structures \cite{labelle2003}. A related (and more general) question
is to define intrinsic Delaunay triangulations on a Riemannian
manifold \cite{leibon2000}. Other types of Delaunay-like structures have been
proposed to approximate submanifolds, most notably the restricted
Delaunay triangulation \cite{edelsbrunner1997rdt}, and the tangential Delaunay complex
\cite{boissonnat2011tancplx}. We might expect that, when the density of points is dense
enough, all these Delaunay-like structures are similar. In fact this
is the type of result that can be found in \cite{labelle2003}
and in \cite{leibon2000}. However, the result of Labelle and
Shewchuk is limited to the 2-dimensional case and the paper of Leibon
and Letscher contains a flaw. These papers in fact miss an important
condition which is not related to the sampling density but to its
genericity. Roughly, the Delaunay triangulation is unstable around
cospherical configurations which ruins any attempt to define Delaunay
triangulations on domains where the metric varies.

The aim of this paper is to introduce a parametrized notion of
genericity for Delaunay triangulations and to state stability results
when we keep away from degeneracies. This paper builds over
preliminary results on anisotropic Delaunay meshes
\cite{Boissonnat:2008:ADL:1456721.1456962} and manifold reconstruction
using the tangential Delaunay complex \cite{boissonnat2011tancplx}. The central idea in
both cases is to define Delaunay triangulations locally and to glue
the local triangulations together by removing inconsistencies among
the local triangulations. This is achieved by first detecting the
so-called cospherical configurations and then by killing them. The
cospherical configurations are themselves fragile and can be killed by
various means, e.g., by weighting the points or by refining the
sample.

The present paper provides a general study of the stability of
Delaunay triangulations, and applies the results to the problem of
meshing compact differentiable submanifolds of Euclidean space. We
introduce the notion of $\delta$-protected Delaunay simplices and of
$\delta$-generic point sets.  This leads to our stability results. The
concept is also related to the well-known notion of thickness (or
fatness) of a simplex: We show that the Delaunay simplices of
$\delta$-generic point sets are thick.  We also show how to produce
such point sets.



==================MESHING

The problem of triangulating manifolds has a long history in the
mathematical literature. In differential topology, seminal
contributions are due to Whitney~\cite{whitney}, Cairns~\cite{cairns},
Munkres~\cite{munkres}, Whitehead~\cite{whitehead} to name a
few. Although these papers are not of an algorithmic nature, they
introduce and study several interesting concepts that have been
extensively used in Computational Geometry recently such as Voronoi
diagrams restricted to a manifold, $\e$-sample of a manifold, fat (or
thick) triangulations. However, these papers do not discuss the
geometric quality of the approximation nor the size of the sample. The
optimal sampling and approximation of convex bodies is also a long
standing problem in convex optimization with major contributions by
Gruber~\cite{gruber1,gruber2} and Dudley~\cite{dudley}. Recently,
Clarkson~\cite{clarkson} extended this line of work to non-convex
smooth manifolds of arbitrary dimensions. 
%In~\cite{clarkson}, tight
%bounds on the Hausdorff error are established. Moreover, as pointed
%out by Clarkson, there exists a general algorithm to construct such
%approximations. 
However, his algorithm follows an intrinsic point of
view which makes it difficult to use in practice since it requires to
compute geodesic distances on the manifold which may be quite
complicated in practice \cite{pc-gcsrp-05}. Other, more practical
algorithms for approximating convex bodies, including the well-known
sandwich algorithm, have been analyzed by
Kamenev~\cite{convex-bodies}. We are not aware of similar studies for
non convex manifolds except for the case of surfaces embedded in $\R
^3$ which has been extensively studied in the Computational Geometry
literature.  See \cite{ECGBook} for a recent survey.  These methods
start by computing some subdivision of the embedding space (such as a
grid or a triangulation of the sample points)  and their
direct extension to higher dimensions would face an exponential
dependence on $d$. A step in this direction is the extension of the
celebrated Marching Cube algorithm to manifolds of higher
dimensions~\cite{marching-cube1,isosurface}.  Continuation methods do
not use any subdivision of the ambient space and are close in spirit
to our approach. They construct a triangulated approximation of a
$k$-dimensional submanifold in a greedy way and extend the current
$k$-dimensional triangulated domain by adding a neighborhood of a
boundary point. Some experimental results can be found in \cite{henderson} but
no theoretical analysis of continuation methods is available.

=================RECONSTRUCTION

Manifold reconstruction consists of computing a piecewise linear
approximation of an unknown manifold $\M \subset \R^d$ from a finite
sample of unorganized points $\pp$ lying on $\M$ or close to
$\M$. When the manifold is a two-dimensional surface embedded in
$\R^3$, the problem is known as the surface reconstruction
problem. Surface reconstruction is a problem of major practical
interest which has been extensively studied in the fields of
Computational Geometry, Computer Graphics and Computer Vision.  In the
last decade, solid foundations have been established and the problem
is now pretty well understood. Refer to Dey's book \cite{bookdey}, and
the survey by Cazals and Giesen in \cite{book1} for recent
results. The output of those methods is a triangulated surface that
approximates $\M$. This triangulated surface is usually extracted from
a 3-dimensional subdivision of the ambient space (typically a grid or
a triangulation). Although rather inoffensive in 3-dimensional space,
such data structures depend exponentially on the dimension of the
ambient space, and all attempts to extend those geometric approaches
to more general manifolds have led to algorithms whose complexities
depend exponentially on
$d$~\cite{manifold3, manifold4,manifold2,homology1}.

The problem in higher dimensions is also of great practical interest
in data analysis and machine learning. In those fields, the general
assumption is that, even if the data are represented as points in a
very high dimensional space $\R^d$, they in fact live on a manifold of
much smaller intrinsic dimension~\cite{seung-lee}. If the manifold is
linear, well-known global techniques like principal component analysis
(PCA) or multi-dimensional scaling (MDS) can be efficiently
applied. When the manifold is highly nonlinear, several more local
techniques have attracted much attention in visual perception and many
other areas of science. Among the prominent algorithms are
Isomap~\cite{isomap}, LLE~\cite{lle}, Laplacian
eigenmaps~\cite{laplacian}, Hessian eigenmaps~\cite{hessian},
diffusion maps~\cite{diffusion,diffusion1}, principal
manifolds~\cite{principal-manifolds}. Most of those methods reduce to
computing an eigendecomposition of some connection matrix. In all
cases, the output is a mapping of the original data points into $\R^k$
where $k$ is the estimated intrinsic dimension of $\M$.  Those methods
come with no or very limited guarantees. For example, Isomap provides
a correct embedding only if $\M$ is isometric to a convex open set of
$\R ^k$ and LLE can only reconstruct topological balls. To be able to
better approximate the sampled manifold, another route is to extend
the work on surface reconstruction and to construct a piecewise linear
approximation of $\M$ from the sample in such a way that, under
appropriate sampling conditions, the quality of the approximation can
be guaranteed. First investigations along this line can be found in
the work of Cheng, Dey and Ramos \cite{manifold2}, and Boissonnat,
Guibas and Oudot \cite{manifold3}. In both cases, however, the
complexity of the algorithms is exponential in the ambient dimension
$d$, which highly reduces their practical relevance.

In this paper, we extend the geometric techniques developed in small
dimensions and propose an algorithm that can reconstruct smooth
manifolds of arbitrary topology while avoiding the computation of data
structures in the ambient space.  We assume that $\M$ is a smooth
manifold of known dimension $k$ and that we can compute the tangent
space to $\M$ at any sample point. Under those conditions, we propose
a provably correct algorithm that %\framebox{\text{\color{red}{Changed}}} %allows to 
construct a simplicial
complex of dimension $k$ that approximates $\M$. The complexity of the
algorithm is linear in $d$, quadratic in the size $n$ of the sample,
and exponential in $k$.  Our work builds on \cite{manifold3} and \cite{manifold2} 
but dramatically reduces the dependence on $d$. To
the best of our knowledge, this is the first certified algorithm for
manifold reconstruction whose complexity depends only linearly on the
ambient dimension. In the same spirit, Chazal and Oudot
\cite{persistence} have devised an algorithm of intrinsic complexity
to solve the easier problem of computing the homology of a manifold
from a sample.

Our approach is based on two main ideas~: the notion of {\it
  tangential Delaunay complex} introduced in %\framebox{\text{\color{red}{Changed: references rearranged}}}
\cite{coordinate-system,thesis1,freeman}, and the technique of sliver
removal by weighting the sample points \cite{sliver1}. The tangential
complex is obtained by gluing local (Delaunay) triangulations around
each sample point. The tangential complex is a subcomplex of the
$d$-dimensional Delaunay triangulation of the sample points but it can
be computed using mostly operations in the $k$-dimensional tangent
spaces at the sample points. Hence the dependence on $k$ rather than
$d$ in the complexity. %\framebox{CHECK}  
However, due to the presence of so-called
inconsistencies, the local triangulations may not form a triangulated
manifold. Although this problem has already been reported \cite{freeman}, no
solution was known except for the case of curves ($k=1$)
\cite{thesis1}.
The idea of removing inconsistencies among local triangulations that
have been computed independently has already been used
for maintaining dynamic meshes \cite{starsplaying} and generating anisotropic
meshes~\cite{anisotropic1}. Our approach is close in spirit to the one
in ~\cite{anisotropic1}.
We show that,  under appropriate sample conditions, we can remove inconsistencies by weighting the
sample points. We can then prove
that the approximation returned by our algorithm is ambient isotopic to $\M$,
and a close geometric approximation of $\M$.

Our algorithm can be seen as a {\em local} version of the cocone
algorithm of Cheng et al. \cite{manifold2}. By local, we mean that we
do not compute any $d$-dimensional data structure like a grid or a
triangulation of the ambient space. Still, the tangential complex is a
subcomplex of the weighted $d$-dimensional Delaunay triangulation of
the (weighted) data points and therefore implicitly relies on a global
partition of the ambient space. This is a key to our analysis and 
% \framebox{\text{\color{red}{Changed}}} %makes
distinguishes our method %depart 
from other local algorithms that have been proposed
in the surface reconstruction literature \cite{prisme-4564a,gopi}.

\subsection*{WP 2:  Dimension-sensitive algorithms and data structures} 

Central to the techniques we intend to develop is the construction of simplicial complexes.  A graph is an example of a 1-dimensional simplicial complex but simplicial complexes are much more powerful than graphs and allow to approximate complicated shapes of arbitrary dimension and topology. They offer a flexible data structure to represent and process higher-dimensional shapes and recent developments have shown that simplicial complexes computed on top of point clouds are primary tools to capture the topology of the underlying space of the data. 

In 3-dimensions, 2 and 3-dimensional simplicial complexes of surfaces and volumes are widely used in graphics, scientific computing and manufacturing. Because of its numerous interesting properties and of the existence of extremely efficient algorithms to compute it, the Delaunay triangulation has become one of the most famous and widely used geometric data structures that spread out accross all sciences. The algorithms we have implemented in CGAL are among the most reliable and fast algorithms. They have been included in the heart of MATLAB. 

The algorithms used in 3 dimensions extend to any dimension but their complexity grows exponentially with the dimension which makes them useless for real applications beyong say dimension 6~\cite{avis,hornus}.  In order for algorithms and implementations to scale with the dimension, we need to exploit (hidden) structure of the data and to design dimension-sensitive algorithms and data structures.

% our focus is not on worst-case complexity, but on (provably) good performance under some given structural properties of the input. These properties may be of statistical nature (when we are dealing with noise, for example), or of geometric nature (when data is of low intrinsic dimension, say). Related to this, we are also aiming at output-sensitive algorithms.

\paragraph{Design of small yet faithfull simplicial complexes.} 
Given a set of points $V$ in $\R ^d$, a number of simplicial complexes
with vertex set $V$ have been proposed. A first class of simplicial
complexes uses a parameter $\alpha$ which can be used to order the
simplices of the complex.

The \u{C}ech complex is the nerve of the set $B_{\alpha}$ of balls of
radius $\alpha$ centered at the points of $V$. The nerve of
$B_{\alpha}$  associates a
$i$-simplex to any subset of $i+1$ balls that have a common
intersection. This is a simplicial
complex that is in general not embeddable in $\R ^d$. Moreover, it is usually very big and
difficult to compute since it requires to detect whether a subset of
balls of  $\R ^d$ intersect. 

A simpler to compute simplicial complex is the Rips complex whose
edges are the same as for the \u{C}ech complex. The higher dimensional
simplices of the Rips complex are obtained by computing the cliques of
sizes 3, 4 etc. in the graph of the edges. This simplicial complex is
much easier to compute than the \u{C}ech complex and it has the
vremarkable property that it can be constructed in a purely
combinatorial way from 
its 1-skeleton.  Such a simplicial complex is called a {\em flag
  complex}. Nevertheless, the Rips complex is not embedded in $\R ^d$
and may have a dimension much higher than the dimension of the underlying structure
of the data.


Various simplicial complexes have been derived from the Delaunay
triangulation of the vertices. The $\alpha$-complex is the nerve of
the restriction of the Delaunay triangulation to the union of the
balls of $B_{\alpha}$. This complex is embedded in $\R^d$ (provided
that the vertices are in general position) but very difficult to
compute in high dimension for the same reason as the \u{C}ech complex.

Other simplicial complexes derived from the Delaunay triangulation do
not involve any parameter, most notably the restricted Delaunay
triangulation, the tangential Delaunay complex and the witness
complex. Those complexes are
especially designed for the case where $V$ samples a topological space
of small dimension $k$, the central hypothesis in Machine
Learning. Both the restricted and the tangential Delaunay complexes are embedded in $\R^d$, have dimension $k$
(under a mild general position assumption). Still, these simplicial
complexes are limited to small $k$.  The witness complex is
another variant of the Delaunay triangulation introduced by
Vin de Silva and Carlsson. The witness complex is embedded in $\R ^d$
and is remarkably easy to compute in any dimension
since the only numerical operations involves in its construction are
comparisons of distances.

It should be noted that the Rips complex and the witness complex can
both be computed from the knowledge of the distances between the
vertices. Hence these complexes can be computed in any discrete metric
space.

 Currently no code allows to manipulate simplicial complexes of arbitrary dimension in a routine way as is possible for 2 and 3-dimensional triangulations of $\R ^3$ \cite{springerflo,DBLP:journals/tog/PaoluzziBCF93,svy-crm-99}. 

We identify four research directions~:
\begin{enumerate}
\item Classifying simplicial complexes
\item Combinatorial and algorithmic complexity 
\item Compact representation
\item New types of simplicial complexes
\end{enumerate}

\paragraph{Classifying simplicial complexes.}
Some equivalences between the various types of simplicial complexes are known. For example,
the Rips and the \u{C}ech complexes are identical for the $L_1$ norm and for the Euclidean norm, we have 
\[ \rips () \subset \cech () \subset \rips () .\]
Recently, we have established conditions under which the witness complex, the restricted Delaunay triangulation and the tangential complex are identical~\cite{}. A more complete classification is required to better understand these structures and their properties. 
It would also lead to  better algorithms.



\paragraph{Combinatorial and algorithmic complexity.}
A main limitation of using simplicial complexes is their combinatorial and algorithmic complexity.  Differently from polytopes, very few results are known. The flag random complex is a noticeable exception~\cite{}. Other types of random abstract complexes have to be studied from a combinatorial point of view. Geometric simplicial complexes should also be considered.  An especially important question is to obtain complexity bounds for simplicial complexes of well sampled substructures (e.g. submanifolds).  We intend to measure the effect of perturbations (either noise or computed perturbations) on the mathematical properties and combinatorial complexity of those structures, and to develop probabilistic analyses. In addition to their combinatorial complexity, the complexity of the construction of the simplicial complexes is to be analyzed.  Parallel and out-of-core algorithms will be also developed.


\paragraph{Compact representation of simplicial complexes.} We are aware of only a few works on the design of data structures for general simplicial complexes. Brisson~\cite{Brisson:1989:RGS:73833.73858} and Lienhardt~\cite{DBLP:journals/ijcga/Lienhardt94} have introduced data structures to represent $d$-dimensional cell complexes, most notably subdivided manifolds. While those data structures have nice algebraic properties, they are very redundant and do not scale to large data sets or high dimensions. More recently, Attali et al.~\cite{Attali2011} have proposed an efficient data structure to represent and simplify flag complexes, a special family of simplicial complexes including the Rips complex. 
Recently, we have experimented with a tree representation for general simplicial
complexes that seems to perform very well. Simplicial complexes of 500 millions of simplices have been constructed and stored on a laptop~\cite{}. 
Theoretical guarantees and experiments on a large scale are mandatory. In
addition, more compact storage could be further obtained by using
well-known succinct representations of trees~\cite{10.1109/SFCS.1989.63533,Munro:2002:SRB:586840.586885,Ferragina:2005:SLT:1097112.1097456,DBLP:conf/icalp/2003}. The problem of finding minimal representations of simplicial complexes is widely open beyond the planar case.


\paragraph{New types of simplicial complexes.}
Among the simplicial complexes discussed above, only the Rips and the witness complexes can be constructed on a discrete metric space where only the distances between points are known. The Delaunay-like simplicial complexes are based on a distance function, usually the Euclidean distance.  Other distance functions and, in particular, geodesic distances would lead to Intrinsic simplicial complexes. First encouraging results in this direction can be found in \cite{}.


\paragraph{Validation.} \framebox{in WP 4?}
We intend to {\em implement} those structures, experiment with them and see how they behave under realistic conditions. We will use known datasets such as the UCI repository \cite{} and also propose new benchmarks that we will make publicly available with the hope that they will be used to compare data structures and algorithms.

\paragraph{Dimension-sensitive algorithms and data structures} \framebox{WP1?}









\subsection*{WP 3 :  Robust geometric models}
Geometric inference. Feature extraction. Persistent homology.  Stability with respect to perturbation of the data. Topology preserving approximation/simplification. Clustering. Applications in Data Analysis, Computer Vision, Numerical Simulation, Robotics, Molecular Biology.


\subsection*{WP4 : A software platform for geometric understanding in high
  dimension}

We intend to develop an open source  software platform that will provide a
comprehensive and robust set of tools for geometric understanding in
high
dimension. 

\paragraph{The need for a software platform.}
On one hand, we perceive the development of such a platform as 
an absolute necessity to serve as a test bench and  evaluation process
for any new algorithmic solution resulting  from  our theoretical work.
On the other hand, we perceive the development of such a platform as
a full research work. Indeed we are convinced that, as robustness
issues triggered the development of a whole branch of theoretical
work, known as Geometric Computing, the need for
highly efficient implementations to turn around the curse of  
dimensionality,  will participate in the emergence of relevant
concepts for new algorithmic foundations
of  geometric understanding in high dimension.


The term platform means that we intend to capitalize upon development
efforts and encourage contributions from researchers external to the
project.  At the end of this project, the platform will offer a
comprehensive and robust set of algorithmic tools for geometry
processing and analysis in higher dimensional spaces.

We also think about this platform as the choice
vector for the diffusion of our algorithmic solutions 
into application domains, such as astrophysics
or structural biology) 
 where the need for handling high dimensional
data crucially arises.  Such a diffusion in various fields
 will in turn provide for various bench set data 
and user feedback.



\paragraph{State of the art.}  
Leaving aside the flourishing field of machine learning
and well-known successful software for linear algebra,
linear and quadratic optimization,  only  few implementations handle
geometry in high dimension. 
Most of those software use multiscale  grids based
data structures that somehow adapt
 to the local  density of data.
A typical example is the popular ANN software to compute  the approximate
nearest-neighbor of a query point
among a  high dimensional point cloud. 
Qhull is a software that can compute convex hulls and Delaunay
triangulations in dimensions larger than 3, but it doesn't see much
development anymore, and as the authors prominently announce on the
webpage: ``Qhull does not support triangulation of non-convex surfaces,
mesh generation of non-convex objects, medium-sized inputs in 9-D and
higher, alpha shapes''. The polymake framework has many features,
it can handle several types of complexes, build Voronoi diagrams and
compute advanced topological characteristics of objects like a finite
representation of the fundamental group. However, it is strongly
oriented towards an interactive use for mathematical experimentation on
a given object and not automated, fast and robust data processing.
% C'est l'impression que j'en ai apres avoir un peu surfe sur differents
% sites, mais je n'ai pas une confiance absolue dans ce que je dis
% ci-dessus.
Only two implementations of persistent homology algorithms are
currently available. One of them is the PLEX package for Matlab,
developed by the Computational Topology group at Stanford University.
The other one is the
library Dionysus proposed by Dmitriy Morozov. These implementations
%do not propose parallel nor out-of-core versions. They 
are known to be
successful in small dimensions but inefficient as soon as the
dimension rises.  Their  maintenance, only assumed by the very few authors
is likely not to be perennial.

%\framebox{Continuation methods : Multifario http://www.research.ibm.com/people/h/henderson/Continuation/ContinuationMethods.html}


\paragraph{Methodology.} 
Two key requirements in the development of this platform will be
efficiency and theoretical guarantees, including handling of
robustness issues. A common
platform is ideal for these goals, as it allows for more
resources to be poured into the optimization, bug-fixing and interface
design than for a single prototype.

 We intend to apply to the development of this software platform 
the very  recipes that made up the success story
of the CGAL library. 
The development work will be based on a strong infrastructure
including a web site, a svn repository and  daily running of testsuites on several  hardware architectures.
Above all we will set up  an editorial board assuming a  serious review of the 
specifications of packages proposed for inclusion in the platform.
We forecast that the platform will reach a sufficient critical mass
to be of interest for a large community of researchers in
computational geometry and neighboring applicative fields,
which in turn will ensure the long range perennial maintenance
of the software.



\paragraph{Planned developments} 
\begin{itemize}
\item The software platform will provide tools to extract from clouds of points the
simplicial complexes that are relevant for geometric understanding,
Rips, tangential Delaunay  and witness complexes to begin with. 
\item The platform will provide efficient data structures to handle those
complexes. We will in particular  focus on parsimonious data
structures, aiming for a partially implicit representation of those simplicial
complexes, thus avoiding the full space cost entailed by the curse of
dimensionality. 
\item The platform is meant to offer state of the art algorithmic tools for geometric
understanding,
including in particular algorithms to
\begin{itemize}
\item  mesh or reconstruct manifolds
\item  compute the persistent homology of a simplicial complex filtration 
\item cluster data
\item compute signatures of shapes
\item visualization tools \framebox{?}
\end{itemize}
\item The platform will aim at providing parallel implementations as well
as out-of-core versions whenever possible to make possible the
handling of huge data sets in high dimensions.
\end{itemize}

\framebox{Datasets}

\framebox{Diffusion}

%  It will namely include tools to extract from
% point clouds various simplicial complexes and filtrations,
%  like Rips, Cech or witness
% complexes, that are relavant for data analysis and topological feature
% extraction. It will provide data structures to handle those simplicial
% complexes.
% We will focus on parsimonious data structures that will contribute
% to turn around the curse of dimensionality. Particularly promising
% are data structure involving a  partial implicit representation of those simplicial
% complexes. The platform will also offer 
% robust and efficient implementations for persistent homology algorithms
% and other state of the art algorithms arising from
% work in this ERC and from neighboring groups. 

% \thispagestyle{empty}
% \pagestyle{myheadings}
% \markboth{\tcg{Titre ou acronyme de l'ADT}}{\tcg{Titre ou acronyme de l'ADT}}

% %\renewcommand{\thefootnote}{\fnsymbol{footnote}}
% %\setcounter{footnote}{1}
% %\renewcommand{\thefootnote}{\arabic{footnote}}
% %\setcounter{footnote}{0}

% \begin{center}
% {\LARGE\bf
% \tcg{Titre de l'ADT\footnote{
% Tout ce qui est en \tcg{vert} explique ce qui est attendu et est \`a remplacer
% par le texte appropri\'e \`a votre ADT. {\bf Merci d'enlever tout le \tcg{vert} au
% moment de la soumission.}
% }
% }}\\[1ex]
% \Large\bf
% Campagne ADT 2012\\
% \end{center}

% \renewcommand{\thefootnote}{\arabic{footnote}}
% %\setcounter{footnote}{0}
















\bibliographystyle{abbrv}
\bibliography{erc}

\section{Resources}

I will devote 70\% of my time to this project, and I will dedicate all my expertise and efforts to conduct and supervise the research work. To this end, I will receive the precious help of 2 permanent researchers of the Geometrica team : Fr\'ed\'eric Chazal who is a  leading researcher in geometric inference and computational topology and Mariette Yvinec who is an expert in geometric computing and a member of the CGAL Editorial Board. They will devote 20\% of their time to this project to co-supervise with me the research and implementation work of the students, postdocs and engineers to be engaged in this project. Other members of Geometrica,
not financially supported by this project, will also collaborate to the project.



\end{document}
\newpage

\appendix

Geometric data are ubiquitous. They arise from measurements and simulations in the natural sciences, and from images, shapes, and text that can easily be converted into geometric point cloud data. Most of these data reside in high-dimensional spaces. For example, configuration spaces of robots with many degrees of freedom, or configuration spaces of macromolecules are subspaces of high-dimensional real vector spaces. Natural and artificial systems like biological or sensor networks are often described by a large number of real parameters, whereas a collection of text documents can be represented as a set of term frequency vectors in Euclidean space; similar interpretations can be given for image, audio, and video data. Hence, processing and analyzing geometric data in high-dimensional spaces is a core task in science and engineering.

At the heart of our proposed project is the insight that most of these data are structured, although this intrinsic geometric structure is often not easy to capture. Our long-term vision is to have efficient and reliable methods for geometric data analysis that find and exploit hidden structure and by that lead to fast and robust geometric data processing in high dimensions. In this project, we aim at laying the foundations of a new field—computational geometric learning— that provides efficient and reliable methods for geometric data analysis.

In the past, an important focus of computational geometry was the design and analysis of exact low-dimensional geometric algorithms and data structures. This focus is well reflected in one of the major success stories of computational geometry, namely the Computational Geom- etry Algorithms Library (CGAL) that provides easy access to efficient and reliable geometric algorithms in the form of a C++ library. In most cases, reliability is achieved through provable correctness guarantees.

Our proposed project aims at extending this success story—combining efficiency with cor- rectness guarantees—to high-dimensional geometric computation. This is not a straightforward task. For many problems, no efficient algorithms exist that compute the exact solution in high dimensions. Hence, there is a need for fast approximate solutions, but we do not want to sacrifice guarantees. The following two kinds of approximation guarantees are particularly desirable: first, the solution approximates an objective better if more time and memory re- sources are employed (algorithmic guarantee), and second, the approximation gets better when the data become more dense and/or more accurate (learning theoretic guarantee). Efficient and practical algorithms and data structures with such guarantees are indispensable for building robust software in the spirit of our long-term vision. With respect to this vision, the targeted breakthroughs of our project are

(i) the design and analysis of efficient and reliable algorithms and data structures for relevant high-dimensional data processing tasks; we aim at theoretical guarantees and efficiency, by exploiting intrinsic structure of the data;

(ii) the implementation of algorithms and data structures for high-dimensional geometric data processing tasks within CGAL, and

(iii) fast and robust application software (building on the fundamental algorithms and data structures) for specific important problems in computational biology, robotics and shape analysis.

=====================

(Maggioni) Our research will focus on three related aspects, each of which is important and of independent interest: Key issues: definition of local similarities, dimensionality reduction, parametrizations of the data, stability with respect to perturbation of the data (e.g. measurement noise, instrument normalization etc...). Properties of functions on the data.





% Carlsson: For example, it is now often the case that we are given data in the form of very long vectors, where all but a few of the coordinates turn out to be irrelevant to the questions of interest, and further that we don’t necessarily know which coordinates are the interesting ones. A related fact is that the data is often very high-dimensional, which severely restricts our ability to visualize it. The data obtained is also often much noisier than in the past and has more missing information (missing data). This is particularly so in the case of biological data, particularly high throughput data from microarray or other sources. Our ability to analyze this data, both in terms of quantity and the nature of the data, is clearly not keeping pace with the data being produced. 
In this paper, we will discuss how geometry and topology can be applied to make useful contributions to the analysis of various kinds of data. Geometry and topology are very natural tools to apply in this direction, since geometry can be regarded as the study of distance functions, and what one often works with are distance functions on large finite sets of data. The mathematical formalism which has been developed for incorporating geometric and topological techniques deals with point clouds, i.e. finite sets of points equipped with a distance function. It then adapts tools from the various branches of geometry to the study of point clouds. The point clouds are intended to be thought of as finite samples taken from a geometric object, perhaps with noise. 

Carlsson : This project will develop topological tools for understanding qualitative properties of data sets. We will use homology as applied to data sets directly and to derived complexes to define invariants or signatures that distinguish between the underlying geometric objects. Important goals will include the identification, location, and classification of qualitative features of the data set, such as the presence of corners, edges, cone points, etc. and the use of homology applied to canonically defined blowups and tangent complexes to distinguish between two dimensional shapes in three dimensional Euclidean space. We will use the recently developed techniques of persistence and landmarking to make homology a stable and readily computable invariant. We will also develop the theory of multidimensional persistence, in which one studies spaces that are equipped with several parameters, in order to better understand data sets in which there are several different parameters describing different geometric properties of the space. The overall goal is to continue to develop and improve the available tools for studying qualitative information about geometric objects. The goal of this project is to develop tools for understanding data sets that are not easy to understand using standard methods of statistics and analysis. This kind of data might include singular points, or might be strongly curved. The data is also high dimensional, in the sense that each data point has many coordinates. For instance, we might have a data set whose points each of which is an image, which has one coordinate for each pixel. Many standard tools rely on linear approximations, which do not work well in strongly curved or singular problems. The kind of tools we have in mind are in part topological, in the sense that they measure more qualitative properties of the spaces involved, such as connectedness, or the number of holes in a space, and so on. For example, the project takes the point of view that it is better to understand qualitative properties before attempting to do more precise quantitative analysis and better to distinguish shapes by understanding them qualitatively rather than doing data base comparisons. Thus, methods will be developed to compute, in a timely, robust, and trustworthy manner, the fundamental geometric properties that any realistic mathematical model associated to a given data set must contain. Then statistical and analytic techniques may be applied to the geometrically correct models in order to extract the detailed information desired by practitioners.

====================

Manifold learning and dimensionality reduction: In the last decades, a set of new geometric methods, known as manifold learning, has been developed in the machine learning community. They are based upon the assumption that the data lie on a submanifold M in Rd and they mainly focus on nonlinear dimensionality reduction of the data set. Some of them, like principal curves [HS89] or generative topographic mapping (GTM) [BSW98], aim at fitting the data set by a parameterized low dimensional (in general 1 or 2) manifold. They usually assume that the topology of the manifold is known and simple (simple curves, planes, discs) and do not deal with data sampling more complicated shapes. Other methods, such as Multidimensional Scaling, LLE or ISOMAP[RS2000, RS2003, TSL2000], intend to find “projections” of the data onto lower dimensional Euclidean spaces that respect various geometric properties of the data. Again, these methods assume very restrictive hypothesis on the geometry of the manifolds sampled by the points to ensure correctness. Geometric diffusion and computational harmonic analysis methods have also been proposed to recover geometric properties of the underlying manifold [CLLMNWZ05]. These approaches consist of projecting the data set onto a low dimensional space generated by the first eigenvectors of a diffusion operator defined on a graph built from the data. They are rather sensitive to the quality of the sampling, and the geometric interpretation of the properties of the projected data is not always clear. From a computational point of view, a common drawback of these methods is that they usually involve computations of eigenvalues and eigenvectors of matrices of the size of the data set, preventing to deal with huge data sets without a pre-processing.

============
The recognition of objects in potentially complex scenes is one major issue in Computer Vision, that weights billions of dollars each year [Lowe]. Prominent advances integrate two essential stages: the induction of features of limited size, over a potentially huge feature space, and the use of these features for learning and classification. This approach is extremely popular, as it is also supported by biological evidences for the hierarchical treatment of vision, beginning from low­level descriptors [Poggio]. Biological evidences also support the integration of particular distortions in vision, that are precisely Bregman divergences [Poggio]. In fact, distortions are used throughout all the process of object recognition in vision: the detection of features using statistical, algebraic or geometric techniques, and the use of these features throughout classifiers.

=============

To achieve this task, we intend to take advantage of recent developments in the theory of topological persistence, which abstract themselves from the classical Euclidean setting and consider the input data as mere point clouds coming from finite or compact metric spaces. On the algorithmic side, we will need to use data structures that do not rely on a particular embedding of the data, but rather on its intrinsic metric. Such structures include the Rips and witness complexes, among others, and studying or limiting their complexity in general settings will be one of the main challenges, alongside with providing lightweight and efficient implementations for practical use

===============

Our long-term vision is to settle   the foundations and develop the algorithmic infrastructure 
on which applications of Geometric Modeling in higher dimensions will be made possible.

================

We believe such pursuits can be the key to breaking current and future computational bottlenecks in many areas of engineering. 

================

Computational geometry enjoys two unique assets: (1) its diversity and potential to affect most forms of computing; (2) its mature algorithmic foundations. The challenge is now to make this potential come true, and build an effective pipeline connecting theory to practice. We believe that this will revitalize the field and open new vistas for geometric research, both of a practical and a theoretical nature. The solutions of the most exciting open problems might be unknown. But the most exciting open problems themselves might be unknown, too. We are confident that creating the pipeline will unveil many of these problems. 


===============

A distinctive feature is the design and implementation of novel algorithmic solutions with certified topol- ogy and numerics as an alternative for heuristics and ad hoc methods, and the development of an experimental geometry kernel for modeling and computing with complex shapes as a proof-of-concept justifying our ap- proach. The results of this project should be directly useful to the application areas mentioned above. We intend to disseminate our work by publication in the appropriate applied research forums, by organizing multidisciplinary workshops aimed at exchange of knowledge and discussion of our work. Moreover, we aim at transferring our new technology by producing high quality software, demonstrating the feasibility of our techniques in practice. Cooperation with our industrial partner includes the assessment, trial, validation and packaging of the software developed in the project, thus guaranteeing a smooth transfer of new technology to application areas.

\end{document}
