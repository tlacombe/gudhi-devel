\thispagestyle{empty}

\mbox{}\vspace{-3.5cm}

\begin{center}
{\Large
{\bf ERC Advanced Grant 2012 \\ research proposal (PART B2)}}
\vspace{1cm}

{\LARGE {\bf  Algorithmic Foundations \\ of 
Geometry Understanding in Higher Dimensions}

\vspace{3mm} 

%{\bf (GUDHI)}
}
\end{center}
\section{The research proposal}

The central focus of this  proposal is  the computer analysis of geometric structures, which we refer to as {\em geometry understanding}. Our ambition is to settle the {\em algorithmic foundations} of geometry understanding  {\em in dimensions higher than 3}. We intend to develop new techniques to approximate highly nonlinear shapes, and to infer geometric and topological properties from data subject to significant {\em defects} and under {\em realistic conditions}. 
We will design {\em scalable representations} of complex shapes,  {\em practical and provably correct algorithms}, and we will deliver {\em perennial software}.

We now present the state of the art and the scientific objectives of our proposal that mirror its four main scientific challenges~: {\em choosing the right representation, bypassing the curse of dimensionality, searching for stable models, and turning theory into practice.}


\subsection{State of the art and scientific objectives}
%The last decade has seen tremendous progress in the understanding of geometry in high-dimensional spaces. 

\paragraph{Dimensionality reduction.} A widely accepted assumption in scientific computing and data analysis is that the objects of interest can be modeled as {\em low-dimensional manifolds}, even if they are embedded in high-dimensional ambient spaces. Low here is to be understood with respect to the ambient dimension and may be significantly higher than 3. This powerful assumption is supported by the fact that data are most of the time associated with physical systems that have relatively few parameters.  This assumption is valid accross science and engineering and is at the heart of dimensionality reduction and manifold learning.
%Examples can be found in fields as varied as chemistry~\cite{mtcw-tco-2010}, %neurosciences~\cite{} and  image processing~\cite{cids-lbsni-2008}.

{\em Dimensionality reduction} techniques intend to infer the intrinsic dimensionality of the data, as well as to provide structure-preserving mappings of the data into lower-dimensional spaces. {\em Nonlinear} dimensionality reduction techniques~\cite{lv-nldr-2007} are capable of discovering {nonlinear} structures and have been successfully applied to analyze data in a wide range of applications.
Nevertheless, the methods come with no or very limited guarantees. For example, Isomap~\cite{tsl-isomap-2000} provides a correct embedding only if the manifold is isometric to a convex open set of $\R ^k$, where $k$ is the estimated dimension of the manifold, and LLE~\cite{rs-lle-2000} can only reconstruct topological balls. {\em Topological methods} are complementary to dimension reduction methods. They intend to better approximate manifolds by constructing piecewise linear approximations, most of the time  in the form of simplicial complexes.

%However, they are only guaranteed to work on data sets sampled from manifolds with low curvature and trivial topology.
\vspace{2mm}  

{\bf Central objective :} {\em To develop geometric and topological methods for geometry understanding in higher dimensions.  To promote the use  of simplicial complexes as an expressive and powerful representation of shapes. }

\paragraph{Simplicial complexes.}  Simplicial complexes can be defined in any dimension and can be used as piecewise linear approximations of general shapes.
They have been extensively used in $\R ^3$ where surfaces or volumes are approximated by 2-dimensional or 3-dimensional simplicial complexes~\cite{geometrica-ecg-book}. 

The situation is quite different in higher dimensions. While simplicial simplexes have been studied for a long time by the mathematical community, the algorithmic side of the theory is still in its infancy. Most of the known simplicial complexes are either extremely difficult to compute even, in moderate dimensions, because their construction involves high-degree algebra (e.g., the  \v{C}ech complex) or easy to compute but so big that they cannot be constructed from real high-dimensional data (e.g., the Rips complex). Current research aims at finding new types of simplicial complexes that are easy to compute while still capturing the essential geometric and topological features of shapes. Recent progress has been made with the discovery of Delaunay-like complexes such as the witness complex~\cite{cds-tewc-2004} and the Delaunay tangential complex~\cite{geometrica-7142i} that offer new compromises between complexity and approximation quality.

 
Another issue is the {\em computer representation} of simplicial complexes. 
Little work has been done on the design of data structures for general
simplicial complexes. Brisson~\cite{Brisson:1989:RGS:73833.73858} and
Lienhardt~\cite{DBLP:journals/ijcga/Lienhardt94} have introduced data
structures to represent $d$-dimensional cell complexes. These data
structures are general and powerful, but very redundant, and they do
not scale to large data sets in high dimensions.  In contrast, one
could store only the 1-skeleton of the complex (edges and vertices). This 
saves a lot of memory space at the penalty of having to reconstruct the full complex when needed, e.g., to store a filtration. This can be done in a purely combinatorial way, thus very efficiently, in the special case of flag complexes (among which the Rips complex is a prominent example). Elaborating on this idea, Attali et al.~\cite{Attali2011} have proposed a concise data structure that is efficient when applied to simplicial complexes that are {\em close} to flag complexes. Still, designing efficient data structures to represent simplicial complexes is a widely open area.

% is enough to recover the full complex in the case of fl

% the special case of flag complexes (among which the Rips complex is a prominent example) are attractive since the full combinatorial structure is entirely determined by its 1-skeleton. This saves memory space at the penalty of having to reconstruct the full complex when needed, e.g., to store a filtration. Recently, Attali et al.~\cite{Attali2011} have proposed an efficient data structure to represent simplicial complexes that are close to flag complexes.
\vspace{2mm}

{\bf Objective 1:} {\em 
To extend  current knowledge of simplicial complexes, most notably their combinatorial and algorithmic properties.   To fully understand Delaunay-like simplicial complexes.  To design new  compact representation of simplicial complexes.}

\paragraph{Geometric approximation.}
In low dimensions, computing an approximation of a given geometric
object is a well-understood problem and good approximations can be
efficiently constructed~\cite{geometrica-bcmrv-ms-06,he-gtmg-2001}.
The situation is quite different in higher dimensions.  Although the
mathematical literature on triangulation of manifolds is abundant, few
effective algorithms have been proposed and tested.  The main issue is
to avoid computing a subdivision of the ambient space since this would lead to algorithms that scale exponentially with the ambient dimension. Very few attempts have been made to exploit the  low-dimensional manifold assumption.
To analyze {dynamical systems} in science and engineering, higher-dimensional {\em continuation methods} have been proposed to mesh solution manifolds~\cite{mh-mpc-2002}. These methods are however lacking guarantees and are restricted in practice to low dimensional manifolds. 
Recently, we made progress towards a better understanding of the complexity issues of triangulating smooth submanifolds and provided a provably correct algorithm that scales {\em linearly} with the ambient dimension~\cite{boissonnat2010meshing}. 

A natural way to bypass the curse of dimensionality is to directly
work on the manifold and to resort to {\em intrinsic geometry}. What
is lacking is a full development of Riemannian computational
geometry. A central question in our context, which has been elusive
for a long time, asks for the existence of Delaunay triangulations on
Riemannian manifolds. Previously announced sampling
criteria~\cite{leibon2000} for the existence of intrinsic Delaunay
triangulations have recently been demonstrated to be insufficient, and
sufficient criteria were introduced~\cite{boissonnat2012stab}. This opens exciting avenues for new developments on Delaunay triangulations in Riemannian manifolds and other non Euclidean spaces. Once again, what is profoundly lacking is the algorithmic side of the theory and the development of efficient, possibly approximate, algorithms.  A first promissing step towards this goal is an anisotropic meshing algorithm we proved correct and implemented~\cite{bwy-luam-08}.

Another fundamental problem is {\em manifold reconstruction}.  In low dimensions, effective reconstruction techniques exist that can provide a faithful approximation of a geometric structure from point samples~\cite{dey-csr-2007}. % Further processing makes it possible to study their topological and geometric properties.
Almost all these methods rely on the computational of a triangulation of the ambient space and previous attempts to extend them to higher dimensions led to algorithms whose complexities are exponential in the ambient dimension.  Despite some very recent results~\cite{geometrica-7142i}, designing practical algorithms that can reconstruct submanifolds of high-dimensional spaces under mild sampling conditions remains widely open. Extension beyond smooth manifolds 
is even more challenging.

% n addition to the computational bottleneck, 
%  the data often suffers from significant defects, including sparsity, noise, and outliers, violating sampling conditions required by extant methods. %  since their complexity depends exponentially on the ambient dimension

\vspace{2mm}

{\bf Objective 2 :}  {\em To   develop new algorithms to {\em  triangulate non Euclidean geometric spaces}. To study intrinsic Delaunay triangulations of Riemannian manifolds.  To reconstruct submanifolds using Delaunay-like simplicial complexes. To construct crude approximations 
with quality guarantees. To extend current knowledge on processing stratified manifolds. }

\paragraph{Geometric and topological inference.}
Since computing precise approximations is currently only possible under strong assumptions that may not be met in some applications, we can look for cruder approximations 
that can still  uncover some of the properties of the structures underlying the data.
%
%is not always mandatory. In some applications,   useful approximations can be obtained at a lower computational cost.
% In {\em robotics}, the goal is to capture the connectivity of configuration spaces and to search paths. Randomized techniques have been proved to be quite successful in constructing graphs to approximate the space of free configurations of robots~\cite{sml-pa-2006}. Those techniques are however limited to simple mechanical systems with no redundancy, no loop nor kinematic constraints. \framebox{check}
%In {\em topological data analysis}, 
%
A prominent example is homology that can be computed without a precise
reconstruction and under less restrictive
conditions~\cite{geometrica-ccl09,nsw-fhm-2008}. The rapidly growing
theory of {\em persistent homology}~\cite{eh-ph-2008,rg-bptd-2008} was
recently introduced as a powerful tool for the study of the
topological invariants of sampled shapes. The approach consists of
building a simplicial complex whose elements are filtered by some
user-defined function. The filtration is then used to remove
topological noise and to report the stable topological features.
These advances in computational topology attracted interest in the
mathematical community and in several fields like neurosciences,
computer vision or sensor
networks~\cite{fpgo-airc-2009,cids-lbsni-2008,dsrg-csnph-2007}.  What
still prevents these new methods from highly impacting applications is
the lack of efficient data structures and algorithms to construct
simplicial complexes and filtrations in high dimensions.

Another major issue is to design methods that are stable with respect
to {\em noise, sparsity, outliers and other defects} that corrupt data
and move them away from the underlying structure we want to
understand. The geometric approaches have so far only considered very restrictive models of noise~\cite{nsw-fhm-2008}.  Larger families of noise models have recently been considered and statistical approaches have been proposed to provide shape approximations that are stable with respect to those models~\cite{gpvw-mme-2011}. These methods however do not provide topological guarantees on the approximation and the question of designing computationally tractable estimators converging at an optimal rate remains open. A major challenge is to design unifying frameworks that embrace {\em statistical approaches} and deterministic methods, and offer topological guarantees.  

{\em Shape descriptors} are used in a variety of applications, including shape classification, shape retrieval, shape matching, shape registration, and symmetry detection.  Shape here must be understood in a broad sense and may be a point cloud, a manifold, or more generally, a compact metric set. A fundamental question is {\em how much information} about a shape can be recovered from descriptors. Most existing work in shape analysis only provides lower bounds on a shape distance (e.g. the Gromov-Hausdorff distance) based on descriptor distance. There are very few exceptions to this rule, and the upper bounds on shape distances these provide come with very loose guarantees on the error~\cite{bbk-gmds-06,ms-gh-05}. Deriving tight upper bounds is thus still widely open.  Descriptors based on topological quantities measured from a point cloud~\cite{ccgmo-ghsssp-09, socg-pbsds-10} opens new directions to explore. \framebox{check}



\vspace{2mm}

{\bf Objective 3 :} To study new {\em robust models for geometric and topological inference}.  To cluster data with a geometric prior. To study topological signatures of shapes.

\paragraph{Theory versus practice.} 
Geometric and topological methods are well behind dimensionality reduction techniques in terms of 
software development and applications.  
%
% Breaking the computational bottleneck is now the main issue.  Settling the {\em algorithmic foundations} of geometry understanding in
% higher dimensions is a grand challenge of great theoretical and practical significance.
%
To go beyond these low-dimensional examples, one needs efficient and robust software to construct and manipulate simplicial complexes in dimensions higher than 3.  Only a very few such software exist.  {\em Qhull} (http://www.qhull.org/) can compute convex hulls and Delaunay triangulations in moderate dimensions, but is of little use in the context of geometry understanding since it only constructs full-dimensional triangulations. {\em Multifario} (http://multifario.sourceforge.net/) is a set of subroutines and data structures dedicated to {\em meshing} manifolds that occur in dynamical systems. The software can in principle approximate manifolds or arbitrary codimension but examples are only reported in dimensions at most 3. Interestingly, this algorithm  uses a multiple parameter continuation approach and has some similarity with our algorithm~\cite{boissonnat2010meshing}, which has theoretical guarantees and whose complexity is only linear in the ambient dimension.  {\em Polymake} can handle several types of complexes, build Voronoi diagrams and compute advanced topological characteristics of objects like a finite representation of the fundamental group. It is however more oriented towards an interactive use for mathematical experimentation
rather than  an automated use at large scale.%, fast and robust data processing.
% C'est l'impression que j'en ai apres avoir un peu surfe sur differents
% sites, mais je n'ai pas une confiance absolue dans ce que je dis
% ci-dessus.

Several libraries exist for homology computation. RedHom computes Betti numbers and torsion coefficients of cubical sets, simplicial complexes and general, regular CW complexes (http://redhom.ii.uj.edu.pl/).
Two implementations of {\em persistent homology} algorithms are currently available, PLEX a package for Matlab (http://comptop.stanford.edu/u/programs/jplex/), and  Dionysus (www.mrzv.org/software/dionysus/). They both offer the construction of several types of simplicial complexes. PLEX has been  successfully applied in low dimensions~\cite{fpgo-airc-2009,rg-bptd-2008,mtcw-tco-2010}. Dionysus offers advanced
functionalities but there is no documentation and its diffusion is limited.
%Dionysus constructs $\alpha$-shapes, Cech and Rips complexes. Various options to compromize between memory and efficiency
%Plex : rips, landmark selection , ex k=1, d=25
\vspace{2mm}

{\bf Objective 4 :}   {\em To develop an open source {\em  software platform for geometric understanding in high dimensions}. %that will provide the software environment for experimenting with our new data structures and algorithms and to integrate them in a coherent library of interoperable modules. 
To apply our techniques to a few key problems where they may have a huge impact. To disseminate our software and to benchmark our results on real data from various applied fields.}

\subsection{Research roadmap} 


% This proposal addresses {\em fundamental
%   research} issues, and its results are expected to serve as a basis
% for groundbreaking advances for {\em applications in scientific computing
% and data analysis}. 
% A major outcome of the project will be a
% high-quality open source software {\em platform} of components
% implementing the main results. 
%


The proposal is structured into four areas that focus on objectives 1-4 mentionned above.
% which we now describe in more detail. 


\subsection*{WP 1: Computational geometry in non euclidean spaces} 
Delaunay triangulation and Voronoi diagrams in non-euclidean spaces, e.g. riemannian manifolds,
Bregman and statistical spaces, intrinsic algorithms (discrete metric spaces). Triangulating Riemannian manifolds, stratified shapes, mesh generation, reconstruction.  Parameterization of data/nonlinear shapes. 

In the last decades, a set of new geometric methods, known as manifold learning, have been developed with the intent of parametrizing nonlinear shapes embedded in high-dimensional spaces. Let us mention MDS, LLE, Isomap to name a few. While these methods are able to parametrize nonlinear manifolds, they assume however very restrictive hypotheses on the geometry of the manifolds sampled by the datapoints to ensure correctness.  Moreover, from a computational point of view, a common drawback of these methods is that they usually involve computations of eigenvalues and eigenvectors of matrices of the size of the dataset, preventing to deal with huge datasets without a pre-processing.

Our goal is to design new methods and algorithms for sampling and approximating shapes of any codimension in high dimensional spaces.

Approximation of submanifolds using Delaunay refinement. We intend to extend the Delaunay refinement paradigm to higher-dimensional manifolds. For surfaces of R3, this approach has been proven to have several advantages over grid methods, leading to better quality and complexity of the approximation. We expect this advantage to be even stronger as the dimension increases. Two main issues should be considered. First, since the size of the Delaunay triangulation depends exponentially on the ambient dimension, we intend to compute instead lighter data structures such as the ones proposed in Work Package 2 (witness complex, Rips complex or tangential complex). Second, in order to get guaranteed approximation properties, we will have to propose mechanisms to remove badly shaped simplices (slivers) in higher dimensions. Although techniques have been proposed for sliver removal in higher dimensions, they are either inefficient or do not have theoretical guarantees. We intend to develop new techniques based on optimal sampling of convex function to get rid of slivers. Such an approach has proven to be very successful in practice for surfaces in R3. We intend to give better theoretical foundations to this method and to implement it for higher-dimensional manifolds. An important special case we intend to consider is isosurfacing in arbitrary dimension and codimension [90]. By isosurfacing, one means constructing f−1(c) where f is a discrete function f:Zd→Rk and c a given point in Rk.

================DELAUNAY

===========STABILITY


Delaunay triangulations are one of the most useful constructions in
Computational Geometry that have found applications in many domains of
science. Delaunay triangulations have been extensively studied since
the work of B. Delaunay in 1934 and discovery of new important
properties has never declined. Rather surprisingly though, the
stability of those structures has not been studied in a systematic
way. Related work can be found in the context of kinetic data
structures \cite{Agarwal:2010:KSD:1810959.1810984} or in the context
of robust computation \cite{Salesin:1989:EGB:73833.73857}. Our
motivation comes from recent attemps to extend Delaunay triangulations
beyond Euclidean spaces. A first example is the generation of
anisotropic meshes. In such an application, a metric tensor field is given
that varies over a domain of $\rem$ we want to mesh. Anisotropic
Voronoi diagrams and anisotropic Delaunay triangulations then emerge
as natural structures \cite{labelle2003}. A related (and more general) question
is to define intrinsic Delaunay triangulations on a Riemannian
manifold \cite{leibon2000}. Other types of Delaunay-like structures have been
proposed to approximate submanifolds, most notably the restricted
Delaunay triangulation \cite{edelsbrunner1997rdt}, and the tangential Delaunay complex
\cite{boissonnat2011tancplx}. We might expect that, when the density of points is dense
enough, all these Delaunay-like structures are similar. In fact this
is the type of result that can be found in \cite{labelle2003}
and in \cite{leibon2000}. However, the result of Labelle and
Shewchuk is limited to the 2-dimensional case and the paper of Leibon
and Letscher contains a flaw. These papers in fact miss an important
condition which is not related to the sampling density but to its
genericity. Roughly, the Delaunay triangulation is unstable around
cospherical configurations which ruins any attempt to define Delaunay
triangulations on domains where the metric varies.

The aim of this paper is to introduce a parametrized notion of
genericity for Delaunay triangulations and to state stability results
when we keep away from degeneracies. This paper builds over
preliminary results on anisotropic Delaunay meshes
\cite{Boissonnat:2008:ADL:1456721.1456962} and manifold reconstruction
using the tangential Delaunay complex \cite{boissonnat2011tancplx}. The central idea in
both cases is to define Delaunay triangulations locally and to glue
the local triangulations together by removing inconsistencies among
the local triangulations. This is achieved by first detecting the
so-called cospherical configurations and then by killing them. The
cospherical configurations are themselves fragile and can be killed by
various means, e.g., by weighting the points or by refining the
sample.

The present paper provides a general study of the stability of
Delaunay triangulations, and applies the results to the problem of
meshing compact differentiable submanifolds of Euclidean space. We
introduce the notion of $\delta$-protected Delaunay simplices and of
$\delta$-generic point sets.  This leads to our stability results. The
concept is also related to the well-known notion of thickness (or
fatness) of a simplex: We show that the Delaunay simplices of
$\delta$-generic point sets are thick.  We also show how to produce
such point sets.



==================MESHING

The problem of triangulating manifolds has a long history in the
mathematical literature. In differential topology, seminal
contributions are due to Whitney~\cite{whitney}, Cairns~\cite{cairns},
Munkres~\cite{munkres}, Whitehead~\cite{whitehead} to name a
few. Although these papers are not of an algorithmic nature, they
introduce and study several interesting concepts that have been
extensively used in Computational Geometry recently such as Voronoi
diagrams restricted to a manifold, $\e$-sample of a manifold, fat (or
thick) triangulations. However, these papers do not discuss the
geometric quality of the approximation nor the size of the sample. The
optimal sampling and approximation of convex bodies is also a long
standing problem in convex optimization with major contributions by
Gruber~\cite{gruber1,gruber2} and Dudley~\cite{dudley}. Recently,
Clarkson~\cite{clarkson} extended this line of work to non-convex
smooth manifolds of arbitrary dimensions. 
%In~\cite{clarkson}, tight
%bounds on the Hausdorff error are established. Moreover, as pointed
%out by Clarkson, there exists a general algorithm to construct such
%approximations. 
However, his algorithm follows an intrinsic point of
view which makes it difficult to use in practice since it requires to
compute geodesic distances on the manifold which may be quite
complicated in practice \cite{pc-gcsrp-05}. Other, more practical
algorithms for approximating convex bodies, including the well-known
sandwich algorithm, have been analyzed by
Kamenev~\cite{convex-bodies}. We are not aware of similar studies for
non convex manifolds except for the case of surfaces embedded in $\R
^3$ which has been extensively studied in the Computational Geometry
literature.  See \cite{ECGBook} for a recent survey.  These methods
start by computing some subdivision of the embedding space (such as a
grid or a triangulation of the sample points)  and their
direct extension to higher dimensions would face an exponential
dependence on $d$. A step in this direction is the extension of the
celebrated Marching Cube algorithm to manifolds of higher
dimensions~\cite{marching-cube1,isosurface}.  Continuation methods do
not use any subdivision of the ambient space and are close in spirit
to our approach. They construct a triangulated approximation of a
$k$-dimensional submanifold in a greedy way and extend the current
$k$-dimensional triangulated domain by adding a neighborhood of a
boundary point. Some experimental results can be found in \cite{henderson} but
no theoretical analysis of continuation methods is available.

=================RECONSTRUCTION

Manifold reconstruction consists of computing a piecewise linear
approximation of an unknown manifold $\M \subset \R^d$ from a finite
sample of unorganized points $\pp$ lying on $\M$ or close to
$\M$. When the manifold is a two-dimensional surface embedded in
$\R^3$, the problem is known as the surface reconstruction
problem. Surface reconstruction is a problem of major practical
interest which has been extensively studied in the fields of
Computational Geometry, Computer Graphics and Computer Vision.  In the
last decade, solid foundations have been established and the problem
is now pretty well understood. Refer to Dey's book \cite{bookdey}, and
the survey by Cazals and Giesen in \cite{book1} for recent
results. The output of those methods is a triangulated surface that
approximates $\M$. This triangulated surface is usually extracted from
a 3-dimensional subdivision of the ambient space (typically a grid or
a triangulation). Although rather inoffensive in 3-dimensional space,
such data structures depend exponentially on the dimension of the
ambient space, and all attempts to extend those geometric approaches
to more general manifolds have led to algorithms whose complexities
depend exponentially on
$d$~\cite{manifold3, manifold4,manifold2,homology1}.

The problem in higher dimensions is also of great practical interest
in data analysis and machine learning. In those fields, the general
assumption is that, even if the data are represented as points in a
very high dimensional space $\R^d$, they in fact live on a manifold of
much smaller intrinsic dimension~\cite{seung-lee}. If the manifold is
linear, well-known global techniques like principal component analysis
(PCA) or multi-dimensional scaling (MDS) can be efficiently
applied. When the manifold is highly nonlinear, several more local
techniques have attracted much attention in visual perception and many
other areas of science. Among the prominent algorithms are
Isomap~\cite{isomap}, LLE~\cite{lle}, Laplacian
eigenmaps~\cite{laplacian}, Hessian eigenmaps~\cite{hessian},
diffusion maps~\cite{diffusion,diffusion1}, principal
manifolds~\cite{principal-manifolds}. Most of those methods reduce to
computing an eigendecomposition of some connection matrix. In all
cases, the output is a mapping of the original data points into $\R^k$
where $k$ is the estimated intrinsic dimension of $\M$.  Those methods
come with no or very limited guarantees. For example, Isomap provides
a correct embedding only if $\M$ is isometric to a convex open set of
$\R ^k$ and LLE can only reconstruct topological balls. To be able to
better approximate the sampled manifold, another route is to extend
the work on surface reconstruction and to construct a piecewise linear
approximation of $\M$ from the sample in such a way that, under
appropriate sampling conditions, the quality of the approximation can
be guaranteed. First investigations along this line can be found in
the work of Cheng, Dey and Ramos \cite{manifold2}, and Boissonnat,
Guibas and Oudot \cite{manifold3}. In both cases, however, the
complexity of the algorithms is exponential in the ambient dimension
$d$, which highly reduces their practical relevance.

In this paper, we extend the geometric techniques developed in small
dimensions and propose an algorithm that can reconstruct smooth
manifolds of arbitrary topology while avoiding the computation of data
structures in the ambient space.  We assume that $\M$ is a smooth
manifold of known dimension $k$ and that we can compute the tangent
space to $\M$ at any sample point. Under those conditions, we propose
a provably correct algorithm that %\framebox{\text{\color{red}{Changed}}} %allows to 
construct a simplicial
complex of dimension $k$ that approximates $\M$. The complexity of the
algorithm is linear in $d$, quadratic in the size $n$ of the sample,
and exponential in $k$.  Our work builds on \cite{manifold3} and \cite{manifold2} 
but dramatically reduces the dependence on $d$. To
the best of our knowledge, this is the first certified algorithm for
manifold reconstruction whose complexity depends only linearly on the
ambient dimension. In the same spirit, Chazal and Oudot
\cite{persistence} have devised an algorithm of intrinsic complexity
to solve the easier problem of computing the homology of a manifold
from a sample.

Our approach is based on two main ideas~: the notion of {\it
  tangential Delaunay complex} introduced in %\framebox{\text{\color{red}{Changed: references rearranged}}}
\cite{coordinate-system,thesis1,freeman}, and the technique of sliver
removal by weighting the sample points \cite{sliver1}. The tangential
complex is obtained by gluing local (Delaunay) triangulations around
each sample point. The tangential complex is a subcomplex of the
$d$-dimensional Delaunay triangulation of the sample points but it can
be computed using mostly operations in the $k$-dimensional tangent
spaces at the sample points. Hence the dependence on $k$ rather than
$d$ in the complexity. %\framebox{CHECK}  
However, due to the presence of so-called
inconsistencies, the local triangulations may not form a triangulated
manifold. Although this problem has already been reported \cite{freeman}, no
solution was known except for the case of curves ($k=1$)
\cite{thesis1}.
The idea of removing inconsistencies among local triangulations that
have been computed independently has already been used
for maintaining dynamic meshes \cite{starsplaying} and generating anisotropic
meshes~\cite{anisotropic1}. Our approach is close in spirit to the one
in ~\cite{anisotropic1}.
We show that,  under appropriate sample conditions, we can remove inconsistencies by weighting the
sample points. We can then prove
that the approximation returned by our algorithm is ambient isotopic to $\M$,
and a close geometric approximation of $\M$.

Our algorithm can be seen as a {\em local} version of the cocone
algorithm of Cheng et al. \cite{manifold2}. By local, we mean that we
do not compute any $d$-dimensional data structure like a grid or a
triangulation of the ambient space. Still, the tangential complex is a
subcomplex of the weighted $d$-dimensional Delaunay triangulation of
the (weighted) data points and therefore implicitly relies on a global
partition of the ambient space. This is a key to our analysis and 
% \framebox{\text{\color{red}{Changed}}} %makes
distinguishes our method %depart 
from other local algorithms that have been proposed
in the surface reconstruction literature \cite{prisme-4564a,gopi}.

\subsection*{WP 2:  Dimension-sensitive algorithms and data structures} 

Central to the techniques we intend to develop is the construction of simplicial complexes.  A graph is an example of a 1-dimensional simplicial complex but simplicial complexes are much more powerful than graphs and allow to approximate complicated shapes of arbitrary dimension and topology. They offer a flexible data structure to represent and process higher-dimensional shapes and recent developments have shown that simplicial complexes computed on top of point clouds are primary tools to capture the topology of the underlying space of the data. 

In 3-dimensions, 2 and 3-dimensional simplicial complexes of surfaces and volumes are widely used in graphics, scientific computing and manufacturing. Because of its numerous interesting properties and of the existence of extremely efficient algorithms to compute it, the Delaunay triangulation has become one of the most famous and widely used geometric data structures that spread out accross all sciences. The algorithms we have implemented in CGAL are among the most reliable and fast algorithms. They have been included in the heart of MATLAB. 

The algorithms used in 3 dimensions extend to any dimension but their complexity grows exponentially with the dimension which makes them useless for real applications beyong say dimension 6~\cite{avis,hornus}.  In order for algorithms and implementations to scale with the dimension, we need to exploit (hidden) structure of the data and to design dimension-sensitive algorithms and data structures.

% our focus is not on worst-case complexity, but on (provably) good performance under some given structural properties of the input. These properties may be of statistical nature (when we are dealing with noise, for example), or of geometric nature (when data is of low intrinsic dimension, say). Related to this, we are also aiming at output-sensitive algorithms.

\paragraph{Design of small yet faithfull simplicial complexes.} 
Given a set of points $V$ in $\R ^d$, a number of simplicial complexes
with vertex set $V$ have been proposed. A first class of simplicial
complexes uses a parameter $\alpha$ which can be used to order the
simplices of the complex.

The \u{C}ech complex is the nerve of the set $B_{\alpha}$ of balls of
radius $\alpha$ centered at the points of $V$. The nerve of
$B_{\alpha}$  associates a
$i$-simplex to any subset of $i+1$ balls that have a common
intersection. This is a simplicial
complex that is in general not embeddable in $\R ^d$. Moreover, it is usually very big and
difficult to compute since it requires to detect whether a subset of
balls of  $\R ^d$ intersect. 

A simpler to compute simplicial complex is the Rips complex whose
edges are the same as for the \u{C}ech complex. The higher dimensional
simplices of the Rips complex are obtained by computing the cliques of
sizes 3, 4 etc. in the graph of the edges. This simplicial complex is
much easier to compute than the \u{C}ech complex and it has the
vremarkable property that it can be constructed in a purely
combinatorial way from 
its 1-skeleton.  Such a simplicial complex is called a {\em flag
  complex}. Nevertheless, the Rips complex is not embedded in $\R ^d$
and may have a dimension much higher than the dimension of the underlying structure
of the data.


Various simplicial complexes have been derived from the Delaunay
triangulation of the vertices. The $\alpha$-complex is the nerve of
the restriction of the Delaunay triangulation to the union of the
balls of $B_{\alpha}$. This complex is embedded in $\R^d$ (provided
that the vertices are in general position) but very difficult to
compute in high dimension for the same reason as the \u{C}ech complex.

Other simplicial complexes derived from the Delaunay triangulation do
not involve any parameter, most notably the restricted Delaunay
triangulation, the tangential Delaunay complex and the witness
complex. Those complexes are
especially designed for the case where $V$ samples a topological space
of small dimension $k$, the central hypothesis in Machine
Learning. Both the restricted and the tangential Delaunay complexes are embedded in $\R^d$, have dimension $k$
(under a mild general position assumption). Still, these simplicial
complexes are limited to small $k$.  The witness complex is
another variant of the Delaunay triangulation introduced by
Vin de Silva and Carlsson. The witness complex is embedded in $\R ^d$
and is remarkably easy to compute in any dimension
since the only numerical operations involves in its construction are
comparisons of distances.

It should be noted that the Rips complex and the witness complex can
both be computed from the knowledge of the distances between the
vertices. Hence these complexes can be computed in any discrete metric
space.

 Currently no code allows to manipulate simplicial complexes of arbitrary dimension in a routine way as is possible for 2 and 3-dimensional triangulations of $\R ^3$ \cite{springerflo,DBLP:journals/tog/PaoluzziBCF93,svy-crm-99}. 

We identify four research directions~:
\begin{enumerate}
\item Classifying simplicial complexes
\item Combinatorial and algorithmic complexity 
\item Compact representation
\item New types of simplicial complexes
\end{enumerate}

\paragraph{Classifying simplicial complexes.}
Some equivalences between the various types of simplicial complexes are known. For example,
the Rips and the \u{C}ech complexes are identical for the $L_1$ norm and for the Euclidean norm, we have 
\[ \rips () \subset \cech () \subset \rips () .\]
Recently, we have established conditions under which the witness complex, the restricted Delaunay triangulation and the tangential complex are identical~\cite{}. A more complete classification is required to better understand these structures and their properties. 
It would also lead to  better algorithms.



\paragraph{Combinatorial and algorithmic complexity.}
A main limitation of using simplicial complexes is their combinatorial and algorithmic complexity.  Differently from polytopes, very few results are known. The flag random complex is a noticeable exception~\cite{}. Other types of random abstract complexes have to be studied from a combinatorial point of view. Geometric simplicial complexes should also be considered.  An especially important question is to obtain complexity bounds for simplicial complexes of well sampled substructures (e.g. submanifolds).  We intend to measure the effect of perturbations (either noise or computed perturbations) on the mathematical properties and combinatorial complexity of those structures, and to develop probabilistic analyses. In addition to their combinatorial complexity, the complexity of the construction of the simplicial complexes is to be analyzed.  Parallel and out-of-core algorithms will be also developed.


\paragraph{Compact representation of simplicial complexes.} We are aware of only a few works on the design of data structures for general simplicial complexes. Brisson~\cite{Brisson:1989:RGS:73833.73858} and Lienhardt~\cite{DBLP:journals/ijcga/Lienhardt94} have introduced data structures to represent $d$-dimensional cell complexes, most notably subdivided manifolds. While those data structures have nice algebraic properties, they are very redundant and do not scale to large data sets or high dimensions. More recently, Attali et al.~\cite{Attali2011} have proposed an efficient data structure to represent and simplify flag complexes, a special family of simplicial complexes including the Rips complex. 
Recently, we have experimented with a tree representation for general simplicial
complexes that seems to perform very well. Simplicial complexes of 500 millions of simplices have been constructed and stored on a laptop~\cite{}. 
Theoretical guarantees and experiments on a large scale are mandatory. In
addition, more compact storage could be further obtained by using
well-known succinct representations of trees~\cite{10.1109/SFCS.1989.63533,Munro:2002:SRB:586840.586885,Ferragina:2005:SLT:1097112.1097456,DBLP:conf/icalp/2003}. The problem of finding minimal representations of simplicial complexes is widely open beyond the planar case.


\paragraph{New types of simplicial complexes.}
Among the simplicial complexes discussed above, only the Rips and the witness complexes can be constructed on a discrete metric space where only the distances between points are known. The Delaunay-like simplicial complexes are based on a distance function, usually the Euclidean distance.  Other distance functions and, in particular, geodesic distances would lead to Intrinsic simplicial complexes. First encouraging results in this direction can be found in \cite{}.


\paragraph{Validation.} \framebox{in WP 4?}
We intend to {\em implement} those structures, experiment with them and see how they behave under realistic conditions. We will use known datasets such as the UCI repository \cite{} and also propose new benchmarks that we will make publicly available with the hope that they will be used to compare data structures and algorithms.

\paragraph{Dimension-sensitive algorithms and data structures} \framebox{WP1?}









\subsection*{WP 3 :  Robust geometric models}
Geometric inference. Feature extraction. Persistent homology.  Stability with respect to perturbation of the data. Topology preserving approximation/simplification. Clustering. Applications in Data Analysis, Computer Vision, Numerical Simulation, Robotics, Molecular Biology.


\subsection*{WP4 : A software platform for geometric understanding in high
  dimension}

We intend to develop an open source  software platform that will provide a
comprehensive and robust set of tools for geometric understanding in
high
dimension. 

\paragraph{The need for a software platform.}
On one hand, we perceive the development of such a platform as 
an absolute necessity to serve as a test bench and  evaluation process
for any new algorithmic solution resulting  from  our theoretical work.
On the other hand, we perceive the development of such a platform as
a full research work. Indeed we are convinced that, as robustness
issues triggered the development of a whole branch of theoretical
work, known as Geometric Computing, the need for
highly efficient implementations to turn around the curse of  
dimensionality,  will participate in the emergence of relevant
concepts for new algorithmic foundations
of  geometric understanding in high dimension.


The term platform means that we intend to capitalize upon development
efforts and encourage contributions from researchers external to the
project.  At the end of this project, the platform will offer a
comprehensive and robust set of algorithmic tools for geometry
processing and analysis in higher dimensional spaces.

We also think about this platform as the choice
vector for the diffusion of our algorithmic solutions 
into application domains, such as astrophysics
or structural biology) 
 where the need for handling high dimensional
data crucially arises.  Such a diffusion in various fields
 will in turn provide for various bench set data 
and user feedback.



\paragraph{State of the art.}  
Leaving aside the flourishing field of machine learning
and well-known successful software for linear algebra,
linear and quadratic optimization,  only  few implementations handle
geometry in high dimension. 
Most of those software use multiscale  grids based
data structures that somehow adapt
 to the local  density of data.
A typical example is the popular ANN software to compute  the approximate
nearest-neighbor of a query point
among a  high dimensional point cloud. 
Qhull is a software that can compute convex hulls and Delaunay
triangulations in dimensions larger than 3, but it doesn't see much
development anymore, and as the authors prominently announce on the
webpage: ``Qhull does not support triangulation of non-convex surfaces,
mesh generation of non-convex objects, medium-sized inputs in 9-D and
higher, alpha shapes''. The polymake framework has many features,
it can handle several types of complexes, build Voronoi diagrams and
compute advanced topological characteristics of objects like a finite
representation of the fundamental group. However, it is strongly
oriented towards an interactive use for mathematical experimentation on
a given object and not automated, fast and robust data processing.
% C'est l'impression que j'en ai apres avoir un peu surfe sur differents
% sites, mais je n'ai pas une confiance absolue dans ce que je dis
% ci-dessus.
Only two implementations of persistent homology algorithms are
currently available. One of them is the PLEX package for Matlab,
developed by the Computational Topology group at Stanford University.
The other one is the
library Dionysus proposed by Dmitriy Morozov. These implementations
%do not propose parallel nor out-of-core versions. They 
are known to be
successful in small dimensions but inefficient as soon as the
dimension rises.  Their  maintenance, only assumed by the very few authors
is likely not to be perennial.

%\framebox{Continuation methods : Multifario http://www.research.ibm.com/people/h/henderson/Continuation/ContinuationMethods.html}


\paragraph{Methodology.} 
Two key requirements in the development of this platform will be
efficiency and theoretical guarantees, including handling of
robustness issues. A common
platform is ideal for these goals, as it allows for more
resources to be poured into the optimization, bug-fixing and interface
design than for a single prototype.

 We intend to apply to the development of this software platform 
the very  recipes that made up the success story
of the CGAL library. 
The development work will be based on a strong infrastructure
including a web site, a svn repository and  daily running of testsuites on several  hardware architectures.
Above all we will set up  an editorial board assuming a  serious review of the 
specifications of packages proposed for inclusion in the platform.
We forecast that the platform will reach a sufficient critical mass
to be of interest for a large community of researchers in
computational geometry and neighboring applicative fields,
which in turn will ensure the long range perennial maintenance
of the software.



\paragraph{Planned developments} 
\begin{itemize}
\item The software platform will provide tools to extract from clouds of points the
simplicial complexes that are relevant for geometric understanding,
Rips, tangential Delaunay  and witness complexes to begin with. 
\item The platform will provide efficient data structures to handle those
complexes. We will in particular  focus on parsimonious data
structures, aiming for a partially implicit representation of those simplicial
complexes, thus avoiding the full space cost entailed by the curse of
dimensionality. 
\item The platform is meant to offer state of the art algorithmic tools for geometric
understanding,
including in particular algorithms to
\begin{itemize}
\item  mesh or reconstruct manifolds
\item  compute the persistent homology of a simplicial complex filtration 
\item cluster data
\item compute signatures of shapes
\item visualization tools \framebox{?}
\end{itemize}
\item The platform will aim at providing parallel implementations as well
as out-of-core versions whenever possible to make possible the
handling of huge data sets in high dimensions.
\end{itemize}

\framebox{Datasets}

\framebox{Diffusion}

%  It will namely include tools to extract from
% point clouds various simplicial complexes and filtrations,
%  like Rips, Cech or witness
% complexes, that are relavant for data analysis and topological feature
% extraction. It will provide data structures to handle those simplicial
% complexes.
% We will focus on parsimonious data structures that will contribute
% to turn around the curse of dimensionality. Particularly promising
% are data structure involving a  partial implicit representation of those simplicial
% complexes. The platform will also offer 
% robust and efficient implementations for persistent homology algorithms
% and other state of the art algorithms arising from
% work in this ERC and from neighboring groups. 

% \thispagestyle{empty}
% \pagestyle{myheadings}
% \markboth{\tcg{Titre ou acronyme de l'ADT}}{\tcg{Titre ou acronyme de l'ADT}}

% %\renewcommand{\thefootnote}{\fnsymbol{footnote}}
% %\setcounter{footnote}{1}
% %\renewcommand{\thefootnote}{\arabic{footnote}}
% %\setcounter{footnote}{0}

% \begin{center}
% {\LARGE\bf
% \tcg{Titre de l'ADT\footnote{
% Tout ce qui est en \tcg{vert} explique ce qui est attendu et est \`a remplacer
% par le texte appropri\'e \`a votre ADT. {\bf Merci d'enlever tout le \tcg{vert} au
% moment de la soumission.}
% }
% }}\\[1ex]
% \Large\bf
% Campagne ADT 2012\\
% \end{center}

% \renewcommand{\thefootnote}{\arabic{footnote}}
% %\setcounter{footnote}{0}

















\section{Resources}

\paragraph{Research environment.}
The PI and his research team, Geometrica, are part of INRIA, the french national institute for research in mathematics and informatics. Part of the group, including the PI, is located in the INRIA research center in Sophia Antipolis  (500 employees and 38 research groups) and part of the group is hosted by the INRIA research center in Saclay in Paris's area (26 research teams). Two members of the research center in Sophia Antipolis are members of the French Academy of Science, two have received an ERC advanced grant and two have received an ERC junior grant. Because of its partial location in Saclay, the group benefits from tight collaborations with the Ecole Normale Sup\'erieure and the Ecole Polytechnique. In particular, 4 members of the group teach at these prestigious institutions. Geometrica currently includes 10 permanent researchers,  2 postdoctoral researchers, 10 Ph.D. students, and 1 research engineer. 

\paragraph{The team members.}
The team members who are directly involved in this proposal are the PI (J-D. Boissonnat) and 2 permanent researchers of the Geometrica team : Fr\'ed\'eric Chazal and Mariette Yvinec.  J-D. Boissonnat will conduct and supervise the research activities of Gudhi and will be involved in the project for at least 70\% of his time.  Fr\'ed\'eric Chazal and Mariette Yvinec will each devote 20\% of their time to this project to co-supervise with the PI the research and implementation work of the students, postdocs and engineers to be engaged in this project. Fr\'ed\'eric Chazal, located in Saclay,  is a world expert in geometric inference and computational topology. Mariette Yvinec, located in Sophia Antipolis,  is a member of the CGAL Editorial Board. She  will be bring her unique expertise in geometric computing. Other members of Geometrica, not financially supported by this project, will also contribute to the ideas and expertise of Gudhi.

\paragraph{External team members.} We will establish a strong collaboration with F. Cazals, leader of the ABS research group on computational structural biology at INRIA Sophia Antipolis, to work on applications of geometry understanding in molecular biology (see Focus Area 4).

\paragraph{Available resources.} Our European ICT Fet-Open project Computational Geometric Learning (CG-Learning) will still be active until november 2013 for a maximum remaining amount of ??? Euros and will ensure a smooth start of Gudhi.  The ANR Pr\'esage (??? Keuros) will provide additional resources for the Geometrica activities in probabilistic techniques in geometry.

The Geometrica team is equipped with numerous PCs and has access to a large PC cluster owned by INRIA Sophia Antipolis.

\paragraph{Requested resources: personnal costs.}\mbox{}\\
-- 70\% of PI's salary over 5 years with a 70\% commitment of his time\\
-- 20\% of 2 PI's close collaborators over 5 years with a 20\% commitment of their time\\
-- 2 full-time post-doctoral researchers during the 5 years of the project\\
-- 2 full-time research engineers during 2.5 years covering the 5 years of the project\\
-- 1 full-time Ph.D. student during years 1, 2, 3\\
-- 1 full-time Ph.D. student during years 2, 3, 4\\
-- 1 full-time Ph.D. student during years 3, 4, 5\\
-- 4 men-months of invited professors in years 1-5.

\paragraph{Requested resources: other direct costs.}\mbox{}\\
-- travel \\
-- hardware : a multicore computer 5 KEuros

The rest of the costs consists of eligible indirect costs, at the rate of 20\% of the direct costs. \framebox{?}
The grand total amounts to ???? Euros over a period of 5 years, as detailed in the table below.

\framebox{include Table}

The funded 4 PhD students will have their research devoted to the fundamental aspects of the 3 first Focus Areas A1-A3 of this proposal, and one on the applications, co-advised with F. Cazals  (Focus Area 4). There will be a lot of synergy between their works, in particular in relation with the development of the platform. The funded research engineers  will help stabilize the software modules, as well as for the construction of new datasets to be made available to the scientific community.

We expect several researchers among our current partners (in particular in the CG-Learning project) to visit us each year and participate to Gudhi. We will also welcome talents from new groups who could bring a complementary expertise to the success of Gudhi. These visits will be funded in part by the ”invited professors” budget above, and in part by INRIA and other resources.
We will also organize a workshop early after the beginning of Gudhi to make it known to the international community, and to help attracting talented scientists for the success of Gudhi.

\newpage

{\small%footnotesize
%\bibliographystyle{alpha}
\bibliographystyle{apalike}
\bibliography{erc}
}

