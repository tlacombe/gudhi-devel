\section{Research proposal (15p.)}

\subsection{State-of-the-art and objectives}

Geometric modeling has undergone huge progress during the past decades. The pressing needs
of multimedia, video games, numerical simulations, manufacturing, computer-aided medicine, culturage heritage and other applications asked for techniques to respresent, process and analyze
{\em 3D} shapes.  Progress has been contributed by  varied disciplines, most notably computer graphics, geometry processing, computer-aided design, computational geometry and topology, scientific computing. This considerable research effort resulted in settling theoretical and algorithmic foundations of 3D geometric modeling, and providing efficient algorithms and codes for applications such as surface reconstruction, mesh generation and point cloud processing~\cite{geometrica-bcmrv-ms-06,dey-csr-2007}.  
The situation is very different and much less has been done for higher dimensional shapes.%   are much more difficult to handle than surfaces in 3-space for two main reasons~: the curse of dimensionality and the unafordable presence of noise.


\paragraph{Dimension reduction.} A first route towards geometry processing in higher dimensional spaces is to try to reduce the dimension. This is 
one of the most popular approaches to high-dimensional data analysis. {\em Dimensionality reduction} consists in mapping the data points down to a linear subspace, whose dimension supposedly coincides with the intrinsic dimension of the data. This approach is elegant in that it helps detect the intrinsic parameters of the data, and by doing so it also reduces the complexity of the problem. Dimensionality reduction techniques fall into two classes: linear methods, e.g. principal component analysis (PCA) or multi-dimensional scaling (MDS), and non-linear methods, e.g. isomap or locally-linear embedding (LLE). The second class of algorithms is more powerful in that it computes more general (in fact, non-linear) mappings. On the whole, dimensionality reduction works well on data sets sampled from manifolds with low curvature and trivial topology. Although the condition on the curvature is mainly a sampling issue, the condition on the topology is mandatory for the mapping onto a linear subspace to make sense. \framebox{Johnson-Lindenstrauss, Wakin}

\paragraph{Simplicial complex approximations.}
In many problems \framebox{examples?}, a linear subspace or a union of linear subspaces is a too crude model and we need to represent nonlinear geometries like a $k$-dimensional submanifold of $\R^d$.  To deal with such complicated geometries, a route is to approximate shapes by simplicial complexes.  Simplicial complexes have been introduced by Poincar\'e in the early days of algebraic topology. Their importance in low dimensions cannot be overerestimated~: 1-dimensional simplicial complexes, better known as graphs or networks, can be found everywhere, 2-dimensional simplicial complexes are the standard representations for surfaces in computer graphics and 3-dimensional simplicial complexes are the representation of choice for scientific computing and numerical simulation when complicated domains are involved. Simplicial complexes can be defined in any dimension and can be used to construct a piecewise linear approximation of a shape with the right topological type (see WP2) or to 
compute the homology of the shape (see WP 3). The main bottleneck in using simplicial complexes in higher dimensions is of a computational nature~:  in high dimensions, simplicial complexes can be huge and very difficult to compute, which limits their current use to low dimensions.

\paragraph{Computational issues.}
The high computational cost of simplicial complexes can be balanced in two ways. First, simplicial complexes are flexible enough to approximate subspaces with a complexity that scales with the complexity of the subspace, not of the ambient space. Since the subspaces of interest 
 in learning theory, data analysis or dynamical systems have usually a small intrinsic dimension,
simplicial complexes may be an option.

The second reason is that coarse topological features such as the homology groups or the Betti numbers (that are invariant under homotopy equivalence) can be estimated from simplicial complexes of a much smaller size than what is needed to reconstruct a homeomorphic approximation. Hence, simplicial complexes may be used to recover important information even if an accurate representation of the data is out of reach.

Simplicial complexes thus have the potential of being the data structure of choice for understanding geometry in high dimensional spaces. The issue is however to define simplicial complexes that are dimension-sensitive and easy to compute.  Various types of simplicial complexes have been proposed such as the \v{C}ech and the Rips complexes, and the more recent Delaunay-like complexes such as the $\alpha$-complex~\cite{eks-sspp-83,he-ubds-95}, the witness complex~\cite{deSilva2008,cds-tewc-2004} and the Delaunay tangential complex~\cite{geometrica-7142i}. They differ by their combinatorial and algorithmic complexities, and their power to approximate a shape.  The algorithmic theory of simplicial complexes is nevertheless in its infancy and much less understood than the computational aspects of triangulations in small dimensions. In particular, little is known about efficient and compact representations of simplicial complexes and experiments have only been reported for simplicial complexes of low-dimensions up to now ~\cite{rg-bptd-2008}. \framebox{true?} 




%  Under appropriate sampling conditions, we have shown that one can reconstruct a provably correct approximation of a smooth $k$-dimensional manifold $M$  embedded in $\R ^d$ in a time that depends only linearly on $d$~\cite{geometrica-7142i}. Researchers have also turned their focus to the somewhat easier problem of inferring topological invariants of the shape without explicitly reconstructing it with the hope that more lightweight data structures and weaker sampling conditions would be appropriate for this simpler task~\cite{nsw-fhm-2008,co-tpr-2008}.







% Many of the results of Computational Geometry have been extended to arbitrary dimension. In particular, worst-case optimal algorithms are known for computing convex hulls, Voronoi diagrams and Delaunay triangulations in any dimension. Hence, in principle, the methods developed for 3D applications should be extendable to higher dimensions. However, the size of these structures depends exponentially on the dimension of the embedding space, which makes them only useful in moderate dimensions~\cite{geometrica-6743i}. Many ideas have been suggested to bypass this curse of dimensionality. A first approach looks for more realistic combinatorial analyses such as smoothed analysis that bounds the expected complexity under some small random perturbation of the data. This led to the celebrated analysis of Linear Programming of Spielmann and Teng \cite{st-saa-2004}. Another route is to trade exact for approximate algorithms \cite{hp-gaa-2011}.  An important example is the search for approximate nearest neighbours. Another example is the theory of core sets that was shown to provide good approximate solution to some optimization problems like computing the smallest enclosing ball, or computing an optimal separating hyperplane (SVM). These tools are both extremely useful but limited to basic operations and have not been yet applied  in the context of geometric modeling. 
% A third approach assumes that the intrinsic dimension  of the object of interest has a much lower intrinsic dimension than the dimension of the embedding space. This is the usual assumption in Manifold Learning. It is then possible to resort to techniques derived from the 3D case and to approximate complex shapes by simplicial complexes (the analogue of triangulations in higher-dimesnional spaces).  Various types of simplicial complexes have been proposed such as the Czech and the RIPS complexes,  and the more recent Delaunay-like complexes such as the $\alpha$-complex~\cite{eks-sspp-83,he-ubds-95}, the witness complex~\cite{deSilva2008,cds-tewc-2004} and the Delaunay tangential complex~\cite{geometrica-7142i}. They differ by their combinatorial and algorithmic complexities, and their power to approximate a shape.
% Under appropriate sampling conditions, we have shown that one can reconstruct a provably correct approximation of a smooth $k$-dimensional manifold $M$  embedded in $\R ^d$ in a time that depends only linearly on $d$~\cite{geometrica-7142i}. Researchers have also turned their focus to the somewhat easier problem of inferring topological invariants of the shape without explicitly reconstructing it with the hope that more lightweight data structures and weaker sampling conditions would be appropriate for this simpler task~\cite{nsw-fhm-2008,co-tpr-2008}.


% A related concept is that of doubling dimension.  Intrinsic algorithms. Discrete metric spaces. Parametrized complexity.



% =====

% In [??], the family of complexes is derived from the so-called alpha-offsets of the input point cloud, defined as the unions of balls of same radius alpha centered at the data points. In [??], the family is derived from the so-called witness complex, first introduced by de Silva [??], which can be viewed as a weak version of the Delaunay triangulation whose construction requires much less computation power. Despite this nice feature, multiscale reconstruction, at least in its current form, still has a complexity that scales up exponentially with the dimension of the ambient space. Hence, it can only be applied to low-dimensional data sets in practice.

\paragraph{Noisy data.} \framebox{to be revised}
When dealing with approximation and samples, one needs stability results to ensure that the quantities that are computed, geometric or topological invariants, are good approximations of the real ones. {\em Topological persistence} was recently introduced as a powerful tool for the study of the topological invariants of sampled spaces~\cite{eh-ph-2008,rg-bptd-2008}. Given a point cloud in Euclidean space, the approach consists in building a simplicial complex whose elements are filtered by some user-defined function. This filter basically gives an order of insertion of the simplices in the complex. The persistence algorithm, first introduced by Edelsbrunner, Letscher and Zomorodian \cite{elz-tps-2002}, is able to track down the topological invariants of the filtered complex as the latter is being built. As proved by Cohen-Steiner et al. \cite{geometrica-cseh-07}, under reasonable conditions on the input point cloud, and modulo a right choice of filter, the most persistent invariants in the filtration correspond to invariants of the space underlying the data. Thus, the information extracted by the persistence algorithm is global, as opposed to the locality relationships used by the dimensionality reduction techniques. In this respect, topological persistence appears as a complementary tool to dimensionality reduction. In particular, it enables to determine whether the input data is sampled from a manifold with trivial topology, a mandatory condition for dimensionality reduction to work properly. Note however that it does not tell how and where to cut the data to remove unwanted topological features.

Multiscale reconstruction is a novel approach~\cite{geometrica-bgo-09}. Taking advantage of the ideas of persistence, the approach consists in building a one-parameter family of simplicial complexes approximating the input at various scales. Differently from above, the family may not necessarily form a filtration, but it has other nice properties. In particular, for a sufficiently dense input data set, the family contains a long sequence of complexes that approximate the underlying space provably well, both in a topological and in a geometric sense. In fact, there can be several such sequences, each one corresponding to a plausible reconstruction at a certain scale. Thus, determining the topology and shape of the original object reduces to finding the stable sequences in the one-parameter family of complexes.   However, multiscale reconstruction, at least in its current form, still has a complexity that scales up exponentially with the dimension of the ambient space. Hence, it can only be applied to low-dimensional data sets in practice.


\paragraph{The research and development pipeline in computational geometry.}
Computational geometry emerged as a discipline in the seventies and has met with considerable success in providing foundations to solve basic geometric problems including data structures, convex hulls, triangulations, Voronoi diagrams, geometric arrangements and geometric optimisation~\cite{by-ag-98}. This initial development of the discipline has been followed, in the mid-nineties, by a vigorous effort to make computational geometry more effective by
combining in a closer way basic theoretical research and the development of robust software.
% is the study of effective methods (algorithms and data structures) of geometric computing, namely methods that are not only theoretically proved but also work well in practice.  
Several EC projects (CGAL, GALIA) established an outstanding research momentum and gave a leading role to Europe in this context.  They led to successful techniques and tools, most notably the CGAL library~\cite{cgal}  that provides a well-organised, robust and efficient software environment for developing geometric applications. CGAL is considered as one of the main achievements of the field and is by now the standard in geometric computing, with a large diffusion worldwide and varied applications in both academia and industry. CGAL is a unique library with no equivalent counterpart in the world.  My research group at INRIA took a leading role in the development of CGAL since its start more than 10 years ago. More recently, we seconded our basic research on  3D shape processing by practical developments in the form of fast, safe and quality-guaranteed software components for mesh generation and shape reconstruction that are now part of the open source library CGAL~\cite{cgal:rty-m3-11}. Those components are used worldwide in academia and in industry for various applications in geometric modeling, medical imaging and geology.


\paragraph{Overall objective.}
Time has come...

% Processing and analyzing complex 3D shapes is a fundamental problem with a long history of scientific successes which is on the agenda of several communities like scientific computing, computer graphics, geometry processing and computer-aided design.  Emblematic problems are mesh generation that aims at sampling and meshing a given domain, and surface reconstruction that constructs an approximation of a surface which is only known through a set of points. During the last decade, the computational geometry community established solid theoretical foundations to these problems leading to recent breakthroughs in {\em mesh generation} \cite{geometrica-bcmrv-ms-06} and {\em surface reconstruction} \cite{dey-csr-2007}.  The Geometrica group took a leading role in this research and contributed major theoretical advances as well as practical developments in the form of fast, safe and quality-guaranteed software components for mesh generation and shape reconstruction that are now part of the open source library CGAL~\cite{cgal:rty-m3-11}. Those components are now used worldwide in academia and in industry for various applications in Geometric Modeling, Medical Imaging and Geology.




\subsection{Methodology}
To reach our overall goal of settling firm algorithmic foundations for geometry understanding in higher dimensions, we will follow the principles that have been guiding my group for more than 10 years and will simultaneously develop
{\em mathematical approaches} providing guarantees even in the presence of noise or outliers,
{\em effective algorithms} that are amenable to theoretical analysis and fully validated experimentally,
and {\em perenial software} development.

The proposal is structured into the following four workpackages:
{\bf WP 1}:  {\em Dimension-sensitive data  structures} will extend current knowledge about simplicial complexes, and  provide efficient data structures and basic algorithms for their representation, construction and manipulation. 
  {\bf WP 2}:  {\em Triangulating non Euclidean geometric spaces} will develop effective algorithms to mesh or reconstruct manifolds and other spaces equiped with various metrics.   {\bf WP 3}: {\em Robust models for geometric inference, comparison and  clustering} will provide the crucial  algorithms for topological data analysis.
 {\bf WP 4}:  {\em  Software platform for geometric understanding in high dimensions} will provide the software environment for experimenting with our new data structures and algorithms, for integrating them in a coherent library of interoperable modules, and for diffusing our results to applied fields. We now describe in more detail each of these workpackages.


\subsection*{WP 1: Computational geometry in non euclidean spaces} 
Delaunay triangulation and Voronoi diagrams in non-euclidean spaces, e.g. riemannian manifolds,
Bregman and statistical spaces, intrinsic algorithms (discrete metric spaces). Triangulating Riemannian manifolds, stratified shapes, mesh generation, reconstruction.  Parameterization of data/nonlinear shapes. 

In the last decades, a set of new geometric methods, known as manifold learning, have been developed with the intent of parametrizing nonlinear shapes embedded in high-dimensional spaces. Let us mention MDS, LLE, Isomap to name a few. While these methods are able to parametrize nonlinear manifolds, they assume however very restrictive hypotheses on the geometry of the manifolds sampled by the datapoints to ensure correctness.  Moreover, from a computational point of view, a common drawback of these methods is that they usually involve computations of eigenvalues and eigenvectors of matrices of the size of the dataset, preventing to deal with huge datasets without a pre-processing.

Our goal is to design new methods and algorithms for sampling and approximating shapes of any codimension in high dimensional spaces.

Approximation of submanifolds using Delaunay refinement. We intend to extend the Delaunay refinement paradigm to higher-dimensional manifolds. For surfaces of R3, this approach has been proven to have several advantages over grid methods, leading to better quality and complexity of the approximation. We expect this advantage to be even stronger as the dimension increases. Two main issues should be considered. First, since the size of the Delaunay triangulation depends exponentially on the ambient dimension, we intend to compute instead lighter data structures such as the ones proposed in Work Package 2 (witness complex, Rips complex or tangential complex). Second, in order to get guaranteed approximation properties, we will have to propose mechanisms to remove badly shaped simplices (slivers) in higher dimensions. Although techniques have been proposed for sliver removal in higher dimensions, they are either inefficient or do not have theoretical guarantees. We intend to develop new techniques based on optimal sampling of convex function to get rid of slivers. Such an approach has proven to be very successful in practice for surfaces in R3. We intend to give better theoretical foundations to this method and to implement it for higher-dimensional manifolds. An important special case we intend to consider is isosurfacing in arbitrary dimension and codimension [90]. By isosurfacing, one means constructing f−1(c) where f is a discrete function f:Zd→Rk and c a given point in Rk.

================DELAUNAY

===========STABILITY


Delaunay triangulations are one of the most useful constructions in
Computational Geometry that have found applications in many domains of
science. Delaunay triangulations have been extensively studied since
the work of B. Delaunay in 1934 and discovery of new important
properties has never declined. Rather surprisingly though, the
stability of those structures has not been studied in a systematic
way. Related work can be found in the context of kinetic data
structures \cite{Agarwal:2010:KSD:1810959.1810984} or in the context
of robust computation \cite{Salesin:1989:EGB:73833.73857}. Our
motivation comes from recent attemps to extend Delaunay triangulations
beyond Euclidean spaces. A first example is the generation of
anisotropic meshes. In such an application, a metric tensor field is given
that varies over a domain of $\rem$ we want to mesh. Anisotropic
Voronoi diagrams and anisotropic Delaunay triangulations then emerge
as natural structures \cite{labelle2003}. A related (and more general) question
is to define intrinsic Delaunay triangulations on a Riemannian
manifold \cite{leibon2000}. Other types of Delaunay-like structures have been
proposed to approximate submanifolds, most notably the restricted
Delaunay triangulation \cite{edelsbrunner1997rdt}, and the tangential Delaunay complex
\cite{boissonnat2011tancplx}. We might expect that, when the density of points is dense
enough, all these Delaunay-like structures are similar. In fact this
is the type of result that can be found in \cite{labelle2003}
and in \cite{leibon2000}. However, the result of Labelle and
Shewchuk is limited to the 2-dimensional case and the paper of Leibon
and Letscher contains a flaw. These papers in fact miss an important
condition which is not related to the sampling density but to its
genericity. Roughly, the Delaunay triangulation is unstable around
cospherical configurations which ruins any attempt to define Delaunay
triangulations on domains where the metric varies.

The aim of this paper is to introduce a parametrized notion of
genericity for Delaunay triangulations and to state stability results
when we keep away from degeneracies. This paper builds over
preliminary results on anisotropic Delaunay meshes
\cite{Boissonnat:2008:ADL:1456721.1456962} and manifold reconstruction
using the tangential Delaunay complex \cite{boissonnat2011tancplx}. The central idea in
both cases is to define Delaunay triangulations locally and to glue
the local triangulations together by removing inconsistencies among
the local triangulations. This is achieved by first detecting the
so-called cospherical configurations and then by killing them. The
cospherical configurations are themselves fragile and can be killed by
various means, e.g., by weighting the points or by refining the
sample.

The present paper provides a general study of the stability of
Delaunay triangulations, and applies the results to the problem of
meshing compact differentiable submanifolds of Euclidean space. We
introduce the notion of $\delta$-protected Delaunay simplices and of
$\delta$-generic point sets.  This leads to our stability results. The
concept is also related to the well-known notion of thickness (or
fatness) of a simplex: We show that the Delaunay simplices of
$\delta$-generic point sets are thick.  We also show how to produce
such point sets.



==================MESHING

The problem of triangulating manifolds has a long history in the
mathematical literature. In differential topology, seminal
contributions are due to Whitney~\cite{whitney}, Cairns~\cite{cairns},
Munkres~\cite{munkres}, Whitehead~\cite{whitehead} to name a
few. Although these papers are not of an algorithmic nature, they
introduce and study several interesting concepts that have been
extensively used in Computational Geometry recently such as Voronoi
diagrams restricted to a manifold, $\e$-sample of a manifold, fat (or
thick) triangulations. However, these papers do not discuss the
geometric quality of the approximation nor the size of the sample. The
optimal sampling and approximation of convex bodies is also a long
standing problem in convex optimization with major contributions by
Gruber~\cite{gruber1,gruber2} and Dudley~\cite{dudley}. Recently,
Clarkson~\cite{clarkson} extended this line of work to non-convex
smooth manifolds of arbitrary dimensions. 
%In~\cite{clarkson}, tight
%bounds on the Hausdorff error are established. Moreover, as pointed
%out by Clarkson, there exists a general algorithm to construct such
%approximations. 
However, his algorithm follows an intrinsic point of
view which makes it difficult to use in practice since it requires to
compute geodesic distances on the manifold which may be quite
complicated in practice \cite{pc-gcsrp-05}. Other, more practical
algorithms for approximating convex bodies, including the well-known
sandwich algorithm, have been analyzed by
Kamenev~\cite{convex-bodies}. We are not aware of similar studies for
non convex manifolds except for the case of surfaces embedded in $\R
^3$ which has been extensively studied in the Computational Geometry
literature.  See \cite{ECGBook} for a recent survey.  These methods
start by computing some subdivision of the embedding space (such as a
grid or a triangulation of the sample points)  and their
direct extension to higher dimensions would face an exponential
dependence on $d$. A step in this direction is the extension of the
celebrated Marching Cube algorithm to manifolds of higher
dimensions~\cite{marching-cube1,isosurface}.  Continuation methods do
not use any subdivision of the ambient space and are close in spirit
to our approach. They construct a triangulated approximation of a
$k$-dimensional submanifold in a greedy way and extend the current
$k$-dimensional triangulated domain by adding a neighborhood of a
boundary point. Some experimental results can be found in \cite{henderson} but
no theoretical analysis of continuation methods is available.

=================RECONSTRUCTION

Manifold reconstruction consists of computing a piecewise linear
approximation of an unknown manifold $\M \subset \R^d$ from a finite
sample of unorganized points $\pp$ lying on $\M$ or close to
$\M$. When the manifold is a two-dimensional surface embedded in
$\R^3$, the problem is known as the surface reconstruction
problem. Surface reconstruction is a problem of major practical
interest which has been extensively studied in the fields of
Computational Geometry, Computer Graphics and Computer Vision.  In the
last decade, solid foundations have been established and the problem
is now pretty well understood. Refer to Dey's book \cite{bookdey}, and
the survey by Cazals and Giesen in \cite{book1} for recent
results. The output of those methods is a triangulated surface that
approximates $\M$. This triangulated surface is usually extracted from
a 3-dimensional subdivision of the ambient space (typically a grid or
a triangulation). Although rather inoffensive in 3-dimensional space,
such data structures depend exponentially on the dimension of the
ambient space, and all attempts to extend those geometric approaches
to more general manifolds have led to algorithms whose complexities
depend exponentially on
$d$~\cite{manifold3, manifold4,manifold2,homology1}.

The problem in higher dimensions is also of great practical interest
in data analysis and machine learning. In those fields, the general
assumption is that, even if the data are represented as points in a
very high dimensional space $\R^d$, they in fact live on a manifold of
much smaller intrinsic dimension~\cite{seung-lee}. If the manifold is
linear, well-known global techniques like principal component analysis
(PCA) or multi-dimensional scaling (MDS) can be efficiently
applied. When the manifold is highly nonlinear, several more local
techniques have attracted much attention in visual perception and many
other areas of science. Among the prominent algorithms are
Isomap~\cite{isomap}, LLE~\cite{lle}, Laplacian
eigenmaps~\cite{laplacian}, Hessian eigenmaps~\cite{hessian},
diffusion maps~\cite{diffusion,diffusion1}, principal
manifolds~\cite{principal-manifolds}. Most of those methods reduce to
computing an eigendecomposition of some connection matrix. In all
cases, the output is a mapping of the original data points into $\R^k$
where $k$ is the estimated intrinsic dimension of $\M$.  Those methods
come with no or very limited guarantees. For example, Isomap provides
a correct embedding only if $\M$ is isometric to a convex open set of
$\R ^k$ and LLE can only reconstruct topological balls. To be able to
better approximate the sampled manifold, another route is to extend
the work on surface reconstruction and to construct a piecewise linear
approximation of $\M$ from the sample in such a way that, under
appropriate sampling conditions, the quality of the approximation can
be guaranteed. First investigations along this line can be found in
the work of Cheng, Dey and Ramos \cite{manifold2}, and Boissonnat,
Guibas and Oudot \cite{manifold3}. In both cases, however, the
complexity of the algorithms is exponential in the ambient dimension
$d$, which highly reduces their practical relevance.

In this paper, we extend the geometric techniques developed in small
dimensions and propose an algorithm that can reconstruct smooth
manifolds of arbitrary topology while avoiding the computation of data
structures in the ambient space.  We assume that $\M$ is a smooth
manifold of known dimension $k$ and that we can compute the tangent
space to $\M$ at any sample point. Under those conditions, we propose
a provably correct algorithm that %\framebox{\text{\color{red}{Changed}}} %allows to 
construct a simplicial
complex of dimension $k$ that approximates $\M$. The complexity of the
algorithm is linear in $d$, quadratic in the size $n$ of the sample,
and exponential in $k$.  Our work builds on \cite{manifold3} and \cite{manifold2} 
but dramatically reduces the dependence on $d$. To
the best of our knowledge, this is the first certified algorithm for
manifold reconstruction whose complexity depends only linearly on the
ambient dimension. In the same spirit, Chazal and Oudot
\cite{persistence} have devised an algorithm of intrinsic complexity
to solve the easier problem of computing the homology of a manifold
from a sample.

Our approach is based on two main ideas~: the notion of {\it
  tangential Delaunay complex} introduced in %\framebox{\text{\color{red}{Changed: references rearranged}}}
\cite{coordinate-system,thesis1,freeman}, and the technique of sliver
removal by weighting the sample points \cite{sliver1}. The tangential
complex is obtained by gluing local (Delaunay) triangulations around
each sample point. The tangential complex is a subcomplex of the
$d$-dimensional Delaunay triangulation of the sample points but it can
be computed using mostly operations in the $k$-dimensional tangent
spaces at the sample points. Hence the dependence on $k$ rather than
$d$ in the complexity. %\framebox{CHECK}  
However, due to the presence of so-called
inconsistencies, the local triangulations may not form a triangulated
manifold. Although this problem has already been reported \cite{freeman}, no
solution was known except for the case of curves ($k=1$)
\cite{thesis1}.
The idea of removing inconsistencies among local triangulations that
have been computed independently has already been used
for maintaining dynamic meshes \cite{starsplaying} and generating anisotropic
meshes~\cite{anisotropic1}. Our approach is close in spirit to the one
in ~\cite{anisotropic1}.
We show that,  under appropriate sample conditions, we can remove inconsistencies by weighting the
sample points. We can then prove
that the approximation returned by our algorithm is ambient isotopic to $\M$,
and a close geometric approximation of $\M$.

Our algorithm can be seen as a {\em local} version of the cocone
algorithm of Cheng et al. \cite{manifold2}. By local, we mean that we
do not compute any $d$-dimensional data structure like a grid or a
triangulation of the ambient space. Still, the tangential complex is a
subcomplex of the weighted $d$-dimensional Delaunay triangulation of
the (weighted) data points and therefore implicitly relies on a global
partition of the ambient space. This is a key to our analysis and 
% \framebox{\text{\color{red}{Changed}}} %makes
distinguishes our method %depart 
from other local algorithms that have been proposed
in the surface reconstruction literature \cite{prisme-4564a,gopi}.

\subsection*{WP 2:  Dimension-sensitive algorithms and data structures} 

Central to the techniques we intend to develop is the construction of simplicial complexes.  A graph is an example of a 1-dimensional simplicial complex but simplicial complexes are much more powerful than graphs and allow to approximate complicated shapes of arbitrary dimension and topology. They offer a flexible data structure to represent and process higher-dimensional shapes and recent developments have shown that simplicial complexes computed on top of point clouds are primary tools to capture the topology of the underlying space of the data. 

In 3-dimensions, 2 and 3-dimensional simplicial complexes of surfaces and volumes are widely used in graphics, scientific computing and manufacturing. Because of its numerous interesting properties and of the existence of extremely efficient algorithms to compute it, the Delaunay triangulation has become one of the most famous and widely used geometric data structures that spread out accross all sciences. The algorithms we have implemented in CGAL are among the most reliable and fast algorithms. They have been included in the heart of MATLAB. 

The algorithms used in 3 dimensions extend to any dimension but their complexity grows exponentially with the dimension which makes them useless for real applications beyong say dimension 6~\cite{avis,hornus}.  In order for algorithms and implementations to scale with the dimension, we need to exploit (hidden) structure of the data and to design dimension-sensitive algorithms and data structures.

% our focus is not on worst-case complexity, but on (provably) good performance under some given structural properties of the input. These properties may be of statistical nature (when we are dealing with noise, for example), or of geometric nature (when data is of low intrinsic dimension, say). Related to this, we are also aiming at output-sensitive algorithms.

\paragraph{Design of small yet faithfull simplicial complexes.} 
Given a set of points $V$ in $\R ^d$, a number of simplicial complexes
with vertex set $V$ have been proposed. A first class of simplicial
complexes uses a parameter $\alpha$ which can be used to order the
simplices of the complex.

The \u{C}ech complex is the nerve of the set $B_{\alpha}$ of balls of
radius $\alpha$ centered at the points of $V$. The nerve of
$B_{\alpha}$  associates a
$i$-simplex to any subset of $i+1$ balls that have a common
intersection. This is a simplicial
complex that is in general not embeddable in $\R ^d$. Moreover, it is usually very big and
difficult to compute since it requires to detect whether a subset of
balls of  $\R ^d$ intersect. 

A simpler to compute simplicial complex is the Rips complex whose
edges are the same as for the \u{C}ech complex. The higher dimensional
simplices of the Rips complex are obtained by computing the cliques of
sizes 3, 4 etc. in the graph of the edges. This simplicial complex is
much easier to compute than the \u{C}ech complex and it has the
vremarkable property that it can be constructed in a purely
combinatorial way from 
its 1-skeleton.  Such a simplicial complex is called a {\em flag
  complex}. Nevertheless, the Rips complex is not embedded in $\R ^d$
and may have a dimension much higher than the dimension of the underlying structure
of the data.


Various simplicial complexes have been derived from the Delaunay
triangulation of the vertices. The $\alpha$-complex is the nerve of
the restriction of the Delaunay triangulation to the union of the
balls of $B_{\alpha}$. This complex is embedded in $\R^d$ (provided
that the vertices are in general position) but very difficult to
compute in high dimension for the same reason as the \u{C}ech complex.

Other simplicial complexes derived from the Delaunay triangulation do
not involve any parameter, most notably the restricted Delaunay
triangulation, the tangential Delaunay complex and the witness
complex. Those complexes are
especially designed for the case where $V$ samples a topological space
of small dimension $k$, the central hypothesis in Machine
Learning. Both the restricted and the tangential Delaunay complexes are embedded in $\R^d$, have dimension $k$
(under a mild general position assumption). Still, these simplicial
complexes are limited to small $k$.  The witness complex is
another variant of the Delaunay triangulation introduced by
Vin de Silva and Carlsson. The witness complex is embedded in $\R ^d$
and is remarkably easy to compute in any dimension
since the only numerical operations involves in its construction are
comparisons of distances.

It should be noted that the Rips complex and the witness complex can
both be computed from the knowledge of the distances between the
vertices. Hence these complexes can be computed in any discrete metric
space.

 Currently no code allows to manipulate simplicial complexes of arbitrary dimension in a routine way as is possible for 2 and 3-dimensional triangulations of $\R ^3$ \cite{springerflo,DBLP:journals/tog/PaoluzziBCF93,svy-crm-99}. 

We identify four research directions~:
\begin{enumerate}
\item Classifying simplicial complexes
\item Combinatorial and algorithmic complexity 
\item Compact representation
\item New types of simplicial complexes
\end{enumerate}

\paragraph{Classifying simplicial complexes.}
Some equivalences between the various types of simplicial complexes are known. For example,
the Rips and the \u{C}ech complexes are identical for the $L_1$ norm and for the Euclidean norm, we have 
\[ \rips () \subset \cech () \subset \rips () .\]
Recently, we have established conditions under which the witness complex, the restricted Delaunay triangulation and the tangential complex are identical~\cite{}. A more complete classification is required to better understand these structures and their properties. 
It would also lead to  better algorithms.



\paragraph{Combinatorial and algorithmic complexity.}
A main limitation of using simplicial complexes is their combinatorial and algorithmic complexity.  Differently from polytopes, very few results are known. The flag random complex is a noticeable exception~\cite{}. Other types of random abstract complexes have to be studied from a combinatorial point of view. Geometric simplicial complexes should also be considered.  An especially important question is to obtain complexity bounds for simplicial complexes of well sampled substructures (e.g. submanifolds).  We intend to measure the effect of perturbations (either noise or computed perturbations) on the mathematical properties and combinatorial complexity of those structures, and to develop probabilistic analyses. In addition to their combinatorial complexity, the complexity of the construction of the simplicial complexes is to be analyzed.  Parallel and out-of-core algorithms will be also developed.


\paragraph{Compact representation of simplicial complexes.} We are aware of only a few works on the design of data structures for general simplicial complexes. Brisson~\cite{Brisson:1989:RGS:73833.73858} and Lienhardt~\cite{DBLP:journals/ijcga/Lienhardt94} have introduced data structures to represent $d$-dimensional cell complexes, most notably subdivided manifolds. While those data structures have nice algebraic properties, they are very redundant and do not scale to large data sets or high dimensions. More recently, Attali et al.~\cite{Attali2011} have proposed an efficient data structure to represent and simplify flag complexes, a special family of simplicial complexes including the Rips complex. 
Recently, we have experimented with a tree representation for general simplicial
complexes that seems to perform very well. Simplicial complexes of 500 millions of simplices have been constructed and stored on a laptop~\cite{}. 
Theoretical guarantees and experiments on a large scale are mandatory. In
addition, more compact storage could be further obtained by using
well-known succinct representations of trees~\cite{10.1109/SFCS.1989.63533,Munro:2002:SRB:586840.586885,Ferragina:2005:SLT:1097112.1097456,DBLP:conf/icalp/2003}. The problem of finding minimal representations of simplicial complexes is widely open beyond the planar case.


\paragraph{New types of simplicial complexes.}
Among the simplicial complexes discussed above, only the Rips and the witness complexes can be constructed on a discrete metric space where only the distances between points are known. The Delaunay-like simplicial complexes are based on a distance function, usually the Euclidean distance.  Other distance functions and, in particular, geodesic distances would lead to Intrinsic simplicial complexes. First encouraging results in this direction can be found in \cite{}.


\paragraph{Validation.} \framebox{in WP 4?}
We intend to {\em implement} those structures, experiment with them and see how they behave under realistic conditions. We will use known datasets such as the UCI repository \cite{} and also propose new benchmarks that we will make publicly available with the hope that they will be used to compare data structures and algorithms.

\paragraph{Dimension-sensitive algorithms and data structures} \framebox{WP1?}









\subsection*{WP 3 :  Robust geometric models}
Geometric inference. Feature extraction. Persistent homology.  Stability with respect to perturbation of the data. Topology preserving approximation/simplification. Clustering. Applications in Data Analysis, Computer Vision, Numerical Simulation, Robotics, Molecular Biology.


\subsection*{WP4 : A software platform for geometric understanding in high
  dimension}

We intend to develop an open source  software platform that will provide a
comprehensive and robust set of tools for geometric understanding in
high
dimension. 

\paragraph{The need for a software platform.}
On one hand, we perceive the development of such a platform as 
an absolute necessity to serve as a test bench and  evaluation process
for any new algorithmic solution resulting  from  our theoretical work.
On the other hand, we perceive the development of such a platform as
a full research work. Indeed we are convinced that, as robustness
issues triggered the development of a whole branch of theoretical
work, known as Geometric Computing, the need for
highly efficient implementations to turn around the curse of  
dimensionality,  will participate in the emergence of relevant
concepts for new algorithmic foundations
of  geometric understanding in high dimension.


The term platform means that we intend to capitalize upon development
efforts and encourage contributions from researchers external to the
project.  At the end of this project, the platform will offer a
comprehensive and robust set of algorithmic tools for geometry
processing and analysis in higher dimensional spaces.

We also think about this platform as the choice
vector for the diffusion of our algorithmic solutions 
into application domains, such as astrophysics
or structural biology) 
 where the need for handling high dimensional
data crucially arises.  Such a diffusion in various fields
 will in turn provide for various bench set data 
and user feedback.



\paragraph{State of the art.}  
Leaving aside the flourishing field of machine learning
and well-known successful software for linear algebra,
linear and quadratic optimization,  only  few implementations handle
geometry in high dimension. 
Most of those software use multiscale  grids based
data structures that somehow adapt
 to the local  density of data.
A typical example is the popular ANN software to compute  the approximate
nearest-neighbor of a query point
among a  high dimensional point cloud. 
Qhull is a software that can compute convex hulls and Delaunay
triangulations in dimensions larger than 3, but it doesn't see much
development anymore, and as the authors prominently announce on the
webpage: ``Qhull does not support triangulation of non-convex surfaces,
mesh generation of non-convex objects, medium-sized inputs in 9-D and
higher, alpha shapes''. The polymake framework has many features,
it can handle several types of complexes, build Voronoi diagrams and
compute advanced topological characteristics of objects like a finite
representation of the fundamental group. However, it is strongly
oriented towards an interactive use for mathematical experimentation on
a given object and not automated, fast and robust data processing.
% C'est l'impression que j'en ai apres avoir un peu surfe sur differents
% sites, mais je n'ai pas une confiance absolue dans ce que je dis
% ci-dessus.
Only two implementations of persistent homology algorithms are
currently available. One of them is the PLEX package for Matlab,
developed by the Computational Topology group at Stanford University.
The other one is the
library Dionysus proposed by Dmitriy Morozov. These implementations
%do not propose parallel nor out-of-core versions. They 
are known to be
successful in small dimensions but inefficient as soon as the
dimension rises.  Their  maintenance, only assumed by the very few authors
is likely not to be perennial.

%\framebox{Continuation methods : Multifario http://www.research.ibm.com/people/h/henderson/Continuation/ContinuationMethods.html}


\paragraph{Methodology.} 
Two key requirements in the development of this platform will be
efficiency and theoretical guarantees, including handling of
robustness issues. A common
platform is ideal for these goals, as it allows for more
resources to be poured into the optimization, bug-fixing and interface
design than for a single prototype.

 We intend to apply to the development of this software platform 
the very  recipes that made up the success story
of the CGAL library. 
The development work will be based on a strong infrastructure
including a web site, a svn repository and  daily running of testsuites on several  hardware architectures.
Above all we will set up  an editorial board assuming a  serious review of the 
specifications of packages proposed for inclusion in the platform.
We forecast that the platform will reach a sufficient critical mass
to be of interest for a large community of researchers in
computational geometry and neighboring applicative fields,
which in turn will ensure the long range perennial maintenance
of the software.



\paragraph{Planned developments} 
\begin{itemize}
\item The software platform will provide tools to extract from clouds of points the
simplicial complexes that are relevant for geometric understanding,
Rips, tangential Delaunay  and witness complexes to begin with. 
\item The platform will provide efficient data structures to handle those
complexes. We will in particular  focus on parsimonious data
structures, aiming for a partially implicit representation of those simplicial
complexes, thus avoiding the full space cost entailed by the curse of
dimensionality. 
\item The platform is meant to offer state of the art algorithmic tools for geometric
understanding,
including in particular algorithms to
\begin{itemize}
\item  mesh or reconstruct manifolds
\item  compute the persistent homology of a simplicial complex filtration 
\item cluster data
\item compute signatures of shapes
\item visualization tools \framebox{?}
\end{itemize}
\item The platform will aim at providing parallel implementations as well
as out-of-core versions whenever possible to make possible the
handling of huge data sets in high dimensions.
\end{itemize}

\framebox{Datasets}

\framebox{Diffusion}

%  It will namely include tools to extract from
% point clouds various simplicial complexes and filtrations,
%  like Rips, Cech or witness
% complexes, that are relavant for data analysis and topological feature
% extraction. It will provide data structures to handle those simplicial
% complexes.
% We will focus on parsimonious data structures that will contribute
% to turn around the curse of dimensionality. Particularly promising
% are data structure involving a  partial implicit representation of those simplicial
% complexes. The platform will also offer 
% robust and efficient implementations for persistent homology algorithms
% and other state of the art algorithms arising from
% work in this ERC and from neighboring groups. 

% \thispagestyle{empty}
% \pagestyle{myheadings}
% \markboth{\tcg{Titre ou acronyme de l'ADT}}{\tcg{Titre ou acronyme de l'ADT}}

% %\renewcommand{\thefootnote}{\fnsymbol{footnote}}
% %\setcounter{footnote}{1}
% %\renewcommand{\thefootnote}{\arabic{footnote}}
% %\setcounter{footnote}{0}

% \begin{center}
% {\LARGE\bf
% \tcg{Titre de l'ADT\footnote{
% Tout ce qui est en \tcg{vert} explique ce qui est attendu et est \`a remplacer
% par le texte appropri\'e \`a votre ADT. {\bf Merci d'enlever tout le \tcg{vert} au
% moment de la soumission.}
% }
% }}\\[1ex]
% \Large\bf
% Campagne ADT 2012\\
% \end{center}

% \renewcommand{\thefootnote}{\arabic{footnote}}
% %\setcounter{footnote}{0}
















\bibliographystyle{abbrv}
\bibliography{erc}

\section{Ressources}

I will devote 70\% of my time to this project, and I will dedicate all my expertise and efforts to conduct and supervise the research work. To this end, I will receive the precious help of 2 permanent researchers of the Geometrica team : Fr\'ed\'eric Chazal who is a  leading researcher in geometric inference and computational topology and Mariette Yvinec who is an expert in geometric computing and a member of the CGAL Editorial Board. They will devote 20\% of their time to this project to co-supervise with me the research and implementation work of the students, postdocs and engineers to be engaged in this project. Other members of Geometrica,
not financially supported by this project, will also collaborate to the project.

