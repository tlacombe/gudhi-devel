\begin{center}
\fbox{\parbox{12cm}{\noindent {\bf Abstract: } We propose to develop a
formal model for Web data management. This model will open new
horizons for the development of the Web in a well-principled way,
enhancing its functionality, performance, and
reliability. Specifically, the goal is to develop a universally
accepted formal framework for describing complex and flexible
interacting Web applications featuring notably data exchange, sharing,
integration, querying and updating. We also propose to develop formal
foundations that will enable peers to concurrently reason about global
data management activities, cooperate in solving specific tasks and
support services with desired quality of service.  Although the
proposal addresses fundamental issues, its goal is to serve as the
basis for ground-breaking future software development for Web data
management.  }}\end{center}

% UPDATE A1

% Fondements de la gestion de donn�es du Web

% L'objectif du projet Webdam est de d�velopper un cadre formel
% universellement accept� pour d�crire des applications Web mettant en
% jeu des interactions complexes et flexibles, avec notamment de
% l'�change, du partage, de l'int�gration, de l'interrogation et des
% mises � jour de donn�es. Nous proposons �galement de d�velopper les
% fondements formels qui permettront � des syst�mes autonomes de
% raisonner ensemble sur une gestion globale de donn�es, de coop�rer
% pour r�soudre des t�ches sp�cifiques et procurer des services avec les
% qualit�s de service souhait�es. Bien que ce projet aborde des
% questions fondamentales, son objectif est aussi de servir de base pour
% des d�veloppemeents futurs de logiciels pour la gestion des donn�es du
% Web, faciliter le d�veloppement du Web, am�liorer ses fonctionnalit�s,
% ses performances et sa fiabilit�.