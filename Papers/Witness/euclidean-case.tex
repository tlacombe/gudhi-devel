\documentclass[11pt,a4paper]{article}

\usepackage[latin1]{inputenc} 
\usepackage[french]{babel}



\usepackage{latexsym}
\usepackage{amsmath}
\usepackage{amssymb}
\usepackage{color}
%\usepackage{here}
\usepackage{graphicx}
\usepackage{paralist}
\usepackage{mathptmx}

\newtheorem{theorem}{Theorem}
\newtheorem{definition}{Definition}
\newtheorem{lemma}{Lemma}

\newtheorem{corol}{Corollary}
\newenvironment{proof}
        {\noindent \textbf{Proof} \hspace{0.3mm}}
        {\hspace{0.3mm}$\square$  \smallskip}      
     

\newcommand{\uu}{\mathbf u}
\newcommand{\nn}{\mathbf n}

\newcommand{\hint}{\it{Hint}}


\newcommand{\MEB}{\mbox{MEB}}
\newcommand{\TT}{\mathcal T}
\newcommand{\SSS}{\mathcal S}

\newcommand{\argmax}{{\mbox{arg max}}}
\newcommand{\norm}[1]{\| #1 \|}
\newcommand{\aff}{{\mbox{aff}}}


%%%%%%JD%%%%%%%%%%%%%%%
\newcommand{\B}{\mathbb B}            
\newcommand{\E}{\mathbb E}            
\newcommand{\e}{\varepsilon}            
\newcommand{\F}{\mathbb F}            
\newcommand{\I}{\mathbb I}
\newcommand{\N}{\mathbb N}
\newcommand{\Q}{\mathbb Q}
\renewcommand{\S}{\mathbb S}
\newcommand{\Z}{\mathbb Z}
\newcommand{\R}{\mathbb R}
\newcommand{\bd}{{\partial}}
\newcommand{\WW}{{{\cal W}}}
%\newcommand{\aff}{{\rm {aff}}}

\newcommand{\conv}{{\rm {conv}}}
\newcommand{\diam}{{\rm diam}}
\newcommand{\link}{{\rm link}}
\newcommand{\str}{{\rm star}}
%\newcommand{\hint}{\it{Hint}}
\newcommand{\UB}{{U(B)}}
\newcommand{\vol}{{\rm vol}}
\newcommand{\vor}{{\rm Vor}}
\newcommand{\reg}{{\rm Reg}}
\newcommand{\ceil}[1]{\left\lceil #1 \right\rceil}
\newcommand{\floor}[1]{\left\lfloor #1 \right\rfloor}
%%%%%%%%%%%%%%%%%%%%%%%
\newcommand{\del}{{\rm Del}} 
\newcommand{\rch}{{\rm rch}}
\newcommand{\wit}{{\rm Wit}}
\newcommand{\twit}{{\rm Kwit}}

\parindent 0pt
\parskip 3mm

%\setlength{\textwidth}{18cm}
%\setlength{\evensidemargin}{-1cm}
%\setlength{\oddsidemargin}{-1cm}

%\input{/Users/Science/Papers/macros-1}

\begin{document}
\title{Computing Delaunay triangulations with low algebraic complexity}
\maketitle

The goal of this note is to show that the Delaunay complex of a finite
set of points $L\subset\R^d$ can be computed using only (squared)
distance comparisons, i.e. using predicates of degree 2. This is to be
compared to the usual approach that needs to evaluate predicates of
algebraic degree $d+2$.

Let $W$ and $L$ denote sets of points in $\R ^d$. $L$ is finite but
$W$ may be not. We will only assume that $W$ is closed. The points of $W$ are called the witnesses and the
points of $L$ are called the landmarks. To make things simple, we
compactify $\R^d$.

 
\section{Background}

\subsection{Witness complex}


\begin{definition}[Witness of a simplex]
  Let $\sigma$ be a (abstract)  simplex with vertices in $L$, and let $w$ be a point
  of $W$. We say that $w$ is a witness of $\sigma$ if 
$$\| w-p\|
  \leq \| w -q \| \;\;\;\;\; \forall p\in \sigma \;\; {\rm  and}\;\;
  \forall q\in L\setminus
  \sigma.$$
\end{definition}


\begin{definition}[Witness complex]
  The witness complex $\wit (L,W)$ is the complex consisting of all
  simplexes $\sigma$ such that for any simplex $\tau\subseteq
  \sigma$, $\tau$ has a witness in $W$.
\end{definition}

The only predicates involved in the construction of $\wit (L,W)$ are
(squared) distance comparisons, i.e. polynomials of degree 2.

\subsection{Delaunay complex}

\begin{definition}[Delaunay center]
A Delaunay center for a simplex $\sigma\subset L$ is a point $x$ that
satisfies
\[ \| p-x\| \leq \| q-x\|\;\;\;\;\;\forall p\in\sigma \;\; {\rm and} \;\;
q\in L.\]
\end{definition}
Note that $x$ is at equal distance from all the vertices of $\sigma$.
If the dimension of $\sigma$ is $d$, the Delaunay center is the
circumcenter of $\sigma$ which we denote $c_{\sigma}$.

\begin{definition}[Delaunay complex]
The Delaunay complex $\del (L)$ of $L$ is the complex consisting of
all simplexes $\sigma$ that have  a Delaunay center.
\end{definition}

Differently from the witness complex, computing $\del(L)$ involves
evaluating the so-called {\tt in\_sphere} predicate that determines if
a point lies inside, on, or outside the sphere circumscribing a
$d$-simplex.
This predicate is a multivariate polynomial of degree $d+2$ in the
coordinates of the vertices.

\subsection{Protection}

\begin{definition}[$\delta$-protection]
We say that a simplex $\sigma \subset L$ is $\delta$-protected  at a 
a point $x\in \R^d$  if
\[ \| x- q\| >  \| x-p\|  +\delta \;\;\;\;\; \forall p \in \sigma\;\;
{\rm  and} \;\;  \forall  q\in L\setminus \sigma.\]
\end{definition}

%Note that when the dimension of $\sigma$ is $d$, $c=c_{\sigma}$ is the center of the unique circumsphere of $\sigma$. We write $r_{\sigma}$ for its radius.

%We also define a stronger notion of $\delta$-protection.

\begin{definition}[Strong $\delta$-protection]
We say that a simplex $\sigma \subset L$ is strongly $\delta$-protected if 
it is $\delta$-protected at its circumcenter $c_{\sigma}$.%  is  such that 
% \[ \| c_{\sigma}- q\| >  \| c_{\sigma}-p\|  +\delta \;\;\;\;\; \forall
% p \in \sigma \;\; {\rm  and}\;\; \forall  q\in L\setminus \sigma.\]
\end{definition}

We say that $L$ is $\delta$-generic when all the Delaunay
$d$-simplices of $\del (L)$ are strongly $\delta$-protected.

\begin{lemma}[Protection]
\label{lemma-protection}
If $L$ is a $\delta$-generic $\lambda$-net, all its Delaunay simplices of all
positive dimensions are strongly $\frac{\delta}{4d}$-protected.
\end{lemma}

\begin{proof}
See \cite{BDGO}. If $L$ is a $\lambda$-net and $\sigma\in \del (L)$ is
$\delta$-protected, then it is $\lambda\delta$-power-protected (Lemma
1.1). By Lemma 1.3, if the $d$-simplices are $\eta$-power-protected,
then any simplex of any dimension $\leq d$ is
$\eta/d$-protected. Combining the two claims, we get that any
simplex is $(\lambda\delta/d)$-power-protected. By Lemma 1.2, any
simplex is then $\delta/(4d)$-protected.
\end{proof}

% \paragraph{Remark.} Write $\delta = \nu\lambda$. Then the bound in the
% Protection lemma becomes $\delta'= \nu^2 \lambda /2d$. The extra
% factor of $\nu$ (compared to the analogous bound for the top
% dimensional simplices) is unsatisfactory.   This
% can be avoided by using weighted distances and weighted
% protection. The bound in the lemma is derived from the bound for the
% weighted protection. Can we obtain a better bound for the euclidean
% protection?

\subsection{Approximate circumcenters} % of Delaunay centers}

Given a simplex $\sigma$, $\Delta_{\sigma}$ denotes its diameter,
$c_{\sigma} \in \aff (\sigma)$ its circumcenter, $r_{\sigma}$ its
circumradius, and $\Theta_{\sigma}$ its thickness.

\begin{definition}[$\alpha$-center]
Let $\sigma$ be a $d$-simplex. We
call $\alpha$-center of $\sigma$ any point $w\in W$ such that 
$$| \| w -p\| - \| w -q\| |  \leq \alpha \;\;\;\;\; \forall p,q \in \sigma .$$
%\item $w$ witnesses $\sigma$.

\end{definition}

We recall the following lemma from~\cite{boissonnat:hal-00807050}
(Lemma 4.3).

\begin{lemma}[Approximate circumcenters]
\label{lemma-stability}
Let $\sigma\subset L$ be a simplex, and let
$\tilde{c}$ be an $\alpha$-center of $\sigma$. If  for all
$i,j\leq k$,
%\begin{enumerate}
%\item 
$\| \tilde{c}-p_i\| < \lambda$,
%\item $\| p_i-p_j\| \geq \lambda$
%\item $| \| \tilde{c}-p_i\| -\|\tilde{c}-p_j\| | \leq \alpha$
%\end{enumerate}
then there exists  $c_{\sigma}$ such that
\begin{enumerate}
\item $\forall p,q\in \sigma$, $\| c_{\sigma}-p\| = \| c_{\sigma}-q\|
  $ (i.e. $c_{\sigma}$ is a circumcenter of $\sigma$)
\item $\| \tilde{c}-c_{\sigma}\| \leq
\frac{\alpha\,
  \lambda}{\Theta_{\sigma} \, \Delta_{\sigma}}$.
\end{enumerate}
\end{lemma}

% \paragraph{Remark.} According to the proof
% in~\cite{boissonnat:hal-00807050}, when $\sigma$ is a $k$-simplex,
% $c_{\sigma}$ can be the barycenter of the circumcenters of any subset of
% $d-k+1$ $d$-cofaces of $\sigma$.

\begin{corol}[Approximate CC of $d$-simplices]
Let $\sigma\subset L$ be a $d$-simplex such that $R_{\sigma}\leq
\lambda \leq  \Delta_{\sigma}$, and let
$\tilde{c}$ be an $\alpha$-center of $\sigma$. %  such that for all
Then there exists a circumcenter  $c_{\sigma}$ of $\sigma$ such
that $\| \tilde{c}_{\sigma}-c_{\sigma}\| 
\leq \frac{\alpha}{\Theta_{\sigma}-\alpha/\lambda} $. 
\label{lemma-stability2}
\end{corol}

\begin{proof}
Let $p$ be a vertex of $\sigma$ furthest from $\tilde{c}$ and write
$\gamma= \| \tilde{c}-p\|$. According to Lemma~\ref{lemma-stability},
there exits  $c_{\sigma}$ that circumscribes $\sigma$ such
that $\| \tilde{c}-c_{\sigma}\| \leq \frac{\alpha\,
  \gamma}{\Theta_{\sigma} \, \Delta_{\sigma}}$.  
Using $\Delta_{\sigma}\geq \lambda$, $\| c_{\sigma} -p\|\leq \lambda$ and
$\gamma \leq \| \tilde{c}-c_{\sigma}\| + \| c_{\sigma} -p\|$, we get
$\| \tilde{c}-c_{\sigma}\| \leq \frac{\alpha}{\Theta_{\sigma}-\alpha/\lambda}$.
\end{proof}



\section{Identity of witness and Delaunay complexes}

We prove in this section that, under some conditions, $\wit
(L,W)=\del(L)$. We start with the easy lemma:


\begin{lemma}
\label{lemma-easy}
If $W'\subseteq W$, then $\wit (L,W') \subseteq \wit (L,W)$.
\end{lemma}

\begin{lemma}
\label{lemma-subset}
$\del (L) \subseteq \wit (L,\R^d)$.
\end{lemma}

\begin{proof}
 Any simplex $\sigma$ of $\del(L)$  has an empty circumscribing ball whose
center $c_{\sigma}$ is a witness of $\sigma$, and  $c_{\sigma}$ is also a
witness for all the faces of $\sigma$. 
\end{proof}


\begin{theorem}[de Silva] 
\label{lemma-silva}
$\wit (L, \R^d) = \del (L)$. 
\end{theorem}

\begin{proof}\cite{aem-ww-2007}
Let $\tau=[p_0,...,p_k]$ be a $k$-simplex of $\wit (L)$  witnessed by a ball
$B_{\tau}$, i.e. $B_{\tau}\cap L = \tau$. Write $S_{\tau}$ for the
sphere bounding $B_{\tau}$.
We prove that {$\tau \in \del (L)$} by a double induction on $k$ and
$l := |S_{\tau} \cap \tau |$. The claim clearly holds for $k = 0$. We
also note that $\tau\in \del (L)$ when $l = k + 1$ since $B_{\tau}$
then circumscribes $\tau$. We can also assume that $|S_{\tau} \cap
\tau | \geq 1$ since we can reduce the radius of $B_{\tau}$ so that it
still witnesses $\tau$ and $S_{\tau}$ contains at least a vertex of
$\tau$. Hence the claim holds for $k=0$ and $l=1$.
Assume that the claim holds for all simplices of $\wit (L, W )$ of dimension up to $k-1$ and for all $l \leq k$, and refer to the figures.

\begin{center}
\includegraphics[width=7cm]{Fig/vin-th.pdf}
%\includegraphics[width=4cm]{/Users/boissonn/Science/Enseign/Exam/2013/Fig/ww-th2.pdf}
\end{center}


We will show that one can find a new witness ball of $\tau$, we still
call $B_{\tau}$, such that the number of points of $S_{\tau}\cap \tau$
gets increased. Write $\sigma = S_{\tau}\cap \tau$ and $w$ for the
center of $B_{\tau}$. By the induction hypothesis, $\sigma$ is a
Delaunay simplex and therefore there exists a Delaunay ball
$B_{\sigma}$ circumscribing $\sigma$. Write $c$ for its
center. Consider the set of balls $F$ centered on the line segment $S
= [wc]$ and circumscribing $\sigma$. Any ball in $F$ is included in
$B_{\tau}\cup B_{\sigma}$ and its bounding sphere circumscribes
$\sigma$. Hence its interior contains no point of $L\setminus
\tau$. Moreover, since the interior of $B_{\sigma}$ is empty but the
interior of $B_{\tau}$ is not, there exists a point on S such the
associated ball of $F$ witnesses $\tau$ and contains $l + 1$ points of
$\tau$ on its boundary. By carrying on the induction, we obtain a
witness ball that contains all the vertices of $\tau$ and thus is a
Delaunay ball of $\tau$.
\end{proof}

We deduce from Lemma~\ref{lemma-easy} and Theorem~\ref{lemma-silva} 
the following corollary

\begin{corol}
 $\wit
(L,W) \subseteq \del (L)$.
\label{lemma-wit-in-del}
\end{corol}


Note that if the points are in general position, $\del(L)$ is embedded in
$\R^d$. Therefore the same is true for  $\wit (L,W )$.





% \begin{lemma}
% Assume that the $W$ is an $\e$-sample.
% Let $\sigma$ be a $\delta$-protected Delaunay $d$-simplex with $\delta < \e$. Any subsimplex
% $\tau\subset \sigma$ has a witness.
% \end{lemma}

% \begin{proof}
% Let $c$ be the circumcenter of $\sigma$. Any point in the ball 
% \end{proof}







\begin{lemma}[Identity of witness and Delaunay complexes]
Assume that $L$ is a $\delta$-generic $\lambda$-net and
that $W$ is an $\e$-sample with $\e\leq \frac{\delta}{8d}$. Then $\wit (L,W)= \del(L)$.
\label{lemma-wit=del}
\end{lemma}

\begin{proof}
  Let $\sigma$ be a $k$-simplex of $\del (L)$. We know from
  Lemma~\ref{lemma-protection} that it is $\delta '$-protected where
  $\delta'= \delta/(4d)$. Hence,
there exists $c\in \R^d$ satisfying
\begin{enumerate}
\item  $\| c-p_i\| = \| c-p_j\| = r\;\;\; \forall p_i,p_j\in \sigma$
\item $\| c-p_l\| > r+\delta' \;\;\; \forall p_l\in L\setminus \sigma $
\end{enumerate}

For any $x\in B(c, \delta'/2 )$, 
any $p_i\in \sigma$  and any $p_l \in L\setminus\sigma$, we have
\[ |x-p_i| \leq |c-p_i| + | c-x| \leq r + \frac{\delta'}{2} \]
and
\[ |x-p_l| \geq |c-p_l| -|x-c| > r+\delta' -\frac{\delta'}{2}= r + \frac{\delta'}{2} \]
Hence, $x$ is a witness of $\sigma$. If $\e \leq
\delta'/2$, there must be a point  $w \in W$ in $B(c, \delta'/2 )$ which
witnesses $\sigma$.
\end{proof}



% To any simplex $\sigma$, we associate its tight witness $w(\sigma)$, i.e. the
% witness that minimizes $| \| w-p_i\| -\|w-p_j\| | $ for all
% $p_i,p_j\in \sigma$ (ties are broken arbitrarily). Since $W$ is an
% $\e$-sample, $w(\sigma)$ satisfies the condition of the previous
% lemma. $w(\sigma)$ will be a witness of $\sigma$ if we can take $c$ so
% as to have protection on $\sigma$.

% The sampling condition in Lemma~\ref{lemma-wit=del} is quite
% restrictive (see the remark above).

\paragraph{Remark.} The previous lemma does not require that the
simplices be thick. Note however that some thickness is ensured by the
$\delta$-genericity assumption \cite{boissonnat:hal-00807050} (see  the remark at the
end of Section~\ref{sec:relaxed}). Note also that there
is no assumption on $\lambda$. 

\section{Relaxed complexes}
\label{sec:relaxed}

Lemma~\ref{lemma-wit=del} identifies witness and Delaunay
complexes. However, we cannot use the lemma to compute $\del (L)$ from
$\wit (L,W)$ since the $\delta$-genericity assumption requires to know
$\del (L)$. To solve this issue,  we introduce relaxed complexes.

\subsection{Relaxed witness complex}

\begin{definition}[$\alpha$-witness of a simplex]
  Let $\sigma$ be a (abstract)  simplex with vertices in $L$, and let $w$ be a point
  of $W$. We say that $w$ is an $\alpha$-witness of $\sigma$ for some
  non negative $\alpha$ if $$\| w-p\|
  \leq \| w -q \| + \alpha \;\;\;\;\;  \forall p\in
  \sigma\;\; {\rm  and}\;\; q\in L\setminus
  \sigma.$$
\end{definition}

Note that a $0$-witness is simply a witness.

% \begin{definition}[Protection of $\alpha$-witnessed simplices]
% If $w$ is an $\alpha$-witness of $\sigma$ and if $\sigma$ is
% $\delta$-protected at $w$, we say that $\sigma$ is $(\delta, \alpha)$-protected.
% \end{definition}


% % We remark that,  if $\alpha<0$, then $\sigma$ is
% % $(-\alpha$)-protected. 
% %Moreover, we have

% \begin{lemma}
% If all the simplices of $W^{\alpha}(L,W)$ are $(\delta,\alpha)$-protected with
% $\delta \geq \alpha$, then  $W^{\alpha}(L,W)=\wit (L,W)$.
% \label{lem-neg-protect}
% \end{lemma}

% \begin{proof}
% Plainly, $\wit (L,W) \subseteq \wit^{\alpha} (L,W)$. Consider the
% opposite inclusion and let $\sigma\in \wit^{\alpha}(L,W)$. If $\sigma$
% is
% $(\delta, \alpha)$-protected, there exists $w$ such that
% \[ \| w-p\| + \delta \leq \| w-q\| +\alpha \;\;\;\;\;
%   \forall p\in \sigma \;\; {\rm and} \;\; q\in L\setminus \sigma .\]
% It follows that $\sigma\in \wit(L,W)$ % and $\wit^{\alpha}(L,W)=\wit (L,W)$
% for all $\alpha \leq \delta$.
% \end{proof}

\begin{definition}[$\alpha$-witness complex]
  The $\alpha$-witness complex $\wit^{\alpha}  (L,W)$ is the complex consisting of all
  simplexes $\sigma$ such that for any simplex $\tau\subseteq
  \sigma$, $\tau$ has an $\alpha$-witness in $W$.
\end{definition}

\begin{lemma}
\label{lem-relaxed-wit}
Let $W$ be an $\e$-sample. $\del (L) \subseteq \wit^{2\e} (L,W)$.
\end{lemma}

\begin{proof}
Let $\sigma$ be a $d$-simplex of $\del (L)$, $c$ and $r$ the center
and the radius of its circumscribing sphere. Since $W$ is an
$\e$-sample, there exists $w\in W$ such that $\| c-w\| \leq \e$.

Hence,  for any $p\in \sigma$ and $q\in L$, we have
\[ \| w-p\|
  \leq \| w-c\| + \| c-q\| \| \leq 2  \| w-c\| + \| w-q\| = \|w-q\| +
  2\e.\]
$w$ is thus a $2\e$-witness of $\sigma$.
Consider now a face $\tau\subset \sigma$.  The previous inequality
also shows that  $w$ is a $2\e$-witness of $\tau$. Since
$\sigma$ and all its faces have a $2\e$-witness, $\sigma$
belongs to $\wit^{2\e}(L,W)$.
\end{proof}

Lemmas~\ref{lem-neg-protect} and \ref{lem-relaxed-wit} lead to a variant of  Lemma~\ref{lemma-wit=del} where protection
is required on the simplices (of all dimensions) of the relaxed witness complex instead of the Delaunay
triangulation. 

\begin{corol}[Identity of witness and Delaunay complexes (2)]
\label{lemma-wit=del2}
Assume that $L$ is a $\lambda$-net and
that $W$ is an $\e$-sample. If $\wit^{2\e}(L,W)=\wit(L,W)$, then  $\wit (L,W)= \del(L)$.
\end{corol}

% \begin{proof}
% By Lemma~\ref{lem-relaxed-wit}, $\del (L) \subseteq \wit^{2\e}
% (L,W)$. Since, by hypothesis,
% $\sigma$ is $2\e$-protected, we  have by Lemma~\ref{lem-neg-protect}
% $\wit ^{2\e}(L,W)=\wit (L,W)$. The other inclusion is provided by Corollary~\ref{lemma-wit-in-del}.
% \end{proof}





% \section{Core-sets for the witness complex}

% In this section, we introduce a variant of the witness complex which
% could be computed faster since it requires a small subset of witnesses.
% As usual, $\sigma$ denotes a $d$-simplex,  $c_{\sigma}$ its
% circumcenter, $r_{\sigma}$ its circumradius, %$\Delta_{\sigma}$ its
%                                 %diameter 
% and $\Theta_{\sigma}$ its thickness.


% % We call $\alpha$-relaxed Delaunay complex the simplicial
% % complex whose $d$-simplices are the subsets of $d+1$ points of
% % $L$ that ha\-ve an $\alpha$-center. We denote it by $\del^{\alpha} (L,W)$. 

% \begin{definition}[Core-set]
%  We call a core-set for  $W$ and $L$ a set $K$ of pseudo-centers $w_{\sigma}$ for
%  all  $\sigma\subset L$. We call core-witness complex $\twit
%  (L,K)=\wit (L,K)$.
% \end{definition}



% % \begin{definition}[Core witness complex]
% %  We call core witness complex  of $L$ and $W$ the witness complex
% %  $\twit (L,W) = \wit  (L, K)$.
% % %complex consisting of all simplices $\sigma$ such that for any simplex
% % %$\tau\subseteq \sigma$, $w_{\tau}$ is a witness of $\tau$.
% % %We denote it by $\twit (L,W)$.
% % \end{definition}

% % Observe
% % that the definition of $\del (L,W)$ does not require
% % that the simplices of dimensions lower than $d$ are witnessed, which
% % is an advantage over the witness complex. 


% % \begin{lemma}
% % If the simplices of $\del(L,W)$ are $\delta$-protected, then $\del
% % (L,W) = \del(L)$. %\subseteq \wit (L,W)$.
% % \end{lemma}

% % \begin{proof}
% % Let $\sigma$ be a $d$-simplex of $\del (L,W)$, $w$ an $\e$-center of
% % $\sigma$, $c$ the circumcenter of $\sigma$. In addition let $r$ be the
% % circumradius of $\sigma$, $p=\argmax_{x\in \sigma}\|
% % w-x\|$ and $\tilde{r}= \| w-p\|$. Since $W$ is an $\e$-sample, we have
% % \[ \tilde{r} = \| w-p\| \leq \| c-p\| + \| c-w\| \leq r+\e \]
% % and, for any $q\in L$, using the empty ball property,
% % \[ r \leq \| q-c\|  \leq \| q-w\| + \| w-c\|  \leq \| q-w\|
% % + \e.\]
% % We conclude from the two inequalities that $\| q-w\| \geq \tilde{r} - 2\e$
% % for all $q\in L$. Hence, if $\sigma$ is $2\e$-protected, then 
% % \end{proof}

% We state now the main result of this section.

% \begin{lemma}[Identity of  core witness and Delaunay complexes]
% \label{lemma-identity}
% Let $W$ be an $\e$-sample and $L$ a $\lambda$-net. If all the $d$-simplices of $\del (L)$
% are $\Theta_0$-thick and $\delta$-protected with $\delta \geq
% \frac{16d\e}{\Theta_0- 2\e/\lambda}$,
% %\frac{16d\e}{\Theta_0}$, 
% then $\twit (L,W) = \del(L)$. %\subseteq \wit (L,W)$.
% \end{lemma}

% \begin{proof}
% 1. Let $W^+= W\cup K$. It follows from the definition of the core-witness complex, Lemma~\ref{lemma-easy} and
% Corollary~\ref{lemma-wit-in-del} that  $\twit (L,W) = \wit (L,K) \subseteq \wit (L,W^+) \subseteq \del (L)$.

% 2. We now prove that $\del (L)\subseteq \twit (L,W)$.  Let
%   $\sigma$ be a simplex of $\del (L)$. % For convenience, we write
%   % $c$, $r$, $w$ instead of $c_{\sigma}$, $r_{\sigma}$  and
%   % $w_{\sigma}$.
%    Let   $p=\argmax_{x\in \sigma}\| w_{\sigma}-x\|$ and $\tilde{r}_{\sigma}= \| w_{\sigma}-p\|$. 
% We now apply Lemma~\ref{lemma-wsigma}.  % Observe that the hypotheses of
% % the lemma are satisfied. Indeed, the first hypothesis holds since $L$
% % is a $\lambda$-sample and $\sigma$ is a Delaunay simplex; the second hypothesis holds since $L$
% % is a $\lambda$-net;  and the third hypothesis holds with $\alpha=
% % \frac{2\e}{\Theta_0} $ by Lemma~\ref{lemma-wsigma}.
% %We have
% \[ \tilde{r} _{\sigma}= \| w_{\sigma}-p\| \leq \| c_{\sigma}-p\| + \|
% c_{\sigma}-w_{\sigma}\| \leq r_{\sigma}+ 
% \frac{2\e}{\Theta_0- 2\e/\lambda},\]
% %\frac{2\e}{\Theta_0} \]
% and, since $\sigma$ is $\frac{\delta}{4d}$-protected (Lemma~\ref{lemma-protection}), we also have for all $q\in
% L\setminus \sigma$
% % \[ \| q-w_{\sigma}\|  \geq \| q-c_{\sigma}\| - \| w_{\sigma}-c\|  \geq r_{\sigma} + \delta -\frac{2\e}{\Theta_0}
% % \geq \tilde{r}_{\sigma} +\delta -\frac{4\e}{\Theta_0} \> \tilde{r}_{\sigma}. \]
% \[ \| q-w_{\sigma}\|  \geq \| q-c_{\sigma}\| - \|
% w_{\sigma}-c_{\sigma}\|  \geq r_{\sigma} + \frac{\delta}{4d}
% -\frac{2\e}{\Theta_0- 2\e/\lambda}
% \geq \tilde{r}_{\sigma} +\frac{\delta}{4d} -\frac{4\e}{\Theta_0- 2\e/\lambda}.\]
% %\frac{4\e}{\Theta_0} \geq \tilde{r}_{\sigma}. \]
% We conclude that $\sigma$ is witnessed by
% $w_{\sigma}$ when $\delta \geq \frac{16d\e}{\Theta_0- 2\e/\lambda}$.
% %
% % We show now that any face $\tau\subset \sigma$ is also witnessed by
% % some $w_{\tau}\in W$.
% % By Lemma~\ref{lemma-protection}, $\tau$ is $\frac{\delta}{4d}$-protected. Write $c_{\tau}$
% % and $r_{\tau}$ for the center and radius of the protection ball. Hence we
% % have as above
% % \[ \forall p\in \tau : \tilde{r}_{\tau}= \| w_{\tau}-p\| \leq \| c_{\tau}-p\| + \| c_{\tau}-w_{\tau}\| \leq r_{\tau}+ \frac{2\e}{\Theta_0} \]
% % Take now for $w_{\tau}$ any point in $W\cap B(c_{\tau},\e)$. For any $q\in
% % L\setminus \tau$, we have
% % \[ \| q-w_{\tau}\|  \geq \| q-c_{\tau}\| - \| w_{\tau}-c_{\tau}\|  \geq r_{\tau} + \frac{\delta}{4d} -\frac{2\e}{\Theta_0}
% % \geq \tilde{r}_{\tau} +\delta -\frac{4\e}{\Theta_0} \geq \tilde{r}_{\tau}, \]
% % if we take now $\delta \geq \frac{16d}{\Theta_0}$.
% % We conclude from the two inequalities that $\tau$ is witnessed by $w_{\tau}$.
% %
% % Moreover, we have
% % \[ \| w_{\tau}-w_{\sigma}\| \leq \| w_{\tau}-c_{\tau}\| + \|
% % c_{\tau}-c_{\sigma}\| + \| c_{\sigma}-w_{\sigma}\| \leq 2\e + ...\]
% %
% % We conclude that $\sigma\in \del (L,W)$.
% \end{proof}

% \paragraph{Remark 1.} By Lemma~\ref{lem-relaxed-wit}, we know that
% $\del (L)\subseteq \wit^{2\alpha}(L,W)$. Hence a sufficient condition
% for the Identity lemma to hold is that all the $d$-simplices of
% $\wit^{2\alpha}(L,W)$ are $\delta$-protected.



\subsection{Relaxed Delaunay complex}

Lemma~\ref{lemma-wit=del2} provides a way to compute $\del (L)$ from
$\wit (L,W)$ provided that we can guarantee some protection on the
simplices of $\wit^{2\e}(L,W)$.  We have somehow trade algebraic
effort for time complexity. 
The naive computation of the (relaxed or not relaxed) witness complex
takes time $O(n^2+ wd)$, which is not good for big $w$.  The goal is now to reduce the complexity on $w$.

\begin{definition}[$\alpha$-Delaunay center]
An $\alpha$-Delaunay center for a simplex $\sigma\subset L$ is a
point $x$ that satisfies
\[ \| p-x\| \leq \| q-x\| + \alpha \;\;\; \forall p\in \sigma \;\;
{\rm and } \;\; q\in L.\]
A simplex that admits an $\alpha$-Delaunay center is called an
$\alpha$-Delaunay simplex.
\end{definition}

Note that a $0$-Delaunay simplex is just a Delaunay simplex.


\begin{definition}[$\alpha$-Delaunay complex]
The collection of all  $\alpha$-Delaunay 
$d$-simplices and their faces constitutes the
$\alpha$-Delaunay complex $\del^{\alpha}(L)$.
 If the
  $\alpha$-Delaunay centers associated to the $d$-simplices can be found in $W$, then the complex is called the
  $\alpha$-Delaunay complex restricted to $W$ and is denoted by $\del^{\alpha}(L,W)$.
\end{definition}

\paragraph{Remark.} We only consider $d$-simplices.


\begin{definition}[Protection of $\alpha$-Delaunay simplices]
If $c$ is an $\alpha$-Delaunay center of $\sigma$ and if $\sigma$ is
$\delta$-protected at $c$, we say that $\sigma$ is $(\delta, \alpha)$-protected.
\end{definition}


de Silva has proved that $\wit^{\alpha}(L,\R^d)=
\del^{\alpha}(L)$, which implies that
$\wit^{\alpha}(L,W) \subseteq \del^{\alpha}(L)$.
We further prove



\begin{lemma}[Identity of relaxed and non-relaxed Delaunay complexes]
\label{lemma-identity3}
Let $W$ be an $\e$-sample and $L$ a $\lambda$-net. If all the $d$-simplices of $\del^{\alpha} (L,W)$
are $\Theta_0$-thick and $\delta$-protected with $\delta \geq
\max(2\e, \frac{2\alpha}{\Theta_0- \alpha/\lambda})$,
then $\del^{\alpha}(L,W)  = \del(L)$.
\end{lemma}

\begin{proof}
% We have $\del (L) \subseteq \wit (L,\R^d)\subseteq
% \wit^{\alpha}(L,\R^d)$.
% By Lemma~\ref{lemma-wita-dela},
% $\wit^{\alpha}(L,W) \subseteq \del^{\alpha}(L)$.
%
\paragraph{1. $\del (L)\subseteq \del^{\alpha}(L,W)$. }
Let $\sigma \subset \del (L)$. Since $W$ is an $\e$-sample, there
exists $w\in W$ such that $\| c_{\sigma}-w\|\leq \e$, where
$c_{\sigma}$ denotes a circumcenter of $\sigma$. Hence, for all
$p\in \sigma$ and $q\in L$, we have
\begin{eqnarray*}
 \| p-w\| & \leq & \| p-c_{\sigma}\| + \| c_{\sigma}-w\| \\
& \leq & \| q-c_{\sigma}\| + \| c_{\sigma}-w\| \\ 
& \leq & \| q-w\| + 2  \| c_{\sigma}-w\| \\
& \leq & \| q-w\| +2\e
\end{eqnarray*}
Hence, $\sigma\in \del^{2\e}(L,W)$.

\paragraph{2. $\del^{\alpha}(L,W)\subseteq \del (L)$.}
Let $\sigma$ be a $d$-simplex of $\del^{\alpha} (L,W)$. Hence $\sigma$ has an
$\alpha$-Delaunay center $c$ and, by Corollary~\ref{lemma-stability2}, we have
\[ \| c_{\sigma}-c\| \leq \beta=
\frac{\alpha}{\Theta_{\sigma}-\alpha/\lambda}, \] 
 where
$c_{\sigma}$ denotes the circumcenter of $\sigma$. Moreover, since
$\sigma$ is $\delta$-protected at $c$,
\[ \| c-q\| > \| c-p\| + \delta,\;\;  \;\; \forall p\in \sigma \;\; {\rm
  and}\;\; \forall q\in L\setminus \sigma.\]
We deduce from the above inequalities that for all $p\in \sigma$ and $q\in L\setminus \sigma$,
\begin{eqnarray*}
\| c_{\sigma}-q\| & \geq &  \| c-q\|  - \| c_{\sigma}-c\| \\
& \geq & \| c-p\| + \delta - \beta \\
& \geq & \| c_{\sigma}-p\| + \delta -2\beta
\end{eqnarray*}
Hence $\sigma\in \del(L)$ if $\delta\geq 2\beta$.
%
% and the radius of its circumscribing sphere. Since $W$ is an
% $\e$-sample, there exists $w\in W$ such that $\| c-w\| \leq \e$.
%
% Hence,  for any $p\in \sigma$ and $q\in L\setminus
%   \sigma$, we have
% \[ \| w-p\|
%   \leq \| w-c\| + \| c-q\| \| \leq 2  \| w-c\| + \| w-q\| = \|w-q\| +
%   2\e.\]
% Hence, $w$ is a $2\e$-relaxed witness of $\sigma$.
% Consider now a face $\tau\subset \sigma$.  The previous inequality
% plainly holds for any $p\in \tau$ and any $q\in L\setminus
%   \sigma$.
% Moreover, for any $p\in \tau$ and $q\in \sigma$, we have
% \[ \| w-p\|
%   \leq \| w-c\| + \| c-q\| \| \leq 2  \| w-c\| + \| w-q\| = \|w-q\| +
%   2\e.\]
% It follows that $w$ is a $2\e$-relaxed witness of $\tau$. Since
% $\sigma$ and all its faces have a $2\e$-relaxed witness, $\sigma$
% belongs to $\wit^{\alpha}(L,W)$.
\end{proof}

\paragraph{Remark.} By a result in~\cite{boissonnat:hal-00807050},
$\delta$-protection implies thickness and we have~:
$\Theta_0\geq \frac{\sqrt{3}\nu_0^2}{4} $ where
$\nu_0= \frac{\delta}{\lambda}$. The condition on the protection
$\delta$ in the lemma becomes asymptotically: $\delta = \Omega (\alpha^{\frac{1}{3}}\lambda^{\frac{2}{3}})$.
% Hence,  a sufficient condition for the lemma to hold is $\delta^3 \geq
% \frac{16}{\sqrt{3}}\; \e\lambda^2$. % If $\nu_0$ is a constant, a
% % sufficient condition is $\delta \geq \frac{16\nu_0^2}{\sqrt{3}}\;
% % \e$, 
% However this condition is more demanding than the analogous condition in
% Lemma~\ref{lemma-wit=del} when $\e = o(\lambda)$ (which is the usual case). To allow for
% a smaller protection $\delta=O (\e)$, we need to enforce that $\Theta_0$
% is a constant.

\section{Algorithms}

\subsection{Computing $2\e$-centers}

The next lemma will provide a mean to compute an
$\alpha$-center of a $d$-simplex. 

\begin{definition}[Pseudo-centers]
$L$ and $W$ are given as usual.
The pseudo-center of a $d$-simplex $\sigma\subset L$ is the point 
$$w_{\sigma} = \arg\min_{x\in W}\max_{p,q\in \sigma} | \| x-p\| -\|x-q\| |. $$
% A pseudo-center of a $k$-simplex %incident to at least $d-k+1$
% %$d$-simplices
% is the barycenter of the pseudo-centers of any subset
% of $d-k+1$ $d$-simplices incident to $\sigma$.
\end{definition}

% \paragraph{Remark.} The barycenters are not necessarily elements of
% $W$. This is no problem in Euclidean space. It might be possible to
% replace a barycenter by its closest witness. This would lead to slightly
% worse bounds but the overall approach will still be valid.


\begin{lemma}
Assume that $W$ is an $\e$-sample and let $\sigma$ be a $d$-simplex
such that $R_{\sigma}\leq \lambda \leq \Delta_{\sigma}$. Then, any
pseudo-center $w_{\sigma}$ is a
$2\e$-center of $\sigma$ and we have  $\|
w_{\sigma}-c_{\sigma}\| \leq \frac{2\e}{\Theta_{\sigma}-
  2\e/\lambda}$, where $c_{\sigma}$ is the circumcenter of
$\sigma$. %defined as in Lemma~\ref{lemma-stability}. 
If $\sigma$ is $\sqrt{\e}$-protected, then   $\|
w_{\sigma}-c_{\sigma}\| = O(\lambda\sqrt{\e})$.
\label{lemma-wsigma}
\end{lemma}

\begin{proof}
%Let $c_{\sigma}$ be the circumcenter defined in Lemma~\ref{lemma-stability}.
Any point in $B(c_{\sigma}, \e)$ is a $2\e$-center of
$\sigma$. Indeed, if $p,q\in \sigma$ and assuming wlog that $\| w-p\|
\geq \| w-q\|$, we have for any $x\in B(c_{\sigma}, \e)$,
\[  \| x-p\| -\| x-q\|  \leq \| c_{\sigma}-p\| + \| x-c_{\sigma}\| - \| c_{\sigma}-q\| + \| x-c_{\sigma}\|
\leq 2\e.\]
Since
$W$ is an $\e$-sample, there exists $w\in W\cap B(c_{\sigma},\e)$
which is a $2\e$-center. It follows from the definition of $w_{\sigma}$
that $w_{\sigma}$ is also a
$2\e$-center of $\sigma$. 

To bound $\| w_{\sigma}-c_{\sigma}\|$, we first consider the case of a
$d$-simplex and apply Corollary~\ref{lemma-stability2}to $\sigma$ with
$\tilde{c}=w_{\sigma}$. We get $\| w_{\sigma}-c_{\sigma}\| \leq
\frac{2\e}{\Theta_{\sigma}- 2\e/\lambda}$.
% Consider now the case of a $k$-simplex $\sigma$. Let $\tau_i$, $i\leq
% d-k+1$, be the $d$-cofaces of $\sigma$ used to define $c_{\sigma}$ (in
% Lemma~\ref{lemma-stability}) and $w_{\sigma}$. By definition,
% \[ w_{\sigma}= \frac{1}{d-k+1}\, \sum_{i=1}^{d-k+1} \, w_{\tau_i}\]
% and we have 
% \[ \| w_{\sigma}-c_{\sigma}\| \leq \frac{1}{d-k+1}\sum_{i=1}^{d-k+1} \|
% w_{\tau_i}-c_{\tau_i}\| \leq \frac{2\e}{\Theta_{\sigma}-
% 2\e/\lambda}.\]
The last statement follows from the remark at the end of section~\ref{sec:relaxed}.
\end{proof}



We show that computing $w_{\sigma}$ reduces to computing the
point of some finite set that is closest to a hyperplane.

% This is an interesting algorithmic question !  Here is a simple
% procedure.

% % Let $\sigma = [p_0,...,p_d]$ and 
% % consider the function $f(x) = \frac{1}{d}\, \sum_{k=0}^d \| x-p_i\|
% % ^2$.

% % $f(x)$ is 

% \begin{enumerate}
% \item[\framebox{{\bf Compute an $\alpha$-center of $\sigma$}}]
% \item Start at $x$ and let $p$ be the vertex of $\sigma$ closest to
%   $x$;
% \item Let $x'$ be the projection of  $x$ onto the bisector of $p$ and the vertex of $\sigma$
%   furthest from $x$;
% \item if $x'$ is not an $\alpha$-center of $\sigma$, then do $x:=x'$ and go to 2;
% \item else return $x'$.
% \end{enumerate}

% Interestingly, projecting $x$ onto the bisector of two vertices
% involves only  polynomials of degree two in the coordinates of the
% vertices and $x$.


We consider the squared-distance case, i.e. $$w_{\sigma}=
\arg\min_{x\in W} \max_{i,j\in \sigma}\, |(x-p_i)^2-(x-p_j)^2|.$$

Let $p$ and $q$ be two vertices of $\sigma$ such that $\max_{i,j\in
  \sigma}\, |(x-p_i)^2-(x-p_j)^2| = (x-p)^2- (x-q)^2$.
We have 
\begin{eqnarray*}
\arg\min_x ((x-p)^2- (x-q)^2) &=& \arg\min_x (-2(p-q)\cdot x
+p^2-q^2)\\
&=& \arg\min (-u_{pq}\cdot (x
-\frac{p+q}{2})),
\end{eqnarray*}
where $u_{pq}$ denotes the unit vector parallel to $p-q$.
Hence the point of $W$
that minimizes $|(x-p)^2- (x-q)^2|$ is the point of $W$ that is closest
to the bisector of $p$ and $q$.

Let $W_{p,q}$ be the subset of $W$ consisting of the points with $p$
as their furthest vertex in $\sigma$ and $q$ as their closest vertex. Observe that $W_{p,q}= W\cap Z_{p,q}$
where $Z_{p,q}$ is the subset of $\R^d$ consisting of the  points that
have $p$ as a furthest vertex in $\sigma$ and $q$ as a closest vertex.
Equivalently, $Z_{p,q}$ is the intersection of $2d$ halfspaces
\[ Z_{p,q}= \left( \bigcap_{r\in\sigma, r\neq p} H_{p,r}^r \right)
\;\left( \bigcap_
  {s\in\sigma, s\neq q} H_{q,s}^q \right) \]
where $H_{x,y}^x$ denotes the halfspace bounded by $H_{x,y}$ that
contains $x$. 
Let  $w_{p,q}= \arg\min_{x\in W_{p,q}}\, |(x-p)^2- (x-q)^2|$.
We then obtain  $w_{\sigma} =\arg\min_{p,q}w_{p,q}$.

The computation of $w_{\sigma}$ can be implemented efficiently using
an octree. 
As suggested above, we compute $w_{p,q}$ for all pairs of vertices $p,q\in \sigma$.
We go down in a child node of a node in the octree only if the two
following conditions are satisfied :
\begin{enumerate}
\item the child node intersects   $Z_{p,q}$
\item it intersects $H_{p,q}$ or it has a corner closer to $H_{p,q}$   than the best current $w_{p,q}$.
\end{enumerate}
 Both cases can be decided by evaluating the sign or the power of a corner with
 respect to some  bisecting hyperplane or comparing two such
 powers. These are algebraic operations of degree 2.


\subsection{Computing Delaunay triangulations}

In this section, we describe an algorithm to compute $\del (L)$ from
$\del^{\alpha} (L,W)$. The algorithm relies on Lemma~\ref{lemma-identity3} and  perturbing
$L$ so as to enforce the necessary protection on the $d$-simplices.

% \begin{enumerate}
% \item[\framebox{{\bf Algorithm 1}}]
% \item[{\bf input}] $L, W,  \delta$
% \item[{\bf output}] $\del (L)$
% \item Compute $\wit (L,W)$
% \item For each point $p\in L$, perturb $p$ so that 
% \begin{enumerate}
% \item for all $q\in L$ inserted before $p$ and all simplices $\sigma
%   \in \str^- (q)$,  $\sigma \in \str^+(q)$
% \item $\str^+ (p)$ is a pseudo $d$-manifold
% \end{enumerate}
% \item[] // $-$ ($+$) refers to   before (after) perturbing $p$
% \item[] // the stars are wrt $\wit (L,W)$)
% \end{enumerate}


% \begin{enumerate}
% \item[\framebox{{\bf Algorithm 2}}]
% \item[{\bf input}] $L,W, \Theta_0$
% \item[{\bf output}] $\del (L)$
% \item Initialize $K$ with a few witnesses
% \item Compute $\wit (L,K)$
% \item {\bf while} there exits a big simplex, insert its pseudo-center
%   in $K$   and update $\wit (L,K)$ 
% \item {\bf while} there exists a vertex $p$ incident to a non $\Theta_0$-thick simplex, perturb $p$
% \item {\bf while} the link of some vertex $p$ of $\wit (L,K)$ is not a pseudo
%   $(d-1)$-manifold, perturb $p$. 
% \end{enumerate}

\newpage

\begin{enumerate}
\item[{\bf Algorithm Delaunay construction}]
\item[{\bf input}] $L,W$, $\e$ the sampling radius of $W$, $\alpha =
  \Theta (\lambda\,\sqrt{\e})$
%an upper  bound $\alpha >0$  on the   CC approximation 
%\item[//] We assume that we can make $W$ as dense as needed   //
%\item[//] Pseudo-centers are always $\alpha$-centers //
\item[{\bf output}] $\del (L)$
\item Initialize $L_c$ with any
  set $\sigma$ of $d+1$  points of $L$; compute $\del (L_c)=\{
\sigma\}$ and the
  pseudo-center $w_{\sigma}$ of $\sigma$
% and an upper bound $\e$ on the error
%  $\max_{p,q\in \sigma} | \| w_{\sigma}-p\| - \| w_{\sigma}-q\|   |$. 

\item[//] $B_{\sigma}^+$ and $B_{\sigma}^-$ denote the balls centered
  at $w_{\sigma}$ of radii respectively $\max_{x\in \sigma} \|
  w_{\sigma}-x\| + \alpha$ and   $\min_{y\in \sigma} \|
  w_{\sigma}-y\| -\alpha$ //
\item {\bf while} $L\setminus L_c \neq \emptyset$ {\bf do}
 
 \begin{enumerate}
\item pick $p\in L \setminus L_c \neq \emptyset$
\item {\bf while}  $p$ lies in $B_{\sigma}^+\setminus B_{\sigma}^-$ for
  some $\sigma \in \del (L_c)$ {\bf do} perturb $p$

\item Look for the conflicting complex $C_p$, i.e. the
    simplices $\sigma \in \del (L_c)$  such that $B_{\sigma}^+$ contains
    $p$ 
    \item Construct the star $S_p$ of $p$ by joining $p$ to the faces of the
      boundary of the conflicting complex
          \item Associate to each $d$-simplex in the star of $p$ its
          pseudo-center,
update $\del (L_c):= \left( \del(L_c)\setminus C_p\right)  \cup S_p$, and $L_c:=L_c\cup \{ p\}$ 

\end{enumerate}
    \item {\bf end.}
    \end{enumerate}
    
\paragraph{Remark 1.} The algorithm is a simple variant of the standard
incremental algorithm to compute $\del (L)$. Its main advantage is to
require only degree 2 predicates. Although there is no explicit
reference to the thickness of the simplices, $\e$  will
depend on the thickness.

\paragraph{Remark 2.} $\e$ can be defined locally and attached at each simplex.


\subsection*{Correctness of the algorithm}

\begin{lemma}
\label{lemma-BBB}
For any simplex $\sigma\in \del(L_c)$, $B_{\sigma}^- \subseteq
B_{\sigma} \subseteq B_{\sigma}^+$.
\end{lemma}

\begin{proof}
Follows from  $\| w_{\sigma}-c_{\sigma}\| =O(\lambda\, \sqrt{\e})$ (Lemma~\ref{lemma-wsigma}).
\end{proof}


\begin{lemma}
$\del (L_c \cup \{ p\})= \left( \del(L_c)\setminus C_p\right)  \cup S_p$.
\end{lemma}

\begin{proof}
It follows from Lemma~\ref{lemma-BBB} and the fact that $p\not\in B_{\sigma}^+\setminus B_{\sigma}^-$ (line
2(b)) that $p\in B_{\sigma}$ iff $p\in B_{\sigma}^+$ (or equivalently
in $B_{\sigma}^-$).
\end{proof}

\begin{lemma}
Assume that Step 2 of the algorithm finds a perturbation such that $p$
lies outside all hoops $B_{\sigma}^+ \setminus B_{\sigma}^-$ for all
$\sigma\in \del (L_c)$. Then, for all points $q$ inserted before $p$ and
for $q=p$, we have $\str ^+ (q, \wit (L,W))= \str^+ (q, \del (L))$.
\end{lemma}

\begin{proof}
The proof is by induction. We first observe that 
$$\str^+ (p, \wit (L,W)) = \str^+ (p,\del^+ (L))$$
since $\str^+ (p, \wit (L,W))$ is a pseudo $d$-manifold by the second rule
of the algorithm, Corollary~\ref{lemma-wit-in-del}  and Lemma~\ref{lemma-pseudoman}.

Assume now that the result holds for any
$q<p$. An easy induction shows that,  immediately before perturbing $p$, 
$\str^- (q, \wit (L,W))= \str^-  (q, \del (L))$ for all $q<p$. Moreover,
 by the first rule of the algorithm, 
$$\str^- (q, \wit
(L,W))=\str^+ (q, \wit (L,W)).$$
Since, by the second rule, $\str^+ (q, \wit (L,W))$ is a
pseudo $d$-manifold, we have
$$\str^+ (q, \wit (L,W)) = \str^+ (q,\del^+ (L))$$
by Corollary~\ref{lemma-wit-in-del}  and Lemma~\ref{lemma-pseudoman}.
\end{proof}

\begin{lemma}
At each step, there exists a perturbation (line 2(b)) that moves $p$ outside all
hoops  $B_{\sigma}^+ \setminus B_{\sigma}^-$ for all $\sigma\in \del (L_c)$. 
\end{lemma}

\begin{proof}



\end{proof}


The following lemma is no longer used.

\begin{lemma}[Pseudo-manifold]
\label{lemma-pseudoman}
Let $C$  be a $d$-subcomplex of a $d$-triangulation $T$ of $\R^d$. Let
 $p$ be  a vertex of $C$. If each $(d-2)$-face of $\link (p,C)$ is a face of exactly two
  $(d-1)$-simplices of $\link (p,C)$, then $C=T$.
\end{lemma}

\begin{proof}
Since $T$ is a triangulation (with no
boundary since $\R^d$ has been compactified), $\link(p,T)$ is a PL
topological $(d-1)$-sphere. In particular, each $(d-2)$-face of $\link (p,T)$ is a face of exactly two
  $(d-1)$-simplices of $\link (p,T)$. Any strict $d$-subcomplex of $T$
  must have a non empty boundary, contradicting
the assumption of the lemma.
\end{proof}

% \subsection{Computing the witness complex}

% By duality, if  we want to compute the $(k+1)$-simplices of $\wit
% (L)$, we consider the $k$-th level in the arrangement of the polar
% (tangent) hyperplanes at the lifted points. Any point in the interior
% of a cell $C$ of this level set is the image of an open ball that contains exactly $k+1$
% specific points of $L$. Such a simplex is witnessed if the vertical
% projection of $C$ contains a witness.

% Hence computing the $k$-th skeleton of the witness complex can be done
% as follows.

% \begin{enumerate}
% \item For $i=1$ to $k$
% \begin{enumerate}
% \item Compute the $k$-th order Voronoi diagram of $L$
% \item Locate each witness in this diagram and enumerate the cells with
%   a witness
% \end{enumerate}
% \end{enumerate}

% The time complexity is $O(k^{d/2} |L|^{d/2}+ $

\section{Extensions}

\begin{enumerate}
\item Weighted distances 
\item Spaces of constant sectional curvature
\item Witness complex of spheres
\end{enumerate}

\section{Tangential complex}

The same mechanism can be used to remove inconsistencies between stars.

%\bibliographystyle{plain}
\bibliographystyle{alpha}
%\bibliography{delrefs}
%\bibliography{abiblio}
\bibliography{../Arijit-Ramsay-Steve/biblio}


\end{document}